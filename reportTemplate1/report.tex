%%% Poster for ASE 2009 by Martin Weiglhofer

\documentclass[final,hyperref={pdfpagelabels=false}]{beamer} 

\mode<presentation> {  
  \usetheme{TUGraz}    
}

\usepackage[english]{babel}
\usepackage[latin1]{inputenc}
\usepackage{amsmath,amsthm, amssymb, latexsym}
\usepackage{epsfig}
\usepackage{multirow}
\usepackage{pifont}
\usepackage{algpseudocode}
\usepackage{colortbl}
\usepackage{booktabs}

\newcommand{\Break}{\State{\textbf{break}}}

\newcommand{\no}{\color{tugorange} \ding{55}}
\newcommand{\yes}{\color{tuggreen} \ding{51}}

\usepackage{tikz}
\usepackage{pgflibraryshapes}
\usetikzlibrary{arrows}
\usetikzlibrary{shapes}
\usetikzlibrary{automata}
\tikzstyle{model_node}=[circle,draw=black,fill=tuggreen!50,minimum size=50pt]
\tikzstyle{tool}=[draw,double,rounded corners,inner sep=10pt]
\tikzstyle{aut_node}=[circle,draw=black,fill=black,inner sep=0pt,minimum size=7pt,font=\footnotesize]

\usefonttheme[onlymath]{serif}
\boldmath
\usepackage[orientation=portrait,size=a0,scale=1.4,debug]{beamerposter}                       


\pgfdeclareimage[height=4cm]{institute-logo}{images/ist_logo}
\logoleft{\pgfuseimage{institute-logo}\vspace*{-0.5cm}}

\pgfdeclareimage[height=4cm]{university-logo}{images/logo}
\logoright{\pgfuseimage{university-logo}}

\institute[Institute for Software Technology] 
{
  Institute for Software Technology - Graz University of Technology
}

\date[ASE 2009] 
{Automated Software Engineering 2009}

\subject{ASE 2009}

\title{Using Spectrum-based Fault Localization for Test Case Grouping}
\author[Weiglhofer, Fraser \& Wotawa]{Martin Weiglhofer, Gordon
  Fraser, and Franz Wotawa}
\mail{\{weiglhofer,fraser,wotawa\}@ist.tugraz.at}
\webpage{http://www.ist.tugraz.at/}

\begin{document}

\begin{frame}

\centering
\begin{block}{Problem}

\begin{minipage}{0.64\paperwidth}
\begin{itemize}
  \item Automated model-based test case generation allows the generation
    of arbitrary numbers of different test cases.
  \item In model-based black-box testing many test cases often reveal
    the same misbehavior (see table).
  \item Under this light test result analysis become a tedious and time
    consuming task. 
  \item Our goal: \\\textbf{ Group test cases such that test cases of
      one group most likely detected the same failure.}
\end{itemize}
\end{minipage}
\qquad
\begin{minipage}{0.29\paperwidth}
\begin{center}
  \begin{tabular}{cccccc}
    \toprule
    \multirow{2}{2.3cm}{Test Suite} & \multirow{2}{*}{Impl.} & No. Test &
    \multicolumn{2}{c}{Verdicts} & \multirow{2}{4cm}{\centering Detected Failures} \\
\cmidrule{4-5}
                                &                        & Cases    &      Pass & Fail \\
    \midrule
    \multirow{2}{*}{A} & Commercial  & 6644 & 1700 & \cellcolor{tugorange} 4944 & \cellcolor{tugorange} 9 \\
                       & Open Source & 6644 & 4587 & \cellcolor{tugorange} 2057 & \cellcolor{tugorange} 4 \\
    \multirow{2}{*}{B} & Commercial  &  500 &  207 & \cellcolor{tugorange} 293  & \cellcolor{tugorange} 3 \\
                       & Open Source &  500 &  237 & \cellcolor{tugorange} 263  & \cellcolor{tugorange} 5 \\
    \bottomrule
  \end{tabular}
\end{center}
\end{minipage}

\end{block}

\begin{block}{Our Approach}
  We use spectrum-based fault localization (SFL,
  e.g. see~\cite{abreu2007accuracy}) on the level of the
  specification. SFL calculates the fault-likelihoods for different
  blocks of the specification. Then we compute the similarity between
  test runs w.r.t. these likelihoods, i.e. blocks with a higher
  fault-likelihood have a higher impact on the similarity of test
  runs.

\vspace*{10mm}
\begin{center}
\begin{tikzpicture}[node distance=62mm]

  \begin{scope}[node distance=35mm]
    \node (mtext) at (3,2) {\centering \textbf{Model}}; 
    \node at (0.85,0.8) (model_north) {};
    \node at (2.3,0) (model_west) {};
    \node[model_node] at (0,0) (s1) {};
    \node[model_node] (s2) [below of=s1, right of=s1]{};
    \node[model_node] (s3) [above right of=s2] {};
    
    \draw[->] (s1) to [bend left] (s2);
    \draw[->] (s2) to [bend left] (s1);
    \draw[->] (s1) to [bend left] (s3);
    \draw[->] (s2) to [bend left] (s3);
    \draw[->] (s3) to [bend left] (s2);
  \end{scope}

  \node[tool] (tcgen) [below of=s2]
  { \begin{minipage}{4.5cm}
      \centering
      test case\\
      generator
    \end{minipage}
  };


  \node (testcases) [below of=tcgen,yshift=-60mm] {};
  \begin{scope}[node distance=20mm, yshift=-110mm, xshift=-45mm]
    \begin{scope}[xshift=-0.5cm,yshift=-5.2cm]
      \foreach \x/\y/\xtext/\name/\yoffset/\graph in 
      {
        2/0//$tc_1$/1.8/{
          \node[aut_node] (a0) {} {
            node[aut_node] (a1) [below of=a0,node distance=6mm] {} edge (a0) {
              node[aut_node] (a2) [below of=a1,node distance=6mm] {} edge (a1) {
                node[aut_node] (a3) [below of=a2, left of=a2,node distance=6mm] {} edge (a2){
                  node[aut_node] (a4) [below of=a2, right of=a2,node distance=6mm] {} edge (a2) {
                    node [below] at (a3.north) {\tiny \yes} {
                      node [below] at (a4.north) {\tiny \no} 
                    }
                  }
                }
              }
            }
          };
        }
        ,7/0//$tc_2$/1.8/{
          \node[aut_node] (a0) {} {
            node[aut_node] (a1) [below of=a0,node distance=6mm] {} edge (a0) {
              node[aut_node] (a2) [below of=a1,node distance=6mm] {} edge (a1) {
                node[aut_node] (a3) [below of=a2, left of=a2,node distance=6mm] {} edge (a2) {
                  node[aut_node] (a4) [below of=a2, right of=a2,node distance=6mm] {} edge (a2) {
                    node [below] at (a3.north) {\tiny \yes} {
                      node [below] at (a4.north) {\tiny \no} 
                    }
                  }
                }
              }
            }
          };
        }
        ,12/0//$tc_3$/2/{
          \node[aut_node] (a1) {} {
            node[aut_node] (a2) [below of=a1,node distance=6mm] {} edge (a1) {
              node[aut_node] (a3) [below of=a2, left of=a2,node distance=6mm] {} edge (a2) {
                node[aut_node] (a4) [below of=a2, right of=a2,node distance=6mm] {} edge (a2) {
                  node [below] at (a3.north) {\tiny \yes} {
                    node [below] at (a4.north) {\tiny \no} 
                  }
                }
              }
            }
          };
        }
        ,2/-5.8//$tc_4$/1.3/{
          \node[aut_node] (a0) {} {
            node[aut_node] (a1) [below of=a0,node distance=6mm] {} edge (a0) {
              node[aut_node] (a15) [below of=a1,node distance=6mm] {} edge (a1) {
                node[aut_node] (a2) [below of=a15,node distance=6mm] {} edge (a15) {
                  node[aut_node] (a25) [below of=a2,node distance=6mm] {} edge (a2) {
                    node[aut_node] (a3) [below of=a25, left of=a25,node distance=6mm] {} edge (a25) {
                      node[aut_node] (a4) [below of=a25, right of=a25,node distance=6mm] {} edge (a25) {
                        node [below] at (a3.north) {\tiny \yes} {
                          node [below] at (a4.north) {\tiny \no} 
                        }
                      }
                    }
                  }
                }
              }
            }
          };
        }
        ,7/-5.8//$tc_5$/1.3/{
          \node[aut_node] (a0) {} {
            node[aut_node] (a1) [below of=a0,node distance=6mm] {} edge (a0) {
              node[aut_node] (a15) [below of=a1,node distance=6mm] {} edge (a1) {
                node[aut_node] (a2) [below of=a15,node distance=6mm] {} edge (a15) {
                  node[aut_node] (a3) [below of=a2, left of=a2,node distance=6mm] {} edge (a2) {
                    node[aut_node] (a4) [below of=a2, right of=a2,node distance=6mm] {} edge (a2) {
                      node[aut_node] (a5) [below of=a3, left of=a3,node distance=6mm] {} edge (a3) {
                        node[aut_node] (a6) [below of=a3, right of=a3,node distance=6mm] {} edge (a3) {
                          node [below] at (a5.north) {\tiny \yes} {
                            node [below] at (a4.north) {\tiny \no} {
                              node [below] at (a6.north) {\tiny \no} 
                            }
                          }
                        }
                      }
                    }
                  }
                }
              }
            }
          };
        }
        ,12/-5.8//$tc_6$/2.0/{
          \node[aut_node] (a1) {} {
            node[aut_node] (a2) [below of=a1,node distance=6mm] {} edge (a1) {
              node[aut_node] (a3) [below of=a2, left of=a2,node distance=6mm] {} edge (a2) {
                node[aut_node] (a4) [below of=a2, right of=a2,node distance=6mm] {} edge (a2) {
                  node [below] at (a3.north) {\tiny \yes} {
                    node [below] at (a4.north) {\tiny \no} 
                  }
                }
              }
            }
          };
        }
      }
      {
        \draw[shift={(\x,\y)},scale=3.8] (0.75,0) -- (0,0) -- (0,-1.35) --
        (1,-1.35) -- (1,-0.25) -- (0.75,-0.25) -- (0.75,0) --
        (1,-0.25);
        \node[shift={(\x+3,\y)}] at (0,0) (n1) [anchor=north east] 
        {\begin{minipage}{1.5cm}\centering \scriptsize \name\end{minipage}};
        
        \begin{scope}[shift={(\x+1.8,\y-\yoffset)}]
          \graph
        \end{scope}
      };
    \end{scope}
    \end{scope}


  \node[tool] (tcexec) [right of=testcases,  node distance=200mm] 
  {\begin{minipage}{4.5cm}
      \centering
      test case\\
      executor
    \end{minipage}
  };

  \node[fill=gray,text=white,inner sep=1cm] (iut) [below of=tcexec,node distance=70mm] {\textbf{IUT}};

  \begin{scope}[xshift=150mm, yshift=15mm]
    \begin{scope}[xshift=-0.5cm,yshift=-5.2cm]
\foreach \x/\y/\xtext/\name/\yoffset/\graph in 
      {
        2/0/\yes/$tc_1$/1.8/{
          \node[aut_node] (a0) {} {
            node[aut_node] (a1) [below of=a0,node distance=6mm] {} edge (a0) {
              node[aut_node] (a2) [below of=a1,node distance=6mm] {} edge (a1) {
                node[aut_node] (a3) [below of=a2, left of=a2,node distance=6mm] {} edge (a2)% {
              }
            }
          };
        }
        ,7/0/\no/$tc_2$/1.8/{
          \node[aut_node] (a0) {} {
            node[aut_node] (a1) [below of=a0,node distance=6mm] {} edge (a0) {
              node[aut_node] (a2) [below of=a1,node distance=6mm] {} edge (a1) {
                  node[aut_node] (a4) [below of=a2, right of=a2,node distance=6mm] {} edge (a2) %{
              }
            }
          };
        }
        ,12/0/\no/$tc_3$/2/{
          \node[aut_node] (a1) {} {
            node[aut_node] (a2) [below of=a1,node distance=6mm] {} edge (a1) {
                node[aut_node] (a4) [below of=a2, right of=a2,node distance=6mm] {} edge (a2) % {
            }
          };
        }
        ,2/-5.8/\yes/$tc_4$/1.3/{
          \node[aut_node] (a0) {} {
            node[aut_node] (a1) [below of=a0,node distance=6mm] {} edge (a0) {
              node[aut_node] (a15) [below of=a1,node distance=6mm] {} edge (a1) {
                node[aut_node] (a2) [below of=a15,node distance=6mm] {} edge (a15) {
                  node[aut_node] (a25) [below of=a2,node distance=6mm] {} edge (a2) {
                    node[aut_node] (a3) [below of=a25, left of=a25,node distance=6mm] {} edge (a25) % {
                  }
                }
              }
            }
          };
        }
        ,7/-5.8/\no/$tc_5$/1.3/{
          \node[aut_node] (a0) {} {
            node[aut_node] (a1) [below of=a0,node distance=6mm] {} edge (a0) {
              node[aut_node] (a15) [below of=a1,node distance=6mm] {} edge (a1) {
                node[aut_node] (a2) [below of=a15,node distance=6mm] {} edge (a15) {
                    node[aut_node] (a4) [below of=a2, right of=a2,node distance=6mm] {} edge (a2) % {
                }
              }
            }
          };
        }
        ,12/-5.8/\no/$tc_6$/2.0/{
          \node[aut_node] (a1) {} {
            node[aut_node] (a2) [below of=a1,node distance=6mm] {} edge (a1) {
                node[aut_node] (a4) [below of=a2, right of=a2,node distance=6mm] {} edge (a2) % {
            }
          };
        }
      }
      {
        \draw[shift={(\x,\y)},scale=3.8] (0.75,0) -- (0,0) -- (0,-1.35) --
        (1,-1.35) -- (1,-0.25) -- (0.75,-0.25) -- (0.75,0) --
        (1,-0.25);
        \node[shift={(\x+3,\y)}] at (0,0) (n1) [anchor=north east] 
        {\begin{minipage}{1.5cm}\centering \scriptsize \name \\ \huge \xtext \end{minipage}};
        
        \begin{scope}[shift={(\x+1.8,\y-\yoffset)}]
          \graph
        \end{scope}
      };
    \end{scope}
  \end{scope}

  \node[tool] (coverage) [above of=tcexec, node distance=660] 
  {\begin{minipage}{130mm}
      \begin{center}\textbf{Model Coverage}\end{center}
      \small Derive for each executed path the hit coverage items
      (e.g. lines, conditions).
    \end{minipage}};

  \node (m4) [right of=coverage, node distance=220mm, anchor=north, yshift=30mm] 
{
\small
    \begin{tabular}{clcccccclc}
      \hline
      Executed & & \multicolumn{6}{c}{Coverage Items} & & \multirow{2}{*}{Verdict}\\
      \cline{3-8}
      Paths & & 1 & 2 & 3 & 4 & 5 & 6 & & \\
      \hline
        $tc_1$  & & 0 & 1 & 0 & 1 & 1 & 0 & & \yes \\
        $tc_2$  & & 0 & 1 & 1 & 1 & 0 & 0 & & \no \\
        $tc_3$  & & 0 & 1 & 1 & 0 & 0 & 0 & & \no \\
        $tc_4$  & & 0 & 1 & 0 & 1 & 1 & 0 & & \yes \\
        $tc_5$  & & 0 & 1 & 0 & 1 & 0 & 1 & & \no \\
        $tc_6$  & & 0 & 1 & 1 & 0 & 0 & 0 & & \no \\
      \hline
      fault- & &\multirow{2}{*}{0.0}&\multirow{2}{*}{0.8}&\multirow{2}{*}{0.8}&\multirow{2}{*}{0.5}&\multirow{2}{*}{0.0}&\multirow{2}{*}{0.5} \\
    likelihood & & \\
      \hline
    \end{tabular}
};

  \node[draw=gray] (calcOchiai) [right of=coverage, node distance=430mm, anchor=north, yshift=25mm]
{
  \begin{minipage}{165mm}
\footnotesize
$$
\frac{a_{1\text{\no}}(c)}{\sqrt{(a_{1\text{\no}}(c)+a_{0\text{\no}}(c))*(a_{1\text{\no}}(c) + a_{1\text{\yes}}(c))}}
$$
where
\begin{eqnarray*}
a_{1\text{\no}}(c) & = |\{r | x_{r,c} = 1 \wedge Verdict_{r} = \text{\no}\}| \\
a_{0\text{\no}}(c) & = |\{r | x_{r,c} = 0 \wedge Verdict_{r} = \text{\no}\}| \\
a_{1\text{\yes}}(c) & = |\{r | x_{r,c} = 1 \wedge Verdict_{r} = \text{\yes}\}| 
\end{eqnarray*}
and $x_{r,c}$ denotes the element at row $r$ and column $c$ of the
table.
  \end{minipage}
};

  \draw[->,overlay,red,ultra thick] (512mm,-70mm) to [looseness=2.0,out=0,in=220] (calcOchiai.west);

  \node[draw=gray] (calculationx) [right of=coverage, node distance=430mm, anchor=north, yshift=-90mm] 
{
\begin{minipage}{165mm}
\footnotesize

\begin{center}
    \begin{tabular}{clcc>{\columncolor{tugorange}[0pt]}c>{\columncolor{tugorange}[0pt]}cc>{\columncolor{tugorange}[0pt]}c}
      \hline
        $tc_5$  & & 0 & 1 & 0 & 1 & 0 & 1 \\
        $tc_6$  & & 0 & 1 & 1 & 0 & 0 & 0 \\
     \hline
       fault- & &\multirow{2}{*}{0.0}&\multirow{2}{*}{0.8}&\multirow{2}{*}{\cellcolor{block body.bg}0.8}&\multirow{2}{*}{\cellcolor{block body.bg}0.5}&\multirow{2}{*}{0.0}&\multirow{2}{*}{\cellcolor{block body.bg}0.5} \\
     likelihood & & \\
      \hline
    \end{tabular}
\end{center}
\vspace*{10mm}
    \begin{eqnarray*}
      s(tc_5,tc_6) &=& 1 - \frac{0.8+0.5+0.5}{\Sigma (\text{fault-likelihood})} \\
      &=& 1 - \frac{1.8}{0.8+0.8+0.5+0.5} \\
      &=& \colorbox{tuggreen}{0.305}
    \end{eqnarray*}
\end{minipage}
};

  \draw[->,overlay,red,ultra thick] (677mm,-120mm) -- (669mm,-140mm);
  \draw[->,overlay,red,ultra thick] (693mm,-120mm) -- (693mm,-140mm);
  \draw[->,overlay,red,ultra thick] (719mm,-120mm) -- (717mm,-140mm);

  \draw[->,overlay,red,ultra thick] (500mm,-190mm) to [looseness=1.0,out=0,in=200] (calculationx.west);

  \node (similarity) [below of=m4, node distance=140mm]
{
\begin{minipage}{100mm}
  \textbf{Similarity Values}\\\\
  \small
  \begin{tabular}{c|ccc}
        & $tc_3$     & $tc_5$     & $tc_6$ \\
\hline
      $tc_2$ & 0.814 & 0.491 & 0.814\\
      $tc_3$ &       & 0.491 & 1.000\\
      $tc_5$ &       &       & \cellcolor{tuggreen} 0.305\\
    \end{tabular}
\end{minipage}
};


  \node[tool] (groupalg) [below of=similarity, node distance=100mm]
  {\begin{minipage}{130mm}
      \begin{center}\textbf{Test Case Grouping}\end{center}
      \small A test case $tc$ is added to a group if the similarity
      between any test case of the group and $tc$ is above the threshold.
    \end{minipage}};

  \node       (threshold) [left of=groupalg, node distance=120mm, anchor=center] 
  {\begin{minipage}{50mm}\centering Similarity Threshold\\\small(e.g. 0.8)\end{minipage}};

  \node[ellipse,draw=black] (result) [right of=groupalg, node distance=200mm] 
  {
    \begin{minipage}{100mm}
      \begin{center}\textbf{Result}\end{center}
      Group 1: $tc_2$ $tc_3$ $tc_6$\\
      Group 2: $tc_5$
    \end{minipage}
  };
  
  \pgfsetendarrow{\pgfarrowsingle}
  \draw[ultra thick] (s2.south)++(0,-5mm) -- (tcgen.north);
  \draw[ultra thick] (testcases.east)++(80mm,0) -- (tcexec.west);
  \draw[ultra thick] (tcexec.north) -- ++(0,40mm) node [anchor=west, midway] 
  {\begin{minipage}{80mm}\flushleft executed paths and verdicts\end{minipage}};
  \draw[ultra thick] (coverage.east) -- ++(55mm,0);
  \draw[ultra thick] (m4.south) -- (similarity.north);
  \draw[ultra thick] (threshold.east) -- (groupalg.west);
  \draw[ultra thick] (similarity.south) -- (groupalg.north);
  \draw[ultra thick] (groupalg.east) -- (result.west);

  \pgfsetstartarrow{\pgfarrowsingle}
  \draw[ultra thick] (iut) -- (tcexec);
  \pgfsetendarrow{}
  \draw[ultra thick] (testcases.north)++(0,60mm) -- (tcgen.south);
  \draw[ultra thick] (coverage.south) -- ++(0,-25mm);
  \draw[ultra thick] (coverage.west) -- ++(-80mm,0);
  \pgfsetstartarrow{}


\end{tikzpicture}
\end{center}
\end{block}

  \begin{minipage}{0.395\textwidth}
    \begin{block}{Experimental Setup}
      \begin{itemize}
      \item System under test: Registrar entity of the Session
        Initiation Protocol~\cite{rfc3261sip}. 
      \item Formal model: {\sc lotos} (process algebra)
        specification. 
      \item Test Suites:
        \begin{itemize}
          \item Test Suite A came from \cite{aichernig2007protocol}
          \item Test Suite B: 500 randomly generated test cases
          \end{itemize}
      \item Coverage Criteria:
        \begin{itemize}
          \item process coverage, action coverage, decision coverage
              and condition coverage
         \end{itemize}
      \item Distribution of the failed test cases over the detected failures:
      \begin{center}
        \epsfig{file=images/faultdist,width=0.73\columnwidth}
      \end{center}
      \end{itemize}
    \end{block}
\end{minipage}
\quad
\begin{minipage}{0.585\textwidth}
\newcommand{\mysp}{{\color{block body.bg}0}}
    \begin{block}{Some Results {\color{block title.bg}p}}
\begin{center}
  \begin{tabular}{ccp{10mm}ccccp{12mm}cccc}
  \toprule
   \multicolumn{1}{c}{\multirow{3}{25mm}{\centering Test Suite}} & 
   \multicolumn{1}{c}{\multirow{3}{45mm}{\centering Coverage Item}} && 
   \multicolumn{4}{c}{Commercial SIP Registrar} &&
   \multicolumn{4}{c}{Open Source SIP Registrar} \\
   \cmidrule{4-7} \cmidrule{9-12}
   &&& 
      \multirow{2}{42mm}{\centering Failed TCs} & 
      \multirow{2}{33mm}{\centering Fail\-ures} &
      \multirow{2}{40mm}{\centering No. Groups} & 
      \multirow{2}{45mm}{\centering Identified Failures} 
   && 
      \multirow{2}{42mm}{\centering Failed TCs} & 
      \multirow{2}{33mm}{\centering Fail\-ures} &
      \multirow{2}{40mm}{\centering No. Groups} & 
      \multirow{2}{45mm}{\centering Identified Failures} 
   \\\\
  \midrule
  \multicolumn{1}{c}{\multirow{4}{10mm}
  {\begin{center}A\end{center}}} 
                  &   Process   && 4944 & 9 & \mysp5 & 1 && 2057 & 4 & \mysp5 & 1 \\
                  &   Action    && 4944 & 9 &     18 & 5 && 2057 & 4 &     17 & 4 \\
                  &   Decision  && 4944 & 9 &     26 & 6 && 2057 & 4 &     25 & 4 \\
                  &   Condition && 4944 & 9 &     43 & 9 && 2057 & 4 &     39 & 4 \\
  \midrule
  \multirow{4}{10mm}
  {\begin{center}B\end{center}}  
                  &   Process    && 293 & 3 & \mysp1 & 1 && 263 & 5 & \mysp5 & 2 \\ 
                  &   Action     && 293 & 3 &     21 & 3 && 263 & 5 &     27 & 5 \\ 
                  &   Decision   && 293 & 3 &     24 & 3 && 263 & 5 &     27 & 5 \\ 
                  &   Condition  && 293 & 3 &     35 & 3 && 263 & 5 &     47 & 5 \\ 
  \bottomrule
  \end{tabular}
  \vspace*{8mm}
  \end{center}
  \begin{itemize}
    \item Column \emph{Test Suite}: Name of the Test Suite.
    \item Column \emph{Coverage Item}: Used coverage criterion for calculating
      the hit matrix.
    \item Column \emph{Failed TCs}: Number of failed test cases.
    \item Column \emph{Failures}: The number of failures that were
      revealed by the failed test cases.
    \item Column \emph{No. Groups}: The number of groups generated by
      our approach.
    \item Column \emph{Identified Failures}: The number of failures
      that were identified when analyzing only one test case per
      group.
  \end{itemize}
  \end{block}
  \end{minipage}

\begin{minipage}{0.64\textwidth}
  \begin{block}{Some References}
    \bibliographystyle{IEEEtran}
    \bibliography{ase2009}  
  \end{block}
\end{minipage}
\quad
  \begin{minipage}{0.34\textwidth}
   \begin{alertblock}{\textbf{At a glance}}
    \begin{itemize}
    \item Problem: in model-based testing many test cases often reveal
      the same failure.
      \item Idea: group test cases based on their similarity.
      \item Results: for our case study we observed an average
        reduction of 92\% in the number of test cases to be analyzed.
    \end{itemize}
  \end{alertblock}
  \end{minipage}


\end{frame}                                            
\end{document}


%%% Local Variables: 
%%% mode: latex
%%% TeX-master: t
%%% TeX-PDF-mode: t
%%% End: 
