% enginnering.tex

\documentclass[UTF8,10pt]{report}

%% 导言区,一般用于加载宏包
% 目录
\usepackage[titles]{tocloft}
% 引入中文
\usepackage{ctex} 
% 超链接
\usepackage[colorlinks,linkcolor=blue]{hyperref}
\usepackage{url}
% 引用
\usepackage{cite}
% \usepackage{natbib}
% 图片
\usepackage{graphicx}
% 数学
% 矩阵
\usepackage{mathtools}
\usepackage{amssymb}
\usepackage{amsmath}
\usepackage{arydshln}
% 嵌入代码
\usepackage{listings}
\usepackage{listings-ext}

% 给方程,图片,表格编号,section可能换成chapter
\renewcommand{\theequation}{\arabic{section}-\arabic{equation}}
\renewcommand{\thefigure}{\arabic{section}-\arabic{figure}}
\renewcommand{\thetable}{\arabic{section}-\arabic{table}}

\setlength{\baselineskip}{1.5em} % 行间距
\setlength{\parskip}{1.0ex} % 段间距
\setlength{\parindent}{0pt} % 段落缩进

% 文档开始
\begin{document}
%-----使用封面代替
%-----封面
\begin{center}
    \quad \\
    \vspace{5cm}
    \hspace{0cm}\Large{Lua} \\
    \small{分析Lua技术} \\
    \hspace{0cm}\Large{李美杰} \\
    \hspace{0cm}\small{2020.11.22}
    \clearpage
\end{center}
%-----

%-----留白
\thispagestyle{empty}
\clearpage
%-----

%-----目录
\tableofcontents
\clearpage
%-----

% 代码区域样式
\lstset{
    language=C,
    numbers=left,
    frame=box
}

%-----正文
% enginnering.tex

\documentclass[UTF8,10pt]{report}

%% 导言区,一般用于加载宏包
% 目录
\usepackage[titles]{tocloft}
% 引入中文
\usepackage{ctex} 
% 超链接
\usepackage[colorlinks,linkcolor=blue]{hyperref}
\usepackage{url}
% 引用
\usepackage{cite}
% \usepackage{natbib}
% 图片
\usepackage{graphicx}
% 数学
% 矩阵
\usepackage{mathtools}
\usepackage{amssymb}
\usepackage{amsmath}
\usepackage{arydshln}
% 嵌入代码
\usepackage{listings}
\usepackage{listings-ext}

% 给方程,图片,表格编号,section可能换成chapter
\renewcommand{\theequation}{\arabic{section}-\arabic{equation}}
\renewcommand{\thefigure}{\arabic{section}-\arabic{figure}}
\renewcommand{\thetable}{\arabic{section}-\arabic{table}}

\setlength{\baselineskip}{1.5em} % 行间距
\setlength{\parskip}{1.0ex} % 段间距
\setlength{\parindent}{0pt} % 段落缩进

% 文档开始
\begin{document}
%-----使用封面代替
%-----封面
\begin{center}
    \quad \\
    \vspace{5cm}
    \hspace{0cm}\Large{Lua} \\
    \small{分析Lua技术} \\
    \hspace{0cm}\Large{李美杰} \\
    \hspace{0cm}\small{2020.11.22}
    \clearpage
\end{center}
%-----

%-----留白
\thispagestyle{empty}
\clearpage
%-----

%-----目录
\tableofcontents
\clearpage
%-----

% 代码区域样式
\lstset{
    language=C,
    numbers=left,
    frame=box
}

%-----正文
% enginnering.tex

\documentclass[UTF8,10pt]{report}

%% 导言区,一般用于加载宏包
% 目录
\usepackage[titles]{tocloft}
% 引入中文
\usepackage{ctex} 
% 超链接
\usepackage[colorlinks,linkcolor=blue]{hyperref}
\usepackage{url}
% 引用
\usepackage{cite}
% \usepackage{natbib}
% 图片
\usepackage{graphicx}
% 数学
% 矩阵
\usepackage{mathtools}
\usepackage{amssymb}
\usepackage{amsmath}
\usepackage{arydshln}
% 嵌入代码
\usepackage{listings}
\usepackage{listings-ext}

% 给方程,图片,表格编号,section可能换成chapter
\renewcommand{\theequation}{\arabic{section}-\arabic{equation}}
\renewcommand{\thefigure}{\arabic{section}-\arabic{figure}}
\renewcommand{\thetable}{\arabic{section}-\arabic{table}}

\setlength{\baselineskip}{1.5em} % 行间距
\setlength{\parskip}{1.0ex} % 段间距
\setlength{\parindent}{0pt} % 段落缩进

% 文档开始
\begin{document}
%-----使用封面代替
%-----封面
\begin{center}
    \quad \\
    \vspace{5cm}
    \hspace{0cm}\Large{Lua} \\
    \small{分析Lua技术} \\
    \hspace{0cm}\Large{李美杰} \\
    \hspace{0cm}\small{2020.11.22}
    \clearpage
\end{center}
%-----

%-----留白
\thispagestyle{empty}
\clearpage
%-----

%-----目录
\tableofcontents
\clearpage
%-----

% 代码区域样式
\lstset{
    language=C,
    numbers=left,
    frame=box
}

%-----正文
% enginnering.tex

\documentclass[UTF8,10pt]{report}

%% 导言区,一般用于加载宏包
% 目录
\usepackage[titles]{tocloft}
% 引入中文
\usepackage{ctex} 
% 超链接
\usepackage[colorlinks,linkcolor=blue]{hyperref}
\usepackage{url}
% 引用
\usepackage{cite}
% \usepackage{natbib}
% 图片
\usepackage{graphicx}
% 数学
% 矩阵
\usepackage{mathtools}
\usepackage{amssymb}
\usepackage{amsmath}
\usepackage{arydshln}
% 嵌入代码
\usepackage{listings}
\usepackage{listings-ext}

% 给方程,图片,表格编号,section可能换成chapter
\renewcommand{\theequation}{\arabic{section}-\arabic{equation}}
\renewcommand{\thefigure}{\arabic{section}-\arabic{figure}}
\renewcommand{\thetable}{\arabic{section}-\arabic{table}}

\setlength{\baselineskip}{1.5em} % 行间距
\setlength{\parskip}{1.0ex} % 段间距
\setlength{\parindent}{0pt} % 段落缩进

% 文档开始
\begin{document}
%-----使用封面代替
%-----封面
\begin{center}
    \quad \\
    \vspace{5cm}
    \hspace{0cm}\Large{Lua} \\
    \small{分析Lua技术} \\
    \hspace{0cm}\Large{李美杰} \\
    \hspace{0cm}\small{2020.11.22}
    \clearpage
\end{center}
%-----

%-----留白
\thispagestyle{empty}
\clearpage
%-----

%-----目录
\tableofcontents
\clearpage
%-----

% 代码区域样式
\lstset{
    language=C,
    numbers=left,
    frame=box
}

%-----正文
\input{subpart/lua.tex}
\input{subpart/vm.tex}
\input{subpart/chunk.tex}
\input{subpart/opcode.tex}
\input{subpart/state.tex}
\input{subpart/table.tex}

%-----参考文档
\bibliographystyle{plain}
\bibliography{lua.bib}
%-----

\end{document}
\chapter{Virtual Machine}

从lua1.0就内置了虚拟机,lua脚本并不是直接被lua解释器解释执行,而是先由lua编译器编译为字节码,然后交给lua虚拟机执行。

Lua5.0\cite{tiollua5}实现参考文章

高级编程语言虚拟机是对真实计算机的模拟和抽象,如同真实机器有一套自己的指令集一样,虚拟机也有自己的指令集,按照实现方式,可分为
\begin{itemize}
    \item {Stack-based基于栈的虚拟机需要使用PUSH指令往栈顶压入值、弹出值,其他指令则是对栈顶值进行操作,特点是指令个数多指令长度较短}
    \item {Register-based基于寄存器的虚拟机由于直接对寄存器进行寻址,特点是指令个数少指令长度长}
\end{itemize}

与WebAssemlby等虚拟机一样,Lua的vm指令也分为
\begin{itemize}
    \item {操作码opcode}
    \item {静态参数static arguments}
\end{itemize}

指令在执行时,通常还需要动态参数,一般称作操作数operands。

从汇编的本质是CPU二进制指令的文本形式角度来看,指令集对应的是字节码的二进制形式。

在lua术语中,一段可以被lua解释器解释执行的字节码被称作chunk。

\section{编译期}

广义编译期会把编译过程分为不同的阶段,每个解读通过管道pipeline来串联形成完整的编译期。
主要编译阶段有
\begin{itemize}
    \item {前端Front End}
    \begin{itemize}
        \item {预处理}
        \item {词法分析}
        \item {语法分析}
        \item {语义分析}
        \item {中间代码生成}
    \end{itemize}
    \item {中端Middle End}
    \begin{itemize}
        \item {中间代码优化}
    \end{itemize}
    \item {后端Back End}
    \begin{itemize}
        \item {目标代码生成}
    \end{itemize}
\end{itemize}

Lua语言不支持宏特性、不需要预处理,又是动态语言,语义分析也不需要。
lua编译过程中是一次遍历就生成指令的,并没有对源码和语法结构多次遍历,是通过指令回填技术来实现支持顺序的。

Lua解析与代码生成的内部实现\cite{parsingAndCodeGeneration}
\chapter{Chunk}

Lua实现是把脚本编译成二进制的内部格式,然后给lua vm来执行,可以通过

luac -l -l src.lua

来编译脚本文件为二进制格式并保存到磁盘上。
在lua的术语中,编译单位叫做chunk块,相应的编译后的二进制格式叫做binary chunk二进制块。

和任何二进制格式(如WebAssembly模块的二进制格式等)一样,本质上就是一个字节流。在这个字节流内部,
连续N个字节可以构成基础数据类型,如整数、浮点数等,基本数据类型又构成复合型数据类型,复合型数据类型
又可以构成更复杂的数据类型。这个字节流就包含了这些数据结构,也就是二进制块的内容解释后的含义。

字节流是由字节组成,字节的顺序就是很重要了,lua二进制块不像其他具有跨平台性质,如WebAssembly的字节采用
Little-endian,lua直角使用机器的字节顺序。除byte和instruction类型,基本类型都是定长类型,但是定长数据类型的
长度由可以因机器相关或编译时可配置的。


\section{varint}
Lua5.4引入了变长整数类型,目的是让二进制紧凑性更强了。这是因为lua二进制块中存储了很多整数,如调试用的行号、字符串等的长度,
这些整数通常很小,但都占用了固定的长度,WebAssembly模块的二进制采用了LEB128来编码变长整数以节约空间,lua5.4采用类似的编码
格式,存在的差异是
\begin{itemize}
    \item {LEB128是小端字节序,Lua使用大端字节序}
    \item {都利用字节的MSB(Most Significant Bit)来标识是否有后续字节,LES128中MSB为1表示有后续,Lua则相反}
\end{itemize}

这里举例来说明一些过程,有一个整数N,可以用3字节来表示,那编码的流程有三个步骤
\begin{itemize}
    \item {第一步就是把这个3字节(24bit)分成四组,,每组7个bit,剩余bit补0占位}
    \item {第二步是小端字节序,把字节位置颠倒}
    \item {第三步是把0占位的MSB变成1}
\end{itemize}
\begin{center}
    \begin{tabular}{|c|c|c|c|c|} \hline
       \hbox{data} & \hbox{} & \hbox{xxxxxxxx} & \hbox{yyyyyyyy} & \hbox{zzzzzzzz} \\ \hline
       \hbox{step1} & \hbox{00000xxx} & \hbox{0xxxxxyy} & \hbox{0yyyyyyz}  & \hbox{0zzzzzzz} \\ \hline
       \hbox{step2} & \hbox{0zzzzzzz} & \hbox{0yyyyyyz} & \hbox{0xxxxxyy}  & \hbox{00000xxx} \\ \hline
       \hbox{step3} & \hbox{1zzzzzzz} & \hbox{1yyyyyyz} & \hbox{1xxxxxyy}  & \hbox{00000xxx} \\ \hline
    \end{tabular}
\end{center}
lua5.4因为大端的关系,只需要两步
\begin{center}
    \begin{tabular}{|c|c|c|c|c|} \hline
       \hbox{data} & \hbox{} & \hbox{xxxxxxxx} & \hbox{yyyyyyyy} & \hbox{zzzzzzzz} \\ \hline
       \hbox{step1} & \hbox{00000xxx} & \hbox{0xxxxxyy} & \hbox{0yyyyyyz}  & \hbox{0zzzzzzz} \\ \hline
       \hbox{step3} & \hbox{00000xxx} & \hbox{0xxxxxyy} & \hbox{0yyyyyyz}  & \hbox{1zzzzzzz} \\ \hline
    \end{tabular}
\end{center}
这里是对300整数进行编码的结果
\begin{center}
    \begin{tabular}{|c|c|c|} \hline
       \hbox{binary} & \hbox{00000001} & \hbox{00101100} \\ \hline
       \hbox{LES128} & \hbox{10101100} & \hbox{00000001} \\ \hline
       \hbox{Lua5.4} & \hbox{00000010} & \hbox{10101100} \\ \hline
    \end{tabular}
\end{center}

这里要说明一下的是WebAssembly核心规范对WebAssembly模块的二进制格式进行了定义,但Lua的二进制格式完全属于实现细节,没有相应的规范,
也不保证向后兼容性,其权威性就是Lua官方实现的C语言的源代码。

\chapter{Opcode}

与WebAssemlby等虚拟机一样,Lua的vm指令也分为两个部分,操作码opcode和静态参数static arguments。
指令在执行时,通常还需要动态参数,一般称作操作数operands。

WebAssemlby采用了栈式Stack-based虚拟机模型,指令在执行时可以从操作数栈上压入、弹出操作数。而Lua
采用了寄存器式Register-based虚拟机模型,指令在执行时可以操作一个虚拟寄存器,用于读写操作数。指令参数
最主要的作用就是表示寄存器索引,这也是lua虚拟机采用定长指令的原因之一(WebAssemlby采用变长指令)。

理解lua,先要理解它编译后的形态,是指令,就至少要到汇编级别来理解了,毕竟汇编的本质是CPU二进制指令的文本形式。

lua5.3的指令编码如下表所示
\begin{center}
    \begin{tabular}{|c|c|c|c|c|} \hline
       \hbox{} & \hbox{31-24} & \hbox{23-16} & \hbox{15-8} & \hbox{0-7} \\ \hline
       \hbox{iABC} & \hbox{B:9} & \hbox{C:9} & \hbox{A:8}  & \hbox{Opcode:6} \\ \hline
       \hbox{iABx} & \hbox{Bx:18-1} & \hbox{Bx:18-2} & \hbox{A:8}  & \hbox{Opcode:6} \\ \hline
       \hbox{iAsBx} & \hbox{sBx:18-1} & \hbox{sBx:18-2} & \hbox{A:8}  & \hbox{Opcode:6} \\ \hline
       \hbox{iAx} & \hbox{Ax:26-1} & \hbox{Ax:26-2} & \hbox{Ax:26-3}  & \hbox{Opcode:6} \\ \hline
    \end{tabular}
\end{center}
在lua5.4中opcode由6位变成了7位。

编码模式中的字母含义i表示指令instruction,x表示扩展extended,s表示符号signed,字母ABC表示三个参数,A一般用作
目标寄存器索引,B和C既可以是寄存器索引,也可以是常量池索引

\begin{itemize}
    \item {iABC
    \begin{center}
        \begin{tabular}{|c|c|c|c|c|} \hline
           \hbox{} & \hbox{31-24} & \hbox{23-16} & \hbox{15-8} & \hbox{0-7} \\ \hline
           \hbox{5.3} & \hbox{B:9} & \hbox{C:9} & \hbox{A:8}  & \hbox{Opcode:6} \\ \hline
           \hbox{5.4} & \hbox{C:8C:8} & \hbox{B:8-k} & \hbox{A:8}  & \hbox{Opcode:7} \\ \hline
        \end{tabular}
    \end{center}
    }
    \item {iABx}
    \item {iAsBx}
    \item {iAx}
    \item {isJ}
\end{itemize}

基本的四个指令形式,iABC、iABx、iAsBx、iAx。sBx表示signed int。生成lua代码对应的指令的

luac -l -l src.lua

通过上面命令编译后,可以得到编译后的指令代码。

\chapter{State}

Lua State是lua非常核心的概念,全部API函数都是围绕Lua State进行操作的,Lua State内部封装的最为基础的一个状态就是虚拟栈。

虚拟栈是宿主语言(对官方Lua来说就是C语言,其他如luajit是C++)与lua语言进行沟通的桥梁。
\LaTeX{中的表格}
\begin{table}[h]
    % 居中排版
    \centering
    \caption{成绩单可参考图}\ref{fig:ima1}
    
    \begin{tabular}{|c|c|c|c|}
    \hline
    姓名 & 语文 & 数学 & 英语  \\
    \hline
    张三 & 78 & 82 & 45 \\
    \hline
    李四 & 56 & 95 & 68 \\
    \hline
    王五 & 72 & 68 & 76 \\
    \hline

    \end{tabular}
    \label{tab:table1}
\end{table}

\begin{tabular}{|l|c|r|}
    \hline & PR & NPR \\
    \hline 特征 & 客观的 & 主观的 \\
    \hline 效果 & 物理世界的仿真 & 艺术化的风格 \\
    \hline 方法 & 模拟仿真 & 风格化 \\
    \hline 方法 & 模拟仿真 & 风格化 \\
    \hline 方法 & 模拟仿真 & 风格化 \\
    \hline 方法 & 模拟仿真 & 风格化 \\
\end{tabular}


%-----参考文档
\bibliographystyle{plain}
\bibliography{lua.bib}
%-----

\end{document}
\chapter{Virtual Machine}

从lua1.0就内置了虚拟机,lua脚本并不是直接被lua解释器解释执行,而是先由lua编译器编译为字节码,然后交给lua虚拟机执行。

Lua5.0\cite{tiollua5}实现参考文章

高级编程语言虚拟机是对真实计算机的模拟和抽象,如同真实机器有一套自己的指令集一样,虚拟机也有自己的指令集,按照实现方式,可分为
\begin{itemize}
    \item {Stack-based基于栈的虚拟机需要使用PUSH指令往栈顶压入值、弹出值,其他指令则是对栈顶值进行操作,特点是指令个数多指令长度较短}
    \item {Register-based基于寄存器的虚拟机由于直接对寄存器进行寻址,特点是指令个数少指令长度长}
\end{itemize}

与WebAssemlby等虚拟机一样,Lua的vm指令也分为
\begin{itemize}
    \item {操作码opcode}
    \item {静态参数static arguments}
\end{itemize}

指令在执行时,通常还需要动态参数,一般称作操作数operands。

从汇编的本质是CPU二进制指令的文本形式角度来看,指令集对应的是字节码的二进制形式。

在lua术语中,一段可以被lua解释器解释执行的字节码被称作chunk。

\section{编译期}

广义编译期会把编译过程分为不同的阶段,每个解读通过管道pipeline来串联形成完整的编译期。
主要编译阶段有
\begin{itemize}
    \item {前端Front End}
    \begin{itemize}
        \item {预处理}
        \item {词法分析}
        \item {语法分析}
        \item {语义分析}
        \item {中间代码生成}
    \end{itemize}
    \item {中端Middle End}
    \begin{itemize}
        \item {中间代码优化}
    \end{itemize}
    \item {后端Back End}
    \begin{itemize}
        \item {目标代码生成}
    \end{itemize}
\end{itemize}

Lua语言不支持宏特性、不需要预处理,又是动态语言,语义分析也不需要。
lua编译过程中是一次遍历就生成指令的,并没有对源码和语法结构多次遍历,是通过指令回填技术来实现支持顺序的。

Lua解析与代码生成的内部实现\cite{parsingAndCodeGeneration}
\chapter{Chunk}

Lua实现是把脚本编译成二进制的内部格式,然后给lua vm来执行,可以通过

luac -l -l src.lua

来编译脚本文件为二进制格式并保存到磁盘上。
在lua的术语中,编译单位叫做chunk块,相应的编译后的二进制格式叫做binary chunk二进制块。

和任何二进制格式(如WebAssembly模块的二进制格式等)一样,本质上就是一个字节流。在这个字节流内部,
连续N个字节可以构成基础数据类型,如整数、浮点数等,基本数据类型又构成复合型数据类型,复合型数据类型
又可以构成更复杂的数据类型。这个字节流就包含了这些数据结构,也就是二进制块的内容解释后的含义。

字节流是由字节组成,字节的顺序就是很重要了,lua二进制块不像其他具有跨平台性质,如WebAssembly的字节采用
Little-endian,lua直角使用机器的字节顺序。除byte和instruction类型,基本类型都是定长类型,但是定长数据类型的
长度由可以因机器相关或编译时可配置的。


\section{varint}
Lua5.4引入了变长整数类型,目的是让二进制紧凑性更强了。这是因为lua二进制块中存储了很多整数,如调试用的行号、字符串等的长度,
这些整数通常很小,但都占用了固定的长度,WebAssembly模块的二进制采用了LEB128来编码变长整数以节约空间,lua5.4采用类似的编码
格式,存在的差异是
\begin{itemize}
    \item {LEB128是小端字节序,Lua使用大端字节序}
    \item {都利用字节的MSB(Most Significant Bit)来标识是否有后续字节,LES128中MSB为1表示有后续,Lua则相反}
\end{itemize}

这里举例来说明一些过程,有一个整数N,可以用3字节来表示,那编码的流程有三个步骤
\begin{itemize}
    \item {第一步就是把这个3字节(24bit)分成四组,,每组7个bit,剩余bit补0占位}
    \item {第二步是小端字节序,把字节位置颠倒}
    \item {第三步是把0占位的MSB变成1}
\end{itemize}
\begin{center}
    \begin{tabular}{|c|c|c|c|c|} \hline
       \hbox{data} & \hbox{} & \hbox{xxxxxxxx} & \hbox{yyyyyyyy} & \hbox{zzzzzzzz} \\ \hline
       \hbox{step1} & \hbox{00000xxx} & \hbox{0xxxxxyy} & \hbox{0yyyyyyz}  & \hbox{0zzzzzzz} \\ \hline
       \hbox{step2} & \hbox{0zzzzzzz} & \hbox{0yyyyyyz} & \hbox{0xxxxxyy}  & \hbox{00000xxx} \\ \hline
       \hbox{step3} & \hbox{1zzzzzzz} & \hbox{1yyyyyyz} & \hbox{1xxxxxyy}  & \hbox{00000xxx} \\ \hline
    \end{tabular}
\end{center}
lua5.4因为大端的关系,只需要两步
\begin{center}
    \begin{tabular}{|c|c|c|c|c|} \hline
       \hbox{data} & \hbox{} & \hbox{xxxxxxxx} & \hbox{yyyyyyyy} & \hbox{zzzzzzzz} \\ \hline
       \hbox{step1} & \hbox{00000xxx} & \hbox{0xxxxxyy} & \hbox{0yyyyyyz}  & \hbox{0zzzzzzz} \\ \hline
       \hbox{step3} & \hbox{00000xxx} & \hbox{0xxxxxyy} & \hbox{0yyyyyyz}  & \hbox{1zzzzzzz} \\ \hline
    \end{tabular}
\end{center}
这里是对300整数进行编码的结果
\begin{center}
    \begin{tabular}{|c|c|c|} \hline
       \hbox{binary} & \hbox{00000001} & \hbox{00101100} \\ \hline
       \hbox{LES128} & \hbox{10101100} & \hbox{00000001} \\ \hline
       \hbox{Lua5.4} & \hbox{00000010} & \hbox{10101100} \\ \hline
    \end{tabular}
\end{center}

这里要说明一下的是WebAssembly核心规范对WebAssembly模块的二进制格式进行了定义,但Lua的二进制格式完全属于实现细节,没有相应的规范,
也不保证向后兼容性,其权威性就是Lua官方实现的C语言的源代码。

\chapter{Opcode}

与WebAssemlby等虚拟机一样,Lua的vm指令也分为两个部分,操作码opcode和静态参数static arguments。
指令在执行时,通常还需要动态参数,一般称作操作数operands。

WebAssemlby采用了栈式Stack-based虚拟机模型,指令在执行时可以从操作数栈上压入、弹出操作数。而Lua
采用了寄存器式Register-based虚拟机模型,指令在执行时可以操作一个虚拟寄存器,用于读写操作数。指令参数
最主要的作用就是表示寄存器索引,这也是lua虚拟机采用定长指令的原因之一(WebAssemlby采用变长指令)。

理解lua,先要理解它编译后的形态,是指令,就至少要到汇编级别来理解了,毕竟汇编的本质是CPU二进制指令的文本形式。

lua5.3的指令编码如下表所示
\begin{center}
    \begin{tabular}{|c|c|c|c|c|} \hline
       \hbox{} & \hbox{31-24} & \hbox{23-16} & \hbox{15-8} & \hbox{0-7} \\ \hline
       \hbox{iABC} & \hbox{B:9} & \hbox{C:9} & \hbox{A:8}  & \hbox{Opcode:6} \\ \hline
       \hbox{iABx} & \hbox{Bx:18-1} & \hbox{Bx:18-2} & \hbox{A:8}  & \hbox{Opcode:6} \\ \hline
       \hbox{iAsBx} & \hbox{sBx:18-1} & \hbox{sBx:18-2} & \hbox{A:8}  & \hbox{Opcode:6} \\ \hline
       \hbox{iAx} & \hbox{Ax:26-1} & \hbox{Ax:26-2} & \hbox{Ax:26-3}  & \hbox{Opcode:6} \\ \hline
    \end{tabular}
\end{center}
在lua5.4中opcode由6位变成了7位。

编码模式中的字母含义i表示指令instruction,x表示扩展extended,s表示符号signed,字母ABC表示三个参数,A一般用作
目标寄存器索引,B和C既可以是寄存器索引,也可以是常量池索引

\begin{itemize}
    \item {iABC
    \begin{center}
        \begin{tabular}{|c|c|c|c|c|} \hline
           \hbox{} & \hbox{31-24} & \hbox{23-16} & \hbox{15-8} & \hbox{0-7} \\ \hline
           \hbox{5.3} & \hbox{B:9} & \hbox{C:9} & \hbox{A:8}  & \hbox{Opcode:6} \\ \hline
           \hbox{5.4} & \hbox{C:8C:8} & \hbox{B:8-k} & \hbox{A:8}  & \hbox{Opcode:7} \\ \hline
        \end{tabular}
    \end{center}
    }
    \item {iABx}
    \item {iAsBx}
    \item {iAx}
    \item {isJ}
\end{itemize}

基本的四个指令形式,iABC、iABx、iAsBx、iAx。sBx表示signed int。生成lua代码对应的指令的

luac -l -l src.lua

通过上面命令编译后,可以得到编译后的指令代码。

\chapter{State}

Lua State是lua非常核心的概念,全部API函数都是围绕Lua State进行操作的,Lua State内部封装的最为基础的一个状态就是虚拟栈。

虚拟栈是宿主语言(对官方Lua来说就是C语言,其他如luajit是C++)与lua语言进行沟通的桥梁。
\LaTeX{中的表格}
\begin{table}[h]
    % 居中排版
    \centering
    \caption{成绩单可参考图}\ref{fig:ima1}
    
    \begin{tabular}{|c|c|c|c|}
    \hline
    姓名 & 语文 & 数学 & 英语  \\
    \hline
    张三 & 78 & 82 & 45 \\
    \hline
    李四 & 56 & 95 & 68 \\
    \hline
    王五 & 72 & 68 & 76 \\
    \hline

    \end{tabular}
    \label{tab:table1}
\end{table}

\begin{tabular}{|l|c|r|}
    \hline & PR & NPR \\
    \hline 特征 & 客观的 & 主观的 \\
    \hline 效果 & 物理世界的仿真 & 艺术化的风格 \\
    \hline 方法 & 模拟仿真 & 风格化 \\
    \hline 方法 & 模拟仿真 & 风格化 \\
    \hline 方法 & 模拟仿真 & 风格化 \\
    \hline 方法 & 模拟仿真 & 风格化 \\
\end{tabular}


%-----参考文档
\bibliographystyle{plain}
\bibliography{lua.bib}
%-----

\end{document}
\chapter{Chunk}

Lua实现是把脚本编译成二进制的内部格式,然后给lua vm来执行,可以通过

luac -l -l src.lua

来编译脚本文件为二进制格式并保存到磁盘上。
在lua的术语中,编译单位叫做chunk块,相应的编译后的二进制格式叫做binary chunk二进制块。

和任何二进制格式(如WebAssembly模块的二进制格式等)一样,本质上就是一个字节流。在这个字节流内部,
连续N个字节可以构成基础数据类型,如整数、浮点数等,基本数据类型又构成复合型数据类型,复合型数据类型
又可以构成更复杂的数据类型。这个字节流就包含了这些数据结构,也就是二进制块的内容解释后的含义。

字节流是由字节组成,字节的顺序就是很重要了,lua二进制块不像其他具有跨平台性质,如WebAssembly的字节采用
Little-endian,lua直角使用机器的字节顺序。除byte和instruction类型,基本类型都是定长类型,但是定长数据类型的
长度由可以因机器相关或编译时可配置的。


\section{varint}
Lua5.4引入了变长整数类型,目的是让二进制紧凑性更强了。这是因为lua二进制块中存储了很多整数,如调试用的行号、字符串等的长度,
这些整数通常很小,但都占用了固定的长度,WebAssembly模块的二进制采用了LEB128来编码变长整数以节约空间,lua5.4采用类似的编码
格式,存在的差异是
\begin{itemize}
    \item {LEB128是小端字节序,Lua使用大端字节序}
    \item {都利用字节的MSB(Most Significant Bit)来标识是否有后续字节,LES128中MSB为1表示有后续,Lua则相反}
\end{itemize}

这里举例来说明一些过程,有一个整数N,可以用3字节来表示,那编码的流程有三个步骤
\begin{itemize}
    \item {第一步就是把这个3字节(24bit)分成四组,,每组7个bit,剩余bit补0占位}
    \item {第二步是小端字节序,把字节位置颠倒}
    \item {第三步是把0占位的MSB变成1}
\end{itemize}
\begin{center}
    \begin{tabular}{|c|c|c|c|c|} \hline
       \hbox{data} & \hbox{} & \hbox{xxxxxxxx} & \hbox{yyyyyyyy} & \hbox{zzzzzzzz} \\ \hline
       \hbox{step1} & \hbox{00000xxx} & \hbox{0xxxxxyy} & \hbox{0yyyyyyz}  & \hbox{0zzzzzzz} \\ \hline
       \hbox{step2} & \hbox{0zzzzzzz} & \hbox{0yyyyyyz} & \hbox{0xxxxxyy}  & \hbox{00000xxx} \\ \hline
       \hbox{step3} & \hbox{1zzzzzzz} & \hbox{1yyyyyyz} & \hbox{1xxxxxyy}  & \hbox{00000xxx} \\ \hline
    \end{tabular}
\end{center}
lua5.4因为大端的关系,只需要两步
\begin{center}
    \begin{tabular}{|c|c|c|c|c|} \hline
       \hbox{data} & \hbox{} & \hbox{xxxxxxxx} & \hbox{yyyyyyyy} & \hbox{zzzzzzzz} \\ \hline
       \hbox{step1} & \hbox{00000xxx} & \hbox{0xxxxxyy} & \hbox{0yyyyyyz}  & \hbox{0zzzzzzz} \\ \hline
       \hbox{step3} & \hbox{00000xxx} & \hbox{0xxxxxyy} & \hbox{0yyyyyyz}  & \hbox{1zzzzzzz} \\ \hline
    \end{tabular}
\end{center}
这里是对300整数进行编码的结果
\begin{center}
    \begin{tabular}{|c|c|c|} \hline
       \hbox{binary} & \hbox{00000001} & \hbox{00101100} \\ \hline
       \hbox{LES128} & \hbox{10101100} & \hbox{00000001} \\ \hline
       \hbox{Lua5.4} & \hbox{00000010} & \hbox{10101100} \\ \hline
    \end{tabular}
\end{center}

这里要说明一下的是WebAssembly核心规范对WebAssembly模块的二进制格式进行了定义,但Lua的二进制格式完全属于实现细节,没有相应的规范,
也不保证向后兼容性,其权威性就是Lua官方实现的C语言的源代码。

\chapter{Opcode}

与WebAssemlby等虚拟机一样,Lua的vm指令也分为两个部分,操作码opcode和静态参数static arguments。
指令在执行时,通常还需要动态参数,一般称作操作数operands。

WebAssemlby采用了栈式Stack-based虚拟机模型,指令在执行时可以从操作数栈上压入、弹出操作数。而Lua
采用了寄存器式Register-based虚拟机模型,指令在执行时可以操作一个虚拟寄存器,用于读写操作数。指令参数
最主要的作用就是表示寄存器索引,这也是lua虚拟机采用定长指令的原因之一(WebAssemlby采用变长指令)。

理解lua,先要理解它编译后的形态,是指令,就至少要到汇编级别来理解了,毕竟汇编的本质是CPU二进制指令的文本形式。

lua5.3的指令编码如下表所示
\begin{center}
    \begin{tabular}{|c|c|c|c|c|} \hline
       \hbox{} & \hbox{31-24} & \hbox{23-16} & \hbox{15-8} & \hbox{0-7} \\ \hline
       \hbox{iABC} & \hbox{B:9} & \hbox{C:9} & \hbox{A:8}  & \hbox{Opcode:6} \\ \hline
       \hbox{iABx} & \hbox{Bx:18-1} & \hbox{Bx:18-2} & \hbox{A:8}  & \hbox{Opcode:6} \\ \hline
       \hbox{iAsBx} & \hbox{sBx:18-1} & \hbox{sBx:18-2} & \hbox{A:8}  & \hbox{Opcode:6} \\ \hline
       \hbox{iAx} & \hbox{Ax:26-1} & \hbox{Ax:26-2} & \hbox{Ax:26-3}  & \hbox{Opcode:6} \\ \hline
    \end{tabular}
\end{center}
在lua5.4中opcode由6位变成了7位。

编码模式中的字母含义i表示指令instruction,x表示扩展extended,s表示符号signed,字母ABC表示三个参数,A一般用作
目标寄存器索引,B和C既可以是寄存器索引,也可以是常量池索引

\begin{itemize}
    \item {iABC
    \begin{center}
        \begin{tabular}{|c|c|c|c|c|} \hline
           \hbox{} & \hbox{31-24} & \hbox{23-16} & \hbox{15-8} & \hbox{0-7} \\ \hline
           \hbox{5.3} & \hbox{B:9} & \hbox{C:9} & \hbox{A:8}  & \hbox{Opcode:6} \\ \hline
           \hbox{5.4} & \hbox{C:8C:8} & \hbox{B:8-k} & \hbox{A:8}  & \hbox{Opcode:7} \\ \hline
        \end{tabular}
    \end{center}
    }
    \item {iABx}
    \item {iAsBx}
    \item {iAx}
    \item {isJ}
\end{itemize}

基本的四个指令形式,iABC、iABx、iAsBx、iAx。sBx表示signed int。生成lua代码对应的指令的

luac -l -l src.lua

通过上面命令编译后,可以得到编译后的指令代码。


%-----参考文档
\bibliographystyle{plain}
\bibliography{lua.bib}
%-----

\end{document}