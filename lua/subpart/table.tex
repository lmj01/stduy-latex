\chapter{Table}

Lua提供唯一的一种数据结构,就是lua table。它本质上是关联数组,里面存放的是两两关联的键值对。
Lua表的构造器语法与JSON语法非常类型,它更复杂,功能也更强大。

\section{注册表}

Lua给用户提供了一个注册表,Lua本身也使用它,比如全局变量就是借助这个注册表实现的。

\section{元编程}
元程序是指能够处理程序的程序,这里的处理包括读取、生成、分析、转换等
元编程是指编写元程序的编程技术,如C语言中的宏和C++的模板都是在编译期生成代码的技术。

\paragraph{元表}
在Lua中每个值都可以有元表,如果值的类型是表或用户类型,则可以拥有自己的元表,其他类型的值则是每种
类型共享一个元表。

\paragraph{元方法}

\section{迭代器}
迭代器是一种经典的设计模式,使用迭代器对集合或容器的元素进行遍历