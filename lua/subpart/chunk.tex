\chapter{Chunk}

Lua实现是把脚本编译成二进制的内部格式,然后给lua vm来执行,在lua的术语中,编译单位叫做chunk块,相应的编译后的二进制格式叫做binary chunk二进制块。

和任何二进制格式(如WebAssembly模块的二进制格式等)一样,本质上就是一个字节流。在这个字节流内部,
连续N个字节可以构成基础数据类型,如整数、浮点数等,基本数据类型又构成复合型数据类型,复合型数据类型
又可以构成更复杂的数据类型。这个字节流就包含了这些数据结构,也就是二进制块的内容解释后的含义。

字节流是由字节组成,字节的顺序就是很重要了,lua二进制块不像其他具有跨平台性质,如WebAssembly的字节采用
Little-endian,lua直角使用机器的字节顺序。除byte和instruction类型,基本类型都是定长类型,但是定长数据类型的
长度由可以因机器相关或编译时可配置的。

lua\_load会对chunk编译生成一个mainfunc proto,内部所有函数都对应一个proto。整个过程就是以
mainfunc为根节点的proto树结构。


\section{varint}
Lua5.4引入了变长整数类型,目的是让二进制紧凑性更强了。这是因为lua二进制块中存储了很多整数,如调试用的行号、字符串等的长度,
这些整数通常很小,但都占用了固定的长度,WebAssembly模块的二进制采用了LEB128来编码变长整数以节约空间,lua5.4采用类似的编码
格式,存在的差异是
\begin{itemize}
    \item {LEB128是小端字节序,Lua使用大端字节序}
    \item {都利用字节的MSB(Most Significant Bit)来标识是否有后续字节,LES128中MSB为1表示有后续,Lua则相反}
\end{itemize}

这里举例来说明一些过程,有一个整数N,可以用3字节来表示,那编码的流程有三个步骤
\begin{itemize}
    \item {第一步就是把这个3字节(24bit)分成四组,,每组7个bit,剩余bit补0占位}
    \item {第二步是小端字节序,把字节位置颠倒}
    \item {第三步是把0占位的MSB变成1}
\end{itemize}
\begin{center}
    \begin{tabular}{|c|c|c|c|c|} \hline
       \hbox{data} & \hbox{} & \hbox{xxxxxxxx} & \hbox{yyyyyyyy} & \hbox{zzzzzzzz} \\ \hline
       \hbox{step1} & \hbox{00000xxx} & \hbox{0xxxxxyy} & \hbox{0yyyyyyz}  & \hbox{0zzzzzzz} \\ \hline
       \hbox{step2} & \hbox{0zzzzzzz} & \hbox{0yyyyyyz} & \hbox{0xxxxxyy}  & \hbox{00000xxx} \\ \hline
       \hbox{step3} & \hbox{1zzzzzzz} & \hbox{1yyyyyyz} & \hbox{1xxxxxyy}  & \hbox{00000xxx} \\ \hline
    \end{tabular}
\end{center}
lua5.4因为大端的关系,只需要两步
\begin{center}
    \begin{tabular}{|c|c|c|c|c|} \hline
       \hbox{data} & \hbox{} & \hbox{xxxxxxxx} & \hbox{yyyyyyyy} & \hbox{zzzzzzzz} \\ \hline
       \hbox{step1} & \hbox{00000xxx} & \hbox{0xxxxxyy} & \hbox{0yyyyyyz}  & \hbox{0zzzzzzz} \\ \hline
       \hbox{step3} & \hbox{00000xxx} & \hbox{0xxxxxyy} & \hbox{0yyyyyyz}  & \hbox{1zzzzzzz} \\ \hline
    \end{tabular}
\end{center}
这里是对300整数进行编码的结果
\begin{center}
    \begin{tabular}{|c|c|c|} \hline
       \hbox{binary} & \hbox{00000001} & \hbox{00101100} \\ \hline
       \hbox{LES128} & \hbox{10101100} & \hbox{00000001} \\ \hline
       \hbox{Lua5.4} & \hbox{00000010} & \hbox{10101100} \\ \hline
    \end{tabular}
\end{center}

这里要说明一下的是WebAssembly核心规范对WebAssembly模块的二进制格式进行了定义,但Lua的二进制格式完全属于实现细节,没有相应的规范,
也不保证向后兼容性,其权威性就是Lua官方实现的C语言的源代码。


\section{luac}
luac主要有两个用途,一是作为编译期,把lua源文件编译成二进制chunk文件;一是作为反编译器,分析二进制chunk,将信息输出到控制台。

luac -l -l src.lua

通过上面命令编译后,可以得到编译后的指令代码。

或者直接
>./luac -l -l -
> local t = {}; t[0] = 1; local a = t[0]
> Ctrl+D
这样可以一次性看到编译后的二进制块
