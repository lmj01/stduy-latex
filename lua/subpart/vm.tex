\chapter{Virtual Machine}

从lua1.0就内置了虚拟机,lua脚本并不是直接被lua解释器解释执行,而是先由lua编译器编译为字节码,然后交给lua虚拟机执行。

Lua5.0\cite{tiollua5}实现参考文章

高级编程语言虚拟机是对真实计算机的模拟和抽象,如同真实机器有一套自己的指令集一样,虚拟机也有自己的指令集,按照实现方式,可分为
\begin{itemize}
    \item {Stack-based基于栈的虚拟机需要使用PUSH指令往栈顶压入值、弹出值,其他指令则是对栈顶值进行操作,特点是指令个数多指令长度较短}
    \item {Register-based基于寄存器的虚拟机由于直接对寄存器进行寻址,特点是指令个数少指令长度长}
\end{itemize}

与WebAssemlby等虚拟机一样,Lua的vm指令也分为
\begin{itemize}
    \item {操作码opcode}
    \item {静态参数static arguments}
\end{itemize}

指令在执行时,通常还需要动态参数,一般称作操作数operands。

从汇编的本质是CPU二进制指令的文本形式角度来看,指令集对应的是字节码的二进制形式。

在lua术语中,一段可以被lua解释器解释执行的字节码被称作chunk。

\section{编译期}

广义编译期会把编译过程分为不同的阶段,每个解读通过管道pipeline来串联形成完整的编译期。
主要编译阶段有
\begin{itemize}
    \item {前端Front End}
    \begin{itemize}
        \item {预处理}
        \item {词法分析}
        \item {语法分析}
        \item {语义分析}
        \item {中间代码生成}
    \end{itemize}
    \item {中端Middle End}
    \begin{itemize}
        \item {中间代码优化}
    \end{itemize}
    \item {后端Back End}
    \begin{itemize}
        \item {目标代码生成}
    \end{itemize}
\end{itemize}

Lua语言不支持宏特性、不需要预处理,又是动态语言,语义分析也不需要。
lua编译过程中是一次遍历就生成指令的,并没有对源码和语法结构多次遍历,是通过指令回填技术来实现支持顺序的。

Lua解析与代码生成的内部实现\cite{parsingAndCodeGeneration}