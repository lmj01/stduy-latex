\chapter{Opcode}

与WebAssemlby等虚拟机一样,Lua的vm指令也分为两个部分,操作码opcode和静态参数static arguments。
指令在执行时,通常还需要动态参数,一般称作操作数operands。

WebAssemlby采用了栈式Stack-based虚拟机模型,指令在执行时可以从操作数栈上压入、弹出操作数。而Lua
采用了寄存器式Register-based虚拟机模型,指令在执行时可以操作一个虚拟寄存器,用于读写操作数。指令参数
最主要的作用就是表示寄存器索引,这也是lua虚拟机采用定长指令的原因之一(WebAssemlby采用变长指令)。

理解lua,先要理解它编译后的形态,是指令,就至少要到汇编级别来理解了,毕竟汇编的本质是CPU二进制指令的文本形式。

lua5.3的指令编码如下表所示
\begin{center}
    \begin{tabular}{|c|c|c|c|c|} \hline
       \hbox{} & \hbox{31-24} & \hbox{23-16} & \hbox{15-8} & \hbox{0-7} \\ \hline
       \hbox{iABC} & \hbox{B:9} & \hbox{C:9} & \hbox{A:8}  & \hbox{Opcode:6} \\ \hline
       \hbox{iABx} & \hbox{Bx:18-1} & \hbox{Bx:18-2} & \hbox{A:8}  & \hbox{Opcode:6} \\ \hline
       \hbox{iAsBx} & \hbox{sBx:18-1} & \hbox{sBx:18-2} & \hbox{A:8}  & \hbox{Opcode:6} \\ \hline
       \hbox{iAx} & \hbox{Ax:26-1} & \hbox{Ax:26-2} & \hbox{Ax:26-3}  & \hbox{Opcode:6} \\ \hline
    \end{tabular}
\end{center}
在lua5.4中opcode由6位变成了7位。

编码模式中的字母含义i表示指令instruction,x表示扩展extended,s表示符号signed,字母ABC表示三个参数,A一般用作
目标寄存器索引,B和C既可以是寄存器索引,也可以是常量池索引

\begin{itemize}
    \item {iABC
    \begin{center}
        \begin{tabular}{|c|c|c|c|c|} \hline
           \hbox{} & \hbox{31-24} & \hbox{23-16} & \hbox{15-8} & \hbox{0-7} \\ \hline
           \hbox{5.3} & \hbox{B:9} & \hbox{C:9} & \hbox{A:8}  & \hbox{Opcode:6} \\ \hline
           \hbox{5.4} & \hbox{C:8C:8} & \hbox{B:8-k} & \hbox{A:8}  & \hbox{Opcode:7} \\ \hline
        \end{tabular}
    \end{center}
    }
    \item {iABx}
    \item {iAsBx}
    \item {iAx}
    \item {isJ}
\end{itemize}

基本的四个指令形式,iABC、iABx、iAsBx、iAx。sBx表示signed int。生成lua代码对应的指令的

luac -l -l src.lua

通过上面命令编译后,可以得到编译后的指令代码。
