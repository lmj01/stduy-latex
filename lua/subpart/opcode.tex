\chapter{Opcode}


理解lua,先要理解它编译后的形态,是指令,就至少要到汇编级别来理解了,毕竟汇编的本质是CPU二进制指令的文本形式。

lua5.3的指令编码如下表所示
\begin{center}
    \begin{tabular}{|c|c|c|c|c|} \hline
       \hbox{} & \hbox{31-24} & \hbox{23-16} & \hbox{15-8} & \hbox{0-7} \\ \hline
       \hbox{iABC} & \hbox{B:9} & \hbox{C:9} & \hbox{A:8}  & \hbox{Opcode:6} \\ \hline
       \hbox{iABx} & \hbox{Bx:18-1} & \hbox{Bx:18-2} & \hbox{A:8}  & \hbox{Opcode:6} \\ \hline
       \hbox{iAsBx} & \hbox{sBx:18-1} & \hbox{sBx:18-2} & \hbox{A:8}  & \hbox{Opcode:6} \\ \hline
       \hbox{iAx} & \hbox{Ax:26-1} & \hbox{Ax:26-2} & \hbox{Ax:26-3}  & \hbox{Opcode:6} \\ \hline
    \end{tabular}
\end{center}
在lua5.4中opcode由6位变成了7位。
lua5.3中6bit支持64条指令,定义了47条,lua5.4中7bit有128条指令,定义了83条

操作数的是由剩下的bit来编码动态参数,其编码模式中的字母含义i表示指令instruction,x表示扩展extended,s表示符号signed,字母ABC表示三个参数,A一般用作
目标寄存器索引,B和C既可以是寄存器索引,也可以是常量池索引

\begin{itemize}
    \item {iABC
        \begin{center}
            \begin{tabular}{|c|c|c|c|c|} \hline
            \hbox{} & \hbox{31-24} & \hbox{23-16} & \hbox{15-8} & \hbox{0-7} \\ \hline
            \hbox{5.3} & \hbox{B:9} & \hbox{C:9} & \hbox{A:8}  & \hbox{Opcode:6} \\ \hline
            \hbox{5.4} & \hbox{C:8} & \hbox{B:8-k} & \hbox{A:8}  & \hbox{Opcode:7} \\ \hline
            \end{tabular}
        \end{center}
        大部分指令都是iABC模式,lua5.3中的指令,参数B和C即可以表示寄存器索引,也可以表示常量池索引,如果最高位为1表示常量池索引,
        否则为寄存器索引,如下        
        \begin{lstlisting}
            OP_ADD,/* A B C R(A) := RK(B) + RK(C)*/
        \end{lstlisting}
        lua5.4中B和C仅表示寄存器索引,为了能够使用常量池中的常量,新添加了ADDK指令
        \begin{lstlisting}
            OP_ADD, /* A B C R(A) := R(B) + R(C)*/
            OP_ADDK,/* A B C R(A) := R(B) + K(C)*/
        \end{lstlisting}        
    }
    \item {iABx
        \begin{center}
            \begin{tabular}{|c|c|c|c|} \hline
            \hbox{} & \hbox{31-16} & \hbox{15-8} & \hbox{0-7} \\ \hline
            \hbox{5.3} & \hbox{Bx:18} & \hbox{A:8}  & \hbox{Opcode:6} \\ \hline
            \hbox{5.4} & \hbox{Bx:17} & \hbox{A:8}  & \hbox{Opcode:7} \\ \hline
            \end{tabular}
        \end{center}
        以LOADK指令为例,
        \begin{lstlisting}
            OP_LOADK, /* A Bx R(A) := Kst(Bx) lua5.3 */
            OP_LOADK, /* A Bx R[A] := K[Bx]   lua5.4 */
        \end{lstlisting}                
    }
    \item {iAsBx与iABx基本一样,区别是参数B需要被解释为有符号整数。\textbf{对应有符号整数参数,lua并没有采用二的补码Two's Complement,而是采用了偏移二进制Offset Binary,又称为Excess-K编码格式}}
    \item {iAx指令模式只有EXTRAARG指令使用,这是一条伪指令,不能单独执行,只能跟着特定指令后面,为这些指令提供额外扩展参数,}
    \item {isJ是lua5.4新增的指令模式,只有JMP指令使用此模式,sJ表示跳转的PC偏移量}
\end{itemize}

\section{Excess-K}
这种整数编码方式将可以表示最小负整数编码为全0,表示最大正整数编码为全1,其余整数按大小顺序分布在两者之间。
以4bit整数为例来说明
\begin{center}
    \begin{tabular}{|c|c|c|c|} \hline
    \hbox{} & \hbox{Binary} & \hbox{Two's Complement} & \hbox{Excess-K} \\ \hline
    \hbox{0} & \hbox{0000} & \hbox{0}  & \hbox{-8} \\ \hline
    \hbox{1} & \hbox{0001} & \hbox{1}  & \hbox{-7} \\ \hline
    \hbox{2} & \hbox{0010} & \hbox{2}  & \hbox{-6} \\ \hline
    \hbox{3} & \hbox{0011} & \hbox{3}  & \hbox{-5} \\ \hline
    \hbox{4} & \hbox{0100} & \hbox{4}  & \hbox{-4} \\ \hline
    \hbox{5} & \hbox{0101} & \hbox{5}  & \hbox{-3} \\ \hline
    \hbox{6} & \hbox{0110} & \hbox{6}  & \hbox{-2} \\ \hline
    \hbox{7} & \hbox{0111} & \hbox{7}  & \hbox{-1} \\ \hline
    \hbox{8} & \hbox{1000} & \hbox{-8}  & \hbox{0} \\ \hline
    \hbox{9} & \hbox{1001} & \hbox{-7}  & \hbox{1} \\ \hline
    \hbox{10} & \hbox{1010} & \hbox{-6}  & \hbox{2} \\ \hline
    \hbox{11} & \hbox{1011} & \hbox{-5}  & \hbox{3} \\ \hline
    \hbox{12} & \hbox{1100} & \hbox{-4}  & \hbox{4} \\ \hline
    \hbox{13} & \hbox{1101} & \hbox{-3}  & \hbox{5} \\ \hline
    \hbox{14} & \hbox{1110} & \hbox{-2}  & \hbox{6} \\ \hline
    \hbox{15} & \hbox{1111} & \hbox{-1}  & \hbox{7} \\ \hline
    \end{tabular}
\end{center}


