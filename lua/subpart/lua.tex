\chapter{Lua}

lua\cite{luaorg} 目前是没有标准的,是一门小众化语言,但它优势在于小巧,代码量少。
lua将自己定位为高效、轻量级的可嵌入脚本语言,lua核心是以库的形式被实现,其他应用程序只需要链接lua库就
可以使用lua提供的API来执行脚本的能力,lua发布版包含的两个命令行程序lua和luac。它的源代码量小到足够到去全部理解,用来学习是非常适合的,学习编译原理相关的知识,
都可以通过它来一窥乾坤!

\section{conception}

\paragraph{module \& package}
lua的模块是由变量、函数等组成的table对象,其加载机制通过require来导入,require的路径是全局变量package.path中的值,
lua启动后会取查找,执行print(package.path)可以看到输出的结果

\paragraph{C package}
使用C为lua写包,是动态库的调用方法

\paragraph{closure}
closure对象是lua运行期一个函数的实例对象

\paragraph{proto}
proto对象是lua内部代表一个closure原型的对象,有关函数的大部分信息都保存在这里,

\begin{itemize}
    \item {instructions指令列表,函数编译后生成的虚拟机指令}
    \item {constant table常量表,运行期的所有常量,在指令中,常量通过id在常量表中索引}
    \item {child proto table子proto表,所有内嵌于函数的protoa,指令$OP_CLOSURE$的proto就是在这个表中通过id索引}
    \item {\text{local var desc}局部变量描述,函数使用的所有局部变量名称,以及生命周期}
    \item {\text{upvalue desc函数使用到的upvalue描述}}
\end{itemize}




