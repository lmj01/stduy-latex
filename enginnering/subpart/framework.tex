\clearpage
\part{Framework}

\textbf{你始终被锁定在你的基础架构中,一旦达到一定规模,人们就不会轻易改变基础架构}。
可随着现代编程技术的发展,什么都变成可编程的了,大家也不再需要一次性永久运行的了,而是代码的管理需要
迭代,随时可以生成对应的运行代码,在工程上是安全且可持续发展的必然选择,如果现在的代码管理还跟不上
当前的工具,那就是非常落后的管理了。

\chapter{GNU}

\section{C library}
gcc 内置了很多函数,形式为 \textbf{ \_\_builtin\_XXX}为样式。
GNU的编译器提供了对一些关于位处理的函数
\begin{itemize}
    \item { \_\_builtin\_clz, 统计最高位0的个数 }
    \item { \_\_builtin\_ctz, 统计低位0的个数}
\end{itemize}

\section{GCC}

\begin{lstlisting}
    gcc -std=c90 // spec version for runtime
    # -std=c90 or -std=iso:9899:1990
    # -std=c99 , -std=c11, -std=c17
    g++ -std=c++98 # c++03, c++11, c++14, c++17
\end{lstlisting}

\chapter{NodeJS}

\section{grunt}

