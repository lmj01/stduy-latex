\clearpage
\part{Rust}

Rust programming language

\section{特性}

\subsection{类型系统}
借鉴了Haskell的类型系统特性有
\begin{itemize}
    \item {没有空指针}
    \item {默认不可变}
    \item {代数数据类型}
    \item {高阶函数}
    \item {模式匹配}
    \item {trait和关联类型}
    \item {本地类型推导}
\end{itemize}
借鉴了C++的
\begin{itemize}
    \item {确定性析构}
    \item {RAII}
    \item {智能指针}
\end{itemize}
加上本身独有的特性
\begin{itemize}
    \item {仿射类型Affine Type,用来表达Rust所有权中的move语义}
    \item {借用,生命周期}
\end{itemize}

\subsection{生命周期}
很重要的概念就是生命周期,所有权和移动语义,借用概念。相比C/C++中的默认拷贝赋值,rust中不注意调用一下自己就没啦!
默认情况下是移动语义,很多时候要强调是借用,就像C的指针,C++的引用一样,使用起来时要注意这些基础的概念,是很容易
影响写代码的逻辑的。

struct中很重要的概念是生命周期,结构体中的每个字段都是完整的属于自己的,即每个字段的生命周期最终都不会超过
这个结构体。
\begin{lstlisting}
    struct A {
        attr1: i32,
        attr2: String,
    }
    struct RefBoy<'a> {
        local:&'a i32,
    }
\end{lstlisting}
RefBoy的生命期一定不能比'a更长,即结构体里的引用必须要有显式生命周期,一个辈显式写出的生命周期的结构体,其自身
的生命周期一定小于等于其显式写出的任意一个生命周期。

\subsection{注解}
作为新一代的编程语言,注解这种形式的组织代码已经是必备的功能了,不过会发现rust的
注解上更加灵活些
\begin{lstlisting}
    #[derive(Debug)]
    struct Rectangle {
        width: u32,
        height: u32,
    } 
    fn main() {
        let rect1 = Rectangle { width: 30, height: 40};
        println!("rect1 is {:?}", rect1);
    }
\end{lstlisting}
这里增加注解来派生Debug trait,使用调试格式打印Rectangle实例。

\subsection{宏}