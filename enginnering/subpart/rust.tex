\chapter{Rust}

Rust programming language

\section{特性}

Rust有很多特性是非常现代化的,把数据类型与实现分离的逻辑
\begin{lstlisting}
    #[derive(Debug)]
    struct Rectangle {
        width: u32,
        height: u32,
    }
    impl Rectangle {
        fn area(&self) -> u32 {
            self.width * self.height;
        }
    }
\end{lstlisting}
可以看到,struct中的只有属性,所有方法都放在impl模块中去完成,这种解耦相比其他的OOP来说
是非常好的优点。

\subsection{语法}

\subsubsection{自动引用和解引用}
automatic referencing and dereferencing, 
相比C/C++的直接和指针调用方法需要手动转换类型,Rust根据self类型自动推断出调用方法时的类型需要

\subsubsection{enumerations}

枚举,允许通过列举成员variants来定义一个类型, Rust的枚举与F\#、OCaml、Haskell等函数式编程
语言中的代数数据类型algebraic data types相似。

与C/C++的struct相比,Rust的enum可以有类似struct的功能,可对任意类型进行绑定
如标准库中的IpAddr
\begin{lstlisting}
    struct Ipv4Addr {
        // ---snip---
    }
    struct Ipv6Addr {
        // ---snip---
    }
    enum IpAddr {
        V4(Ipv4Addr),
        V6(Ipv6Addr),
    }
\end{lstlisting}

Rust没有空值null的概念,通过枚举Option来表示有值和没值,在逻辑表达上具有一直的接口
\begin{lstlisting}
    enum Option<T> {
        Some(T),
        None,
    }
\end{lstlisting}
通过match语法来使用enum是非常好的方案,因为额外一层来封装保证接口的一直,所以T类型值是不能
直接获取的。而match使用匹配模式来执行代码,模式可由字面值、变量、通配符和许多其他内容构成。
且Rust的匹配是穷尽的exhaustive,即必须穷举完所有情况,不会出现空的。rust提供了通配符 \_ 来处理。


\subsection{类型系统}
借鉴了Haskell的类型系统特性有
\begin{itemize}
    \item {没有空指针}
    \item {默认不可变}
    \item {代数数据类型}
    \item {高阶函数}
    \item {模式匹配}
    \item {trait和关联类型}
    \item {本地类型推导}
\end{itemize}
借鉴了C++的
\begin{itemize}
    \item {确定性析构}
    \item {RAII}
    \item {智能指针}
\end{itemize}
加上本身独有的特性
\begin{itemize}
    \item {仿射类型Affine Type,用来表达Rust所有权中的move语义}
    \item {借用,生命周期}
\end{itemize}

\subsubsection{泛型}
Rust通过在编译时进行泛型代码的单态化monomorphization来保证效率,
单态化是一个通过填充编译时使用的具体类型,将通用代码转换为特定代码的过程。


\subsection{module system}
rust提供了更加完善的模块化编程支持。
\begin{itemize}
    \item {packages包}
    \item {Crates}
    \item {Modules \& Use}
    \item {Path}
\end{itemize}



\subsection{生命周期}
很重要的概念就是生命周期,所有权和移动语义,借用概念。相比C/C++中的默认拷贝赋值,rust中不注意调用一下自己就没啦!
默认情况下是移动语义,很多时候要强调是借用,就像C的指针,C++的引用一样,使用起来时要注意这些基础的概念,是很容易
影响写代码的逻辑的。

struct中很重要的概念是生命周期,结构体中的每个字段都是完整的属于自己的,即每个字段的生命周期最终都不会超过
这个结构体。
\begin{lstlisting}
    struct A {
        attr1: i32,
        attr2: String,
    }
    struct RefBoy<'a> {
        local:&'a i32,
    }
\end{lstlisting}
RefBoy的生命期一定不能比'a更长,即结构体里的引用必须要有显式生命周期,一个辈显式写出的生命周期的结构体,其自身
的生命周期一定小于等于其显式写出的任意一个生命周期。

\section{Attribute}

属性是作用在Rust语言元素上的元数据,有内部与外部之分

\begin{description}
    \item [Inner] \text{一个属性声明在一个元素中,可作用于整个元素内,如crate的属性使用\#![]声明}
    \item [Outer] \text{一个属性声明在一个元素之前,作用于声明之后的元素。使用\#[]声明}
\end{description}

Rust的属性类别有
\begin{itemize}
    \item \text{Macro Attribute, 宏属性}
    \item \text{Derive macro helper attributes 派生宏辅助属性}
    \item \text{Tool attributes工具属性}
    \item \text{Built-in attributes内置属性}
\end{itemize}

\begin{lstlisting}
    #[derive(Debug)]
    struct Rectangle {
        width: u32,
        height: u32,
    } 
    fn main() {
        let rect1 = Rectangle { width: 30, height: 40};
        println!("rect1 is {:?}", rect1);
    }
\end{lstlisting}
这里增加注解来派生Debug trait,使用调试格式打印Rectangle实例。

