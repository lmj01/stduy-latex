\clearpage
\part{Reverse Enginner}
逆向工程,是软件中的一个核心技能,是一个双刃,既能防御,又能破解别人的技术。

软件逆向工程,是指从可运行的程序系统出发,运用解密,反汇编,系统分析,程序理解等计算机技术,对软件的结构,流程
,算法,代码进行逆向拆解和分析,推导除软件产品的源代码,设计原理,结构,算法,处理过程,运行方法及相关文档的过程。
需要各方面的知识体系,分三大类
\begin{itemize}
    \item {基础,程序语言应具备:汇编,C/C++,以及逆向工具的脚本语言,如Python等}
    \item {计算机体系,了解程序底层的运作机制,把猜测范围缩小范围,如寄存器,内存模型,堆栈,处理器相关知识}
    \item {领域,加密解密,网络原理,内核编程,反调试和代码混淆}
\end{itemize}

\chapter{Tool}

逆向最主要的方法就是通过调试分析,反调试,而且在熟练使用工具之于还必须做到快速识别能力,
即调试到关键阶段,却不能猜测出模式与原理,基本无望于结果!

\section{windbg}
内核调试器

\section{ollydbg}
动态调试器

\section{IDA pro}
静态反汇编

\section{RenderDoc}
GPU图形shader的调试工具


\section{步骤}

\subsection{研究保护方法}
开发者为了维护关键技术不被侵犯,采用了各种软件保护技术。
\begin{itemize}
    \item {序列号}
    \item {加密锁}
    \item {反调试技术}
    \item {加壳}
\end{itemize}

\subsection{反汇编目标软件}
通过动态与静态调试分析相结合,跟踪,分析软件的核心代码,理解软件的设计思路,
获取关键信息。

\subsection{生成目标软件的相关文档}
获取的关键信息反推,设计思想,架构,算法等文档
或如何扩展目标软件的功能

\subsection{向目标软件注入代码}
打补丁破解,或增强功能


\section{源码分析}
开源项目越来越多,如何去理解以及破解核心开源的软件组件,也需要大量知识体系。
从宏观到微观,注重大体框架,一步步去理解
