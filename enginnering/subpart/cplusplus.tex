\chapter{C++}

\section{ C++ 核心概念 }
C++引入一个核心概念ADT,Abstract Data Type.是为类型的属性和类型的操作方法提供了一个抽象
的描述,且不受特定的实现和编程语言的约束。C++使用class关键字来封装数据和函数方法。

\subsection{对象模型}
对象模型的发展有几个阶段,分别是

Simple Object Model,所有的member都不放在object中,只保存指向member的指针。这个模型
抽象得太理想,很容易理解,但实际中没有应用。

Table-driven Object Model,是为了对所有class的对象object都有一致的表达方式,让object
内含两个表格指针,分别指向
\begin{itemize}
    \item {member data table}
    \item {function member table}
\end{itemize}

C++的实际模型是继承了简单对象模型和表格驱动模型的优点,满足在空间和存取的时间效率,
缺点也存在,就是class内的一些改动就会导致需要重新编译,影响开发效率。模型结构大致如下:
\begin{itemize}
    \item {non-static data member, 都在class之内}
    \item {static data member,都在class之外}
    \item {function member都在class之外}
    \item {virtual function,与其他不同。每个class有多个指向virtual function的指针,都放在
    一个表格中,称为virtual table(vtbl),每个class内含一个指针指向这个virtual table, 这个指针
    称为vptr}
\end{itemize}

\subsection{成员函数}
构造函数,只在特定情况下会由编译器构建:
\begin{itemize}
    \item {class A中的member是class B的对象,B有默认构造函数,编译器会给A生成一个默认构造函数,
    用来调用成员class B的对象的初始化,其他member是不会自动初始化的,是程序员的负责对它们进行初始化。
    需要避免多个默认构造函数,使用inline}
    \item {class A的基类有默认构造函数,编译器会生成A的默认构造函数,以调用基类的默认构造函数来完成
    初始化。如果基类没有,编译器不会为A生成默认构造函数}
    \item {class A存在虚函数,编译器会默认生成构造函数,构造vtbl和vptr}
    \item {class A存在虚基类,编译器会生成默认构造函数,以初始化虚基类的vtbl}
\end{itemize}

拷贝构造函数,在以下情况下会由编译器生成,
\begin{itemize}
    \item {显示对class A进行初始化}
    \item {class A作为函数的参数}
    \item {函数返回一个class A}
\end{itemize}
位逐次拷贝,bitwise copy semantics。即class A中不存在class member对象时,拷贝就是对class
A中的每个bit进行拷贝,即POD(Plain Old Data),这时支持static initializtion,内存布局和对应的
C语言的struct布局相同。

\subsection{多态机制}
case这个概念是一种编译器指令,它只影响编译器如何解释指针指向的内存大小和内容。

\subsection{struct}
很多适合并没有弄明白C++中的struct和C中的struct的区别,听到更多的是以C++中的概念来理解。
\begin{lstlisting}
    struct S1 {
        int member;
    };
    S1 s1; // variable declare
    struct S2 {
        int member;
    }S2; // S2 is a variable, directly
    S2.member = 1; // directly access the global value
    typedef struct tagS3 {
        int member;
    }S3; // S3 is structure type
    S3 s3; // declare variable 
    s3.member; 
\end{lstlisting}

\subsection{value-category}
C++的表达式具有两个属性,一个是type,一个是value category,判断一个表达式的特征就是参考value-category
来确定的。

C++11之前,只有两类,lvalue和rvalue,即根据赋值符合的左右来区分,左值是表达式结束运算后
依然存在的持久对象,右值是表达式结束后就不再存在的临时对象。

C++11开始,引入了移动语义move semantic后,产生了语义上需要。\textbf{注意,计算机语言是计算机科学与语言学的
交叉科目,很多概念来自语言学内容}。先从自然语言中举个例子来说明
\begin{enumerate}
    \item {他是饭桶}
    \item {这是饭桶}
\end{enumerate}
上面两句话都有饭桶这个词,但是意思是不一样的,语法形式上相同的语句,却因代词不同结构完全无关。
说明一条语句不能仅靠词来确定,还需要整个句子的逻辑性,语义性。C++中也有类似的表达式:
\begin{enumerate}
    \item {vec = vect<int>()}
    \item {vec = another-vec}
\end{enumerate}
形式也是一样,但第一句的语义是表示移动赋值,第二句的语义表示复制拷贝一份。所以在C++11中,重新
定义了value categories:
\begin{itemize}
    \item {是否具有identity,即是否可以与另一个表达式进行比较,具有identity同一性的才具有比较
    的意义,如内存地址,堆栈上的变量都具有identity}
    \item {是否呈现移动的语义}
    \begin{itemize}
        \item {lvalue, 有identity且无移动语义}
        \item {xvalue,有identity且有移动语义}
        \item {prvalue,pure rvalue,无identity且有移动语义}
        \item {glvalue,有identity的表达式}
        \item {rvalue,有移动语义的表达式。如对象成员表达式,object.member,如果member是枚举或非
        静态成员函数,则object.member就是prvalue。因为枚举编译后是要给具体的数字,如5,成员函数
        在编译后是一个相对地址,本质上也是一个具体的数字,CPU使用这些数字时,是直接放在指令内部或
        寄存器中,不会放在内存中,即没有identity}
    \end{itemize}
\end{itemize}

\subsection{CRTP}
Curiously Recurring Template Pattern,奇异递归模板模式。1980年作为F-bound polymorphic
被提出
\begin{itemize}
    \item {静态的多态,无法动态绑定。CRTP强迫继承接口,可与Concept这个概念结合来理解}
    \item {多态的转换调用是在编译期时解决,而不是留到运行期处理,性能得到提高。CRTP起到了虚函数的效果,
    却无虚函数的开销,因为虚函数需要存储虚函数指针,调用时通过虚函数指针查找虚函数表进行调用}
    \item {std::enable\_shared\_from\_this针对std::shared\_ptr就是实例,WTL微软的轻量级窗体库}
\end{itemize}

\section{Template Metaprogramming}

模板应用在两个类型,\textbf{Function Templates}和\textbf{Class Templates}。模板编程是一种plugin-in的运作方式,要分清运行期与编译期的
结果,如果对代码在这两个区间分不清就不能完全控制代码的能力。

\subsection{Type Traits}
It can apply transformation to a type, for example, given \textbf{T}, you can add/remove the \textbf{const} specifier.
A type trait is a simple template struct that contains a member constant, which in turn holds the answer to the question
the type traits asks or the transofrmation it performs.
\begin{lstlisting}[language=c++]
    #include<iostream>
    #include<type_traits>
    class Class{};
    int main() {
        std::cout << std::is_floating_point<Class>::value << std::endl; // 0
        std::cout << std::is_floating_point<float>::value << std::endl; // 1
        std::cout << std::is_floating_point<int>::value << std::endl; // 0
        return 0;
    }
\end{lstlisting} 
编译期做的工作就是为每个type生成对应的traits,这些都是在编译期完成的,这种泛型抽象能力与zero overhead是模板编程的最大优势
\begin{lstlisting}[language=c++]
    struct is_floating_point_Class {
        static const bool value = false;
    }
    struct is_floating_point_float {
        static const bool value = true;
    }
    struct is_floating_point_int {
        static const bool value = false;
    }
\end{lstlisting} 
理解type traits最大的问题就是在于编译期帮忙做了很多事情,而对抽象的原理不理解就更会感觉到恐惧,对未知的恐惧。

\subsubsection{Concepts}
类似于OO中的interface,如果一个class实现了interface,就可以当作interface来使用。concept是一组操作,如果
一个type具有这些操作,就说type是这个concept的一个model。

主要是为了简化泛型编程,一个concept就是一个编译期判断,用于约束模板参数,concepts就是编译期条件的合集。

\subsubsection{Introspection}
内省It is the ability of a program to examine the type or properties of an object.
如很多语言带有的getter与setter调用。

\subsubsection{Reflection}
反射It is the ability of a program to observe and alter its own structure or its behavior.

\subsubsection{Tag Dispatch}
Empty classes can be useful in c++ because it is strongly typed language. If there are two 
empty classes, they are different types.
\begin{lstlisting}[language=c++]
    class Empty{}; // a unique type

    template<bool>
    struct UseSemantics{};

    void function1(int a, UseSemantics<true>{});
    void function2(int b, UseSemantics<false>{});    
\end{lstlisting} 