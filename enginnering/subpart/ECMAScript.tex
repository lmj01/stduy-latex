\chapter{ECMAScript}

European Computer Manufacturers Association制定ECMA-262标准化的脚本程序语言

\section{JavaScript}

\subsection{标准}
\begin{itemize}
    \item {1998, 2.0}
    \item {1999, 3.0}
    \item {2007, 4.0}
    \item {2009, 5.0}
    \item {2015, 6.0}
\end{itemize}

\subsection{Execution Contexts}
ECMAScript的规范中说,at any point in time, there is at most one execution context per agent
that is actually executing code.
下面这段参考了Vue的文档描述的关系进行整理的。
JS是执行是单线程的,它是基于事件循环的,大致分为以下几个步骤:
\begin{itemize}
    \item {1, 所有同步任务都在主线程上执行,形成一个执行栈execution context stack}
    \item {2, 主线程之外,还存在一个任务队列task queue,异步任务完成后就产生一个事件放在队列中}
    \item {3, 一旦执行栈中的所有同步任务执行完毕,就去处理任务队列,执行队列等待的任务}
    \item {4, 主线程不断重复上面的过程}
\end{itemize}
主线程执行的就是一个tick过程,而异步就是一个task,规范中规定task为两大类:
\begin{itemize}
    \item {macro task, 宏任务, setTimeout, MessageChannel, postMessage, setImmediate}
    \item {micro task, 微任务, MutationObserver, Promise.then}
\end{itemize}
它们执行逻辑顺序如下,处理完一个宏任务就需要把所有的微任务处理掉!宏任务的耗时
大于微任务耗时,在浏览器上优先使用微任务,再使用宏任务,且各个浏览器的支持也不一致,
可能存在偏差。
\begin{lstlisting}
    for (marcoTask of marcoTaskQueue) {
        handleMacroTask();
        for (microTask of microTaskQueue) {
            handleMicroTask(microTask);
        }
    }
\end{lstlisting}
在ECMAScript2018的规范中说Agent属于一个running execution context。task对应的是job概念。

JavaScript, TypeScript, AssemblyScript

\chapter{TypeScript}
微软主导的一个语言,向后兼容JavaScript,它使得JavaScript支持超大量的工程化。