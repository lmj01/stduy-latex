\chapter{Compiler}
编译器的技术就是翻译人与机器的沟通,体现在两方面,一是工程化需要,需要大量的交流,不能简简单单的操作硬件来完成;
一是简化人的交流成本,特别在指定领域,快速的结果是推导前进的基础。

\section{Conception}

\subsection{bootstrap}
bootstrap自举
用低级语言先实现一个简单的编译器,然后用这个编译器的语言再去编写一个更高级的编译器.
新编译器就是旧编译器的扩展,这个过程称为自举.

\subsection{Cross-Compilation}
交叉编译
不是每个设备都像PC这样便利,为其他设备进行软件开发的过程需要的技术。分两类
\begin{itemize}
    \item {\textbf{小设备},如嵌入式或手机,在PC上完成编译过程,把打包的二进制放入小设备内存中运行.
    本质就是 PC 运行小设备的编译器进行开发,这样避免掉小设备的内存、开发环境、效率等限制}
    \item {\textbf{跨平台},使用Linux编译一个程序,移植portable到window上.本质就是源设备上的编译器,
    具有生成目标设备上的机器代码的功能}
\end{itemize}

\section{Type System}

Statically typed静态类型
静态编程语言,如C、C++等

Dynamically typed动态类型
如JavaScript、lua等

Gradual typing渐进类型\cite{AbstractingGradualTyping}
如Java、C\#等,动态与静态结合,通过Type Annotated类实现部分动态

\subsection{JIT}
Just In Time,即时编译,混合了静态编译与动态解释,将频繁执行的代码编译成机器码缓存起来,对于执行一次的代码任然是
逐行逐条解释执行的技术。
\newline
\textbf{Method JIT},方法即时编译,如Java。主要的步骤如下
\begin{itemize}
    \item {跟踪热点函数,编译成机器码执行,并缓存}
    \item {非热点函数解释执行}
\end{itemize}
\textbf{Trace JIT}, 跟踪即时编译,如Lua,相比Method JIT,它不检测和优化热点函数,而是检测并优化热点跟踪或执行路径。

\section{GC}

Garbage Collection

\subsection{mark-and-sweep}
系统管理所有创建的对象,每个对象都有对其他对象的引用,root集合代表着已知的系统级别的对象引用。从root集合出发,就可以访问
到系统引用的所有对象,而没有被访问的对象就是垃圾对象,需要被销毁的,状态分为
\begin{itemize}
    \item {white,待访问状态,还未被gc访问到}
    \item {gray,待扫描状态,已被访问到,但它对其他对象的引用还未访问到}
    \item {black,已扫描状态,对象关联的都已访问到}
\end{itemize}
\begin{lstlisting}
    all object set white
    visit root set to  put gray set,
    make white to gray state 
    while gray set no empty {
        fetch object from gray set, object set black 
        for (obj in all object referenced by object) {
            if obj is white 
                obj from white to gray, 
                and add to gray set 
        }
    }
    for any object {
        if object is white {
            destory object 
        } else {
            set object white state
        }
    }
\end{lstlisting}

\section{BNF}
Backus-Naur Form巴科斯范式,用来描述计算机语言语法的符号集,是一种典型的元语言,它严格地表示语法规则,
且描述的语法与上下文无关的,它的扩展是ABNF 
\cite{ABNF}

\subsection{expression}
表达式在编程语言中代表一个可以返回值的语法单位,在不同语言中有不同的形式:
\begin{itemize}
    \item {函数式编程语言,大多数语句都是表达式}
    \item {命令式编程语言,一般将语句分成表达式和陈述句statement}
    \item {常量表达式,在BNF中属于终结符}
\end{itemize}

\section{Tools}
Lex \& Yacc是一套生成语法的工具,可以快速方便成型。相对于手写的词法器,具有一定的优势。
