\chapter{assembly}
老实说,汇编是没有学到家的,相当于熟悉了一遍它的常用语法,没有实战经验,最主要的是想
学的底层技术,并没有融会贯通于硬件层.
\newline
多年后(2019.6)才理解,微处理器是管理计算机的算术、逻辑、控制等,都有一套专门的指令码,在汇编中称为
opcode,操作码,操作码是二进制形式的,如00000011是加法指令,对应着的文本形式就是ADD,即\textbf{汇编语言是
二进制指令的文本形式,与指令存在一一对应的关系,把汇编的文本转换为二进制后就是CPU的指令码}。

\section{Conception}

\paragraph{Primitive \& Compound}

当年学汇编时,没有学到位,且汇编中的混淆不清的概念影响到后来的学习语言。

任何一门语言都有Primitive Value元值和Compound Values组合值。 这两个概念不能牢记,对于内存块的布局就
不能谈理解到位,每种不同的语言处理这两种的区别都是非常大的,很大例子都是元值,而我的思维很容易就只记住了
这种范式,就没有考虑组合值的处理逻辑。

引入value assigned by value-copy or by value-reference两种概念,什么时候需要copy,什么时候引用,在任何
语言中都会有不同的形式,而在汇编语言中都不关心,都是一个地址而已,这也是这些年来学习其他语言时受到的问题,在
学C时这些就没有吃透,加上汇编在C之后,很多概念就慢慢地沉积下来了。

shallow vs deep copies。上面提到了copy,一个primitive是很容易copy的,但是一个compound就很难处理了,它可能递归
处理很多数据,而且这些数据可能还与某些state状态有关,如共享,竞争等情况下问题就又变复杂了。

上面又会引入一个mutable和immutable, 那些是可以在源地址上进行assigned或modified。特别是高级语言中Object的
含义又是一个很广泛的词,在不同的上下文与语境中理解是存在差异的,混用几种编程语言后就会感觉到很难弄明白,特别是
不了解底层知识,概念的原理,实现上的细节,对此可以说完全是糊糊涂涂地在依葫芦画瓢版用编程语言。

