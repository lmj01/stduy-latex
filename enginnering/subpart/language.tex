\chapter{programming language}
编程语言是工具,是服务于软件工程的最基本的工具。针对不同的需要才会产生对应的语言工具,
没有一种语言是万能的,是通用的,每一门编程语言都是特定领域的特定产物。

\section{异常}
异常,一般指程序运行期发生的非正常情况,一般是不可预测的,如内存不足,除零等。
它与软件编程思想、编程语言的发展是与时俱进的,核心思想就是把功能代码模块与系统中
可能出现的错误处理代码分离开来,以此达到代码组织结构更加美观、逻辑清晰,
且对软件系统长时间运行提供保障.

\subsubsection{goto}
goto本质上是对异常处理的思维,也是最初最原始的方法,有点很多,这里列出它的劣势:
\begin{itemize}
    \item {在函数的局部作用域内跳转}
    \item {破坏程序的结构化设计,代码难以测试}
\end{itemize}

\subsubsection{setjmp}
c语言中提供了setjmp和longjmp函数,是结构化异常的基础,首先设置一个跳转点setjmp可以实现,
然后在代码的其他任意地方调用longjmp函数跳回到这个点上。
\newline
支持这些功能的是有一个异常结构体,用于保存当前程序的现场。
\begin{lstlisting}
    typedef struct {
        unsigned j_sp;
        unsigned j_ss;
        unsigned j_flag;
        unsigned j_cs;
        unsigned j_ip;
        unsigned j_bp;
        unsigned j_di;
        unsigned j_es;
        unsigned j_si;
        unsigned j_ds;
    }jmp_buf;
\end{lstlisting}
setjmp和longjmp都与这个结构有关,setjmp还是一个能返回两次的函数,第一次是调用时,第二次是longjmp时。
\begin{lstlisting}
    #include <stdio.h>
    #include <setjmp.h>
    jmp_buf jumper;
    int div(int a, int b) {
        if (b == 0) {
            longjmp(jumper, -3); // the setjmp 
            //call second with the second parameter
        }
        return a / b;
    }
    int main(int argc, char **argv) {
        // longjmp call positon 
        int retcode = setjmp(jumper); 
        if (retcode == 0) {
            printf(" div result is %d\n", div(2, 0));
        } else if (retcode == -3) {
            printf(" divide by zero");
        } else {
            printf(" unhandled error case");
        }
        return 0;
    }
\end{lstlisting}

\subsubsection{全局处理}
C++,Java等高级语言的异常处理机制,是借助全局进行处理,并没有专门提供能够有效区分正常逻辑和错误逻辑的语法,
所有的非正常都当作异常处理,且还会带来较大的性能开销。

\section{调用约定}
函数调用约定,是对函数调用的一个约束和规范,包含
\begin{itemize}
    \item {函数参数的压栈顺序}
    \item {由调用者还是被调用者把参数弹出栈,调用者自己清理堆栈上的参数的,
    可以实现可变参数列表功能}
    \item {产生函数修饰名的方法}
\end{itemize}

\subsection{cdecl}
C declaration,C声明是C语言的一种调用约定,事实上的标准。参数按照从右向左的顺序
依次压栈,调用者负责清栈。不同编译器不同平台实现的细节上是存在差异的。可编译成对应
的汇编代码来分析。函数名加一前缀(下划线字符)。

\subsection{pascal}
基于Pascal语言的调用约定,参数从左向右入栈.被调用者负责在返回前清栈.

\subsection{stdcall}
由微软创建的调用约定,是Windows API的标准调用约定。它结合了cdecl和Pascal两者优势,
参数从右向左入栈,被调用者负责清栈。编译后函数名添加后缀@length,length是传递函数参数
所占的栈空间的字节长度.

\section{语言特性}
特性,即概念的描述方式,王银在他的文章《如何掌握所有的程序语言》中描述的那样:
\textbf{如果你不能用一种语言里面的基本特性写出好的代码,那你换成另外的一种语言也
无济于事。是否能写出好的代码在于人,而不在语言。如果你心中没有清晰简单的思维模型,
你用任何语言特性表述出来的都是一堆乱码。}

而每个语言特性尽量从古老语言中追溯,新语言的特性完全是封装了一个新的概念在其上,编程
语言比自然语言简单清晰,也更加严谨和形式化。所以不要想着用尽语言提供的特性,而是使用
经过千锤百炼的特性才能保证稳定性。

\subsection{指针}
在C,C++语言中,指针是一个很灵活的方法,但不是所有语言都有这种特性,也不适合一些语言。
它们使用指针是因为它们可以直接与硬件进行沟通,而在现代化的快速应用中,很多软件都是建立在
一个抽象层上面进行开发的,就是为了确保与硬件无关,所以这些语言根本不需要指针这种概念,如
Java就没有指针概念!因为它是建立在C,C++编写的应用上面的一种抽象语言。

\subsection{mutable与immutable}
\textbf{mutable}与\textbf{immutable}这两个概念是函数式编程语言中是很基础的概念,它们对变量加强了说明,
是在语义上进行改进的语言。也是区分静态类型与动态类型语言的一个指标,静态类型语言是要严格区分的!

C++中默认是可变的,不可变的需要加上const,这是比较反常的。这也是C/C++最令人痛苦的过程就是,在很多地方的默认
是让人不注意的,特别是学习过程中没有入坑解决这些问题,是很难做到条件反射的,总是会产生错觉!

在Rust这样的语言中,为确保类型安全,明确了语言的语义,在语义上通过编译器来保证逻辑上的清晰,既有函数式风格,
也有系统编程语言的能力,避免存在歧义的语句存在,每一条语句都需要明确到位。

\subsection{constant}
常量,在它的范围内scope只声明一次,不可更改的变量。
常量表达式,就是在编译期可计算的值,而不是运行期中计算得到的值。

\subsection{函数}
特性有如下特性:
\begin{itemize}
    \item {函数是一阶值,First-class valuue, 即函数可以作为另一个函数的返回值或参数,也可作为变量值}
    \item {函数可以嵌套定义,即一个函数内部可以定义另一个函数}
    \item {可以捕获引用环境值}
    \item {允许定义匿名函数}
    \item {把引用环境和函数代码组成一个可调用的实体}
\end{itemize}
根据不同的需求,可以变化为
\begin{itemize}
    \item {lambda,匿名函数}
    \item {Closure,闭包,又称词法闭包lexical closure或函数闭包function closure。
    是由支持高价函数特性的语言技术,是实现静态作用域的一种方式,将函数与声明时的作用域保存
    下来,在被调用时的有效作用域是在声明时的,而不是调用执行时的}
\end{itemize}

\subsection{inline}
内联,短小的函数不存在函数调用的开销,是因为现代编译器都能自动把小函数inline到被调用的地方。
早期的C语言编译器里,宏就是有这么一个功能,替换小函数,这也是大量使用宏的原因。

\subsection{FFI}
Foreign Function Interface, 是指与其他语言交互的接口,现实中的程序基本没有单语言的软件啦!
跨语言调用就成了一门语言的必然趋势,常用方法有两种
\begin{itemize}
    \item {将函数做出一个服务,通过进程通信IPC或网络通信协议RPC,RESTful等方式进行,至少需要两个进程}
    \item {直接FFI调用,直接将其他语言接口内嵌到语言中进行调用}
\end{itemize}
大多数属于兼容C ABI的实现。

\paragraph{生成器}
生成器generator是一种特殊的迭代器,是一个函数,能多次返回的函数,即遇到yield就是返回,下次执行函数时从上一次yield的地方继续执行,这种机制就称为生成器。

yield关键字有两点作用
\begin{itemize}
    \item {保存当前运行状态,断点处,然后暂停执行,将生成器函数挂起}
    \item {将yield关键字后面表达式的值作为返回}
\end{itemize}

\paragraph{协程}
协程是指具有这些函数
\begin{itemize}
    \item {彼此间有不同的局部变量,指令指针,但任共享全局变量}
    \item {可以方便挂起,恢复,并且有多个入口和出口点}
    \item {多个协程间表现为协作运行,同一时刻只能有一个协程运行,即无法并发,不含多线程情况}
\end{itemize}

\paragraph{多态}
在编程语言和类型论中,多态polymorphism指为不同数据类型的实体提供统一的接口。
多态类型polymorphic type可以将自身所支持的操作套用到其他类型的值上。

而据派发dispatch发生时间的不同,多态分为静态(编译时)和动态(运行时)。

C++中的虚继承就属于动态多态,它实现的方式是为继承类加入一个域vptr,指向vtbl。继承类调用方法时是先通过vptr到vtbl中找到函数再调用。
C++中的模板与重载是静态多态,在编译期完成模板展开,直接链接到对应的函数,消除了虚函数的开销。

Rust中引入的trait系统解决了C++中的缺点。

\subsection{Attribut}

属性规范\cite{ECMA334} Meta Item Attribute Syntax

\begin{tabular}{|c|c|}
    \hline
    Style & Example \\ \hline
    \hbox{Meta Word} & \hbox{no\_std} \\ \hline
    \hbox{Meta Name Value Str} & \hbox{doc="example"} \\ \hline
    \hbox{Meta List Path} & \hbox{allow(unused, clippy::inline\_always} \\ \hline
    \hbox{Meta List Idents} & \hbox{macro\_use(foo,bar)} \\ \hline
    \hbox{Meta List Name Value Str} & \hbox{link(name="CoreFoundation", kind="framework")} \\ \hline
\end{tabular}

所有属性语法都是以上的一种或其组合。一个属性必然是下面两种之一,在预处理过程中有不同的行为:

\begin{description}
    \item [active] \text{在处理属性的过程中,删除它们自己,留下所作用的元素}
    \item [insert] \text{在处理属性的过程中,保留它们自己}
\end{description}

\paragraph{修饰器Decorator}
用来修饰类,改变类的行为的。是代码编译时发生的,而不是在运行时,即在编译期阶段
就执行修饰器的作用。动态语言有这个特性,是让编译期来自动处理一些操作,比起静态
语言C来说,相当于宏的作用。

在另一个层面来说,这就是通过编译期提供额外的功能来处理用户的代码,把需要手写
的代码都让编译期代劳。
\begin{itemize}
    \item {Java, Annoation}
    \item {Python/EMACScript, Decorator}
    \item {C\#, Attribute}
    \item {C/C++, Marco}
    \item {rust, derive的Trait}
\end{itemize}


\subsection{编程范型}
The principal programming paradigms.\cite{TPPP}列出了编程范型。

\paragraph{过程}

\paragraph{结构化}

\paragraph{面向对象}

\paragraph{泛型}

\paragraph{函数式}

函数式思想是很重要的,对于第一门语言是C的我来说,有一个很大的障碍来自,学习C语言时,把重要性过渡于强调它的普遍性与通用性。
而当时被灌输时,并没有非常系统化介绍,基本是国外教科书的照搬,很多概念都在纠结,这些概念其实在工作中并没有多大的作用,毕竟
大多数的人还是以业务为导向。

C语言的函数指针呀,后来的Javascript,Lua,Python等都是非常强调这些功能的,对我而言还是非常简陋的被介绍而已!

\paragraph{并发}

\paragraph{分布式}

\paragraph{DSL}
Domain Specific Language,领域编程语言,基于特定领域开发相关的编程语言。

\paragraph{IDL}
interface description language, or interface definition language,
接口描述(定义)语言,是一种语言规范,用来描述组件的API。在跨语言,RPC方面上
大量应用。

\paragraph{Glue code}
glue code language,胶水语言,

\subsection{多线程}

\paragraph{atomic operation}
原子操作是不需要synchronized,是指不会被线程调度机制打断的操作,即这种操作一旦执行,就会
一直运行到结束,中间不会有任何context switch。

\subsection{AOP}
面向切面编程,就是在现有代码程序中,在程序生命周期或者横向流程中,加入/减去一个或多个功能,
且不影响原有的功能。AOP这种非侵入扩充对象,方法和函数行为的技术,在spring中大量使用,
在Javascript中也可大量使用,无侵入使得编程时的视野开阔啦,相比与C语言之类的hook函数。
\begin{itemize}
    \item {非侵入统计代码}
    \item {分离表单的请求和校验}
    \item {职责链式函数调用}
    \item {组合代替继承}
\end{itemize}

