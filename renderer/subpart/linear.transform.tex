
\clearpage
\section{线性变换}
线性变换主要包括缩放、旋转、反射等,不包含平移。线性变换的数学描述为:

\begin{math}
\begin{aligned}
f(\alpha v)=\alpha v, \\
f(u+v)=f(u)+f(v)
\end{aligned}
\end{math}

反射,就是把一个物体变换成它的镜像的映射,在二维空间中,用一条直线作为“镜子”;在三维空间中,使用平面作为“镜子”。

\subsection{ 坐标coordinate}

\subsubsection{重心坐标}
barycentric coordinate.


\subsection{ 矩阵matrix }
矩阵就是线性变换
\subsubsection{特征向量eigenvector}
对于一个给定的线性变换(矩阵M,是向量空间E到自身的一个线性变换,可以旋转、反射、拉伸、压缩或变换的组合),它的特征向量V经过这个线性变换之后,得到新的向量U,VU向量共线,但长度可能不一致。
VU的长度缩放的比例称为特征值eigenvalue。

\subsubsection{余子式minor}
在n阶行列式D中划去任意选定的k行、k列后,余下的元素按照原来的顺序组成的n-k阶行列式M,M称为
行列式D的k阶子式A的余子式Minor。如果k阶子式A在行列式D中的行和列的标号分别为
$i_{1},i_{2},...,i_{k}$和$j_{1},j_{2},...,j_{k}$。则在A的余子式M前面加一个符号:
\begin{math}
    \begin{aligned}
        {-1}^{(i_{1}+i_{2}+...+i_{k}) + (j_{1}+j_{2}+...+j_{k})}
    \end{aligned}
\end{math}
得到的n-k阶行列式,称为行列式D的k阶子式A的代数余子式Cofactor。

\subsubsection{行列式determinant}
行列式其实是一个函数,一个将方阵转换成一个标量的函数。行列式可以看做是有向面积的概念在一般的欧几里得空间中的推广。
行列式表示的是线性变换前后的面积(二维)、(三维为体积)等变化的系数。

n阶行列式det(D)等于其任意行(列)的元素与对应的代数余子式的乘积之和。通过D的i行row来计算

\begin{math}
    \begin{aligned}
   \det(D)=d_{i1}A_{i1}+...+d_{in}A_{in}=\sum_{j=1}^{n}{a_{ij}(-1)^{(i+j)}A_{ij}}
    \end{aligned}
\end{math}

又称为拉普拉斯展开式Laplace expansion,或代数余子式展开式Cofactor expansion
矩阵和行列式是两个完全不一样的概念,行列式的行和列必须相等。

\subsubsection{伴随矩阵adjoint matrix}
行列式D的每个元素的代数余子式$A_{ij}$所构成的矩阵,称为矩阵D的伴随矩阵。

\begin{math}
    \begin{aligned}
A^{*}= \begin{pmatrix} A_{11}&A_{21}&A_{...}&A_{n1} \\
A_{12}&A_{22}&A_{...}&A_{n2} \\
A_{...}&A_{...}&A_{...}&A_{...} \\
A_{1n}&A_{2n}&A_{...}&A_{nn} \end{pmatrix} = (A_{ij})^T
\end{aligned}
\end{math}

伴随矩阵就是线性变换的逆操作,再进行一个缩放的结果,缩放的大小与矩阵的行列式的值有关。
伴随矩阵定理:
A是n阶方阵,$A^*$是A的伴随矩阵,则有$AA^*=A^*A=AE$。

\subsubsection{逆矩阵inverse matrix}
n阶方阵A,存在n阶方阵B,满足$AB=BA=E$,称方阵A可逆,称方阵B是A的逆矩阵,记为$B=A^{-1}$

由伴随矩阵定理与逆矩阵可得

\begin{math}
    \begin{aligned}
\begin{cases} A^*A=E\\ AA*=|A|E \end{cases} \Rightarrow A^{-1}=\frac{A^*}{|A|}
\end{aligned}
\end{math}


\subsubsection{正交矩阵}
两个向量的内积为零,就说这两个向量是正交的,在三维空间中,正交的两个向量相互垂直。如果相互正交
向量的长度均为1,又叫做标准正交基。
在矩阵论中,\textbf{实数}正交矩阵是方块矩阵Q,它的转置矩阵是它的逆矩阵,$Q^TQ=QQ^T=I$。
注意实数二字,正交矩阵中的元素都是实数,包含复数并且同样满足正交性质的矩阵是酉矩阵,也就是推广
到复数域之后的“正交矩阵”。

\subsubsection{相似矩阵}
两个n阶方阵矩阵A与B为相似矩阵当且仅当存在一个n阶方阵的可逆矩阵P,使得$P^{-1}AP=B$或$AP=PB$,
P被称为矩阵A与B之间的相似变换矩阵。
白话就是,矩阵是线性空间中的线性变换的一个描述,在一个线性空间中,只要我们选定一组基,那么对于
任何一个线性变换,都能够用一个确定的矩阵来加以描述。同样的,对于一个线性变换,只要逆选定一组基,
那么就可以找到一个矩阵来描述这个线性变换。换一组基,就得到一个不同的矩阵。所有这些矩阵都是这
同一个线性变换的描述,但又都不是线性变换本身。所有这些同一个线性变换的描述的矩阵互为\textbf{相似矩阵}。

\subsubsection{过渡矩阵}
transition matrix过渡矩阵这个词在数学语境中有很多地方用得,在线性代数中,它用来表示坐标矩阵的变换。
\newline
设V为n维向量空间,有两个基$S=\{v_1,...,v_n\}$和$T=\{w_1,...,w_n\}$,
过渡矩阵$P_{S \leftarrow T}$从T到S的nxn矩阵,它的列项在基S中的形式为:

\begin{math}
    P_{S \rightarrow T} = [[w_1]_S [w_2]_S ... [w_n]_S]
\end{math}

举例说明:设$S=\{ e_1,e_2,e_3\}$为标准基,$T=\{ w_1=\begin{bmatrix} 1\\ 1\\ 0 \end{bmatrix}\quad, w_2=\begin{bmatrix}0\\1\\2\end{bmatrix}\quad 
, w_3=\begin{bmatrix} 1\\1\\1 \end{bmatrix}\quad \}$。

则有\newline
$P_{S \leftarrow T}=\begin{bmatrix} 1&0&1\\1&1&1\\0&2&1\end{bmatrix}$。

\newline   根据这个关系,可以表示$w_3=2e_2+e_3$。

具体数值的例子:设$S=\{ v_1=\begin{bmatrix}1\\2\\3\end{bmatrix}\quad, v_2=\begin{bmatrix}-2\\1\\0\end{bmatrix}\quad 
, v_3=\begin{bmatrix} 1\\0\\1 \end{bmatrix}\quad \}$, $T=\{ w_1=\begin{bmatrix} 1\\1\\0\end{bmatrix}\quad, w_2=\begin{bmatrix}0\\1\\2\end{bmatrix}\quad 
, w_3=\begin{bmatrix} 1\\1\\1 \end{bmatrix}\quad \}$。

\newline  求得在基S中的$w_1$的坐标值$x_1,x_2,x_3$。
\newline
由线性关系可得

\begin{math}
    x_1v_1+x_2v_2+x_3v_3=w_1
\end{math}

它的增广矩阵为
\begin{math}
    [ { \begin{array}{c:c:c:c} 
        \begin{matrix} v_1&v_2&v_3 \end{matrix} &
        \begin{matrix} w_1 \end{matrix} &
        \begin{matrix} w_2 \end{matrix} &
        \begin{matrix} w_3 \end{matrix} 
    \end{array}} ] = \Bigg [ {\begin{array}{c:c:c:c} 
        \begin{matrix} 1&-2&1\\2&1&0\\3&0&1 \end{matrix} &
        \begin{matrix} 1\\1\\0 \end{matrix} &
        \begin{matrix} 0\\1\\2 \end{matrix} &
        \begin{matrix} 1\\1\\1 \end{matrix} 
    \end{array}}
    \Bigg ]
\end{math}

对增广矩阵化简得到
\begin{math}
    \Bigg [ {\begin{array}{c:c:c:c} 
    \begin{matrix} 1&0&0\\0&1&0\\0&0&1 \end{matrix} &
    \begin{matrix} 1.5\\-2\\-4.5 \end{matrix} &
    \begin{matrix} 0\\1\\2 \end{matrix} &
    \begin{matrix} 1\\-1\\-2 \end{matrix} 
\end{array}}
\Bigg ]
\end{math}

得到的最后三列就是$[w_1]_S, [w_2]_S,[w_3]_S$, 即过渡矩阵为
\begin{math}
    P_{S \leftarrow T} = \begin{bmatrix}
        1.5 & 0 & 1 \\
        -2 & 1 & -1 \\
        -4.5 & 2 & 2
    \end{bmatrix}
\end{math}

过渡矩阵应用在:骨骼算法中,
