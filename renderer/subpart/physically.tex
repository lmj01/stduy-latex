\chapter{Physical Terminology}

\section{anisotropy}
各向异性或非均向性,指物体的全部或部分属性性质随方向的不同有不同的变化

在计算机图形学中,各向异性表面在围绕几何法线旋转时外观会发生变化。

任何具有细粒度的东西,会产生影响外观的作用。

\subsection{anisotropy reflect}
基于反射表面上存在的小凹槽或凸起,改变光源照射物体的方向,反射在各个方向上具有不同的特性。

可用来平滑图像,克服高斯模糊的缺陷。

\subsection{isotropy reflect}
表面反射均匀具有模糊效果, 反射总是在与颗粒相反的方向拉伸。

\section{Light}

\subsection{Conception}

\subsubsection{Absorption \& Scattering}

吸收与散射, 当光在一个不均匀介质中或半透明的材质中传播时,会被吸收或散射。

\subsection{Refraction}

\subsubsection{IOR}
index of refraction,折射率,光在真空中的传播速度与光在介质中的传播速度之比。
折射率与介质的电磁性有关,

\paragraph{绝对折射率}

光在某种介质中的速度为v,真空中的光速为c,则介质的绝对折射率为$n=\frac{c}{v}$

\paragraph{相对折射率}

光从介质1射入介质2发生折射时,入射角$\theta_{1}$与折射角$\theta_{2}$的正弦之比$n_{21}$
叫做介质2相对介质1的折射率。
\begin{align*}
    n_{1} \cdot sin(\theta_{1}) &= n_{2} \cdot sin(\theta_{2}) \\
    n_{21} &= \frac{n_{2}}{n_{1}}
\end{align*}

\paragraph{全反射}

光由相对光密介质射入相对光疏介质,且入射角大于等于临界角C。临界角是指入射角满足折射角为$90^{\circ}$。
\begin{align*}
    \frac{n_{1}}{n_{2}} = \frac{sin90^{\circ}}{sinC}
\end{align*}
空气的折射率是$n_{2}=1$, 则介质向空气入射可简化为$n=1/sinC$.

\paragraph{光色散}

对不同的波长,介质的折射率$n(\lambda)$不同,这叫做光色散.

\subsection{Reflectance}

反射率,是用来衡量物质反射能力的量,定义为物体表面反射能量与达到物体表面入射能量的比率。反射率是波长的函数,
指某段波长向一定方向的反射,不同波长就有不同的反射率。
\begin{align*}
    \phi(\lambda) = \frac{E_{R}(\lambda)}{E_{I}(\lambda)} = \frac{\pi L(\lambda)}{E_{I}(\lambda)}
\end{align*}

\subsection{Albedo}

反照率,指地表在太阳辐射的影响下,反射辐射通量与入射辐射通量的比值。albedo是对全波段的,是反射率在所有方向上的积分。

\begin{align*}
    \alpha = \frac{M}{E}
\end{align*}

\subsubsection{White Sky Albedo}

WSA,白空反照率,指忽略太阳直射辐射,只考虑地物对大气散射辐射的反射情况。

\subsubsection{Black Sky Albedo}

BSA,黑空反照率。指忽略大气散射,只考虑地物对太阳直射,入射,辐射的反射情况

\subsubsection{Blue Sky Albedo}

地物真实反照率,又称为蓝空反照率,或半球-半球反射率BHR

\begin{align*}
    BHR \approx (1-s)WSA + sBSA 
\end{align*}

其中s是天空散射光比例。


\chapter{Physical Simulation}

\section{Material Point Method}

物质点法,是一种模拟连续介质的方法,最早被Sulkey等人在1995年发明。
与有限元相比,可以把MPM里面的grid nodes对应到FEM中的DOFs,
把MPM里面的particles对应到FEM中的quadrature points。和FEM里使用显式网格策略
不同,作为一种Element-Free Galerkin(EFG)方法,MPM里面并没有显式的Elements
和Lagrangian grid,只有能够随意移动的粒子,作为quadrature points。这样的特性
非常适合处理大形变,而其背景网格带来的自动碰撞处理,多材料耦合等。由于离散化
出自weak form,使得MPM的physical accuracy有了保证。

\subsection{Moving Least Squares Material Point Method}

移动最小二乘物质点法,

\chapter{Physically Based Rendering}

Physically Based Rendering Toolkit是一个基于物理渲染的开源离线渲染器, 可参考配套的书\cite{PBR3ed}。



\paragraph{Markov Process}

Markov Chain可应用到渲染上是基于Detailed Balance的方法。

光线传递的随机性(光子抵达物体表面材质后被吸收或随机反射)仅由当前状态决定,和历史无关,满足马可夫过程定义。

在渲染中使用Markov Chain Monte Carlo没有复杂的物理,现代所有光线传播的模拟都是基于对渲染方程rendering equation
的求解,而渲染方程已经将原本的物理系统大大的简化了,所以光线传递的物理到底是不是平稳马尔可夫过程和渲染没有任何关系,
渲染只是要计算一个定义域纬度很高的函数而已,对于一个场景,固定的一组输入是会返回一个固定的值,没有什么随机和模糊的东西,
最极端简化就是一个一维函数,Metropolis Sampling的牛逼之处在于它不需要求解函数的积分或者算它的概率分布就能给你提供一系列
和目标高数概率分布一致的样本,这一点对基于Monte Carlo的渲染来说十分重要,叫做重要性采样Importance Sampling。

所以Metropolis Sampling主要用来渲染一些采样路径特别复杂的场景,一般室外大太阳的场景用不着付出额外的计算。Metropolis Light Transport
的论文就是用了一个光源在一个微微打开的门缝后面的场景当例子。
