\clearpage
\part{Physically}

物理学在图形学的意义是非常重要的

\chapter{Physical Terminology}

\section{anisotropy}
各向异性或非均向性,指物体的全部或部分属性性质随方向的不同有不同的变化

在计算机图形学中,各向异性表面在围绕几何法线旋转时外观会发生变化。

任何具有细粒度的东西,会产生影响外观的作用。

\subsection{anisotropy reflect}
基于反射表面上存在的小凹槽或凸起,改变光源照射物体的方向,反射在各个方向上具有不同的特性。

可用来平滑图像,克服高斯模糊的缺陷。

\subsection{isotropy reflect}
表面反射均匀具有模糊效果, 反射总是在与颗粒相反的方向拉伸。

\section{Reflectance}

反射率,是用来衡量物质反射能力的量,定义为物体表面反射能量与达到物体表面入射能量的比率。反射率是波长的函数,
指某段波长向一定方向的反射,不同波长就有不同的反射率。
\begin{align*}
    \phi(\lambda) = \frac{E_{R}(\lambda)}{E_{I}(\lambda)} = \frac{\pi L(\lambda)}{E_{I}(\lambda)}
\end{align*}

\section{Albedo}

反照率,指地表在太阳辐射的影响下,反射辐射通量与入射辐射通量的比值。albedo是对全波段的,是反射率在所有方向上的积分。

\begin{align*}
    \alpha = \frac{M}{E}
\end{align*}

\subsection{White Sky Albedo}

WSA,白空反照率,指忽略太阳直射辐射,只考虑地物对大气散射辐射的反射情况。

\subsection{Black Sky Albedo}

BSA,黑空反照率。指忽略大气散射,只考虑地物对太阳直射,入射,辐射的反射情况

\subsection{Blue Sky Albedo}

地物真实反照率,又称为蓝空反照率,或半球-半球反射率BHR

\begin{align*}
    BHR \approx (1-s)WSA + sBSA 
\end{align*}

其中s是天空散射光比例。


\chapter{Physical Simulation}

\section{Material Point Method}

物质点法,是一种模拟连续介质的方法,最早被Sulkey等人在1995年发明。
与有限元相比,可以把MPM里面的grid nodes对应到FEM中的DOFs,
把MPM里面的particles对应到FEM中的quadrature points。和FEM里使用显式网格策略
不同,作为一种Element-Free Galerkin(EFG)方法,MPM里面并没有显式的Elements
和Lagrangian grid,只有能够随意移动的粒子,作为quadrature points。这样的特性
非常适合处理大形变,而其背景网格带来的自动碰撞处理,多材料耦合等。由于离散化
出自weak form,使得MPM的physical accuracy有了保证。

\subsection{Moving Least Squares Material Point Method}

移动最小二乘物质点法,

