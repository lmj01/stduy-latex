\section{补充}
%多行公式--带编号
\begin{gather}
    a + b +c = b + a \\
    1+2 = 2 + 1
\end{gather}
%多行公式--不带编号1
\begin{gather*}
    a + b = b + a \\
    1+2 = 2 + 1
\end{gather*}

%多行公式--带编号2 \notag 阻止编号
\begin{gather}
a + b = b + a \notag \\
1+2 = 2 + 1 \notag
\end{gather}

% 按&号对齐,--带编号
\begin{align}
    a+b &= b+a \\
    1+2= & 2+1
\end{align}

% 按&号对齐,--不带编号
\begin{align*}
a+b &= b+a \\
1+2 &=2+1
\end{align*}

%一个公式的多行排版--带编号
\begin{equation}
    \begin{split}
    \cos 2x &= \cos^2 x - \sin^2x \\
    &=2\cos^2x-1
    \end{split} 
\end{equation}

%一个公式的多行排版--不带编号
\begin{equation*}
\begin{split}
\cos 2x &= \cos^2 x - \sin^2x \\
&=2\cos^2x-1
\end{split} 
\end{equation*}

%case环境, text{}在数学模式中处理中文-带编号
\begin{equation}
    D(x)=\begin{cases}
    1, & \text{如果} x \in \mathbb{Q};\\
    0, & \text{如果} x \in \mathbb{R}\setminus\mathbb{Q}
    \end{cases}
\end{equation}

%case环境, text{}在数学模式中处理中文-不带编号
\begin{equation*}
D(x)=\begin{cases}
1, & \text{如果} x \in \mathbb{Q};\\
0, & \text{如果} x \in \mathbb{R}\setminus\mathbb{Q}
\end{cases}
\end{equation*}

\subsection{矩阵的省略号}
%\dots 横向省略号
%\vdots 竖向省略号
%\ddots 斜向省略号
\[
A = \begin{bmatrix}
a_{11} & \dots & a_{1n}\\
\vdots& \ddots & \vdots \\
0 & \dots & a_{nn}
\end{bmatrix}_{n \times n}
\]

\subsection{行内小矩阵}
复数可用矩阵
\begin{math}
    \left(
    \begin{smallmatrix}
    x & y \\ -y & x
    \end{smallmatrix}
    \right)
\end{math}
来表示

\subsection{array环境}
\[
\begin{array}{c|c}
1 & 2\\
\hline
0 & 1
\end{array}
\]
