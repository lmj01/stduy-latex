%%%%%%%%%%%%%%%%%%%%%%%%%%%%%%%%%%%%%%%%%%%%%%%%%%%%%%%%%%%%%%%%%
%%% Copyright Stefan M. Moser
%%% 
%%% Version 1.0, 17 September 2004
%%% Version 1.1, 28 October 2004
%%% Version 2.0, 4 June 2006: major revision
%%% Version 2.1, 6 November 2006: corrected small typos
%%% Version 2.2, 19 January 2007: revised paragraph about multline
%%% Version 2.3, 5 April 2007: added note that no additional space is
%%%              needed in front of operators etc.
%%% Version 3.0, 13 October 2008: completely rewritten, added nicer
%%%              layout, added hyperreferences new section about 
%%%              advanced stuff
%%% Version 3.1, 23 October 2008: corrected small typos, added comment
%%%              about needed amsmath.sty, added comment about
%%%              different matrix-environments
%%% Version 3.2, 29 September 2009: added comment about IEEEtran.cls
%%%              and its end-of-proof sign; added strutmode
%%% Version 3.3, 29 March 2011: example of problem with multline, 
%%%              new section 5, unary/binary issue, qed problem with 
%%%              equation, comment about hyperref
%%% Version 3.4, 28 May 2011: many typo corrections; new section 4.3: 
%%%              recommendation of using only IEEEeqnarray; new
%%%              section 6.8 about double-column equations; new
%%%              section 5.5 about page-breaks
%%% Version 3.5, 29 October 2011: improved on the fancy frames; added
%%%              section 6.4; worked over Emacs section and settings;
%%%              fixed some typos
%%% Version 3.6, 3 November 2011: fixed mixup of vertical/horizontal
%%%              alignment 
%%% Version 3.7, 9 February 2012: added section about brackets with
%%%              numbering (new Sec. 6.3), added numcases to Sec. 6.2
%%% Version 4.0, 3 January 2013: Revision based on IEEEtrantools
%%%              v1.3 (or IEEEtran.cls v1.8): removed section about
%%%              hyperlink, added section about IEEEproof, revised
%%%              section about equation-numbering; added \middle
%%% Version 4.1, 7 February 2013: New section 8 about math symbols;
%%%              some minor typos corrected
%%% Version 4.2, 6 August 2013: changed \: to \>; example in Sec. 4.3;
%%%              \\* for numbers on next line, or QED; typos removed.
%%% Version 4.3, 16 November 2015: adapted to current version 1.5 of
%%%              IEEEtrantools (IEEEtran 1.8b); changed away from
%%%              pstricks to tikz so that compilation works for both
%%%              latex and pdflatex; removed Case 2 in Section 3
%%% Version 4.4, 13 March 2016: pointing out a difference between
%%%              equation and IEEEeqnarray with respect to the
%%%              equation number; new Section 4.4; use IEEEeqnarray
%%%              page-layout commands 
%%% Version 4.5, 21 June 2016: adding fix to italic equation number
%%%              also for subnumbers; rewritten and fixed section
%%%              about double-column equations in a two-column
%%%              document 
%%% Version 4.6, 29 September 2017: small formulation corrections;
%%%              fixed and expanded subnumbering (incl. hyperlinks);
%%%              new paragraph 6.1 about the noalign-command; fixed
%%%              the treatment of left/right; added index
%%%
%%% You may change this code, but you always have to keep this 
%%% header (you may add to it).
%%%%%%%%%%%%%%%%%%%%%%%%%%%%%%%%%%%%%%%%%%%%%%%%%%%%%%%%%%%%%%%%%

\documentclass[a4paper,11pt]{article}

\newcommand{\version}{Version 4.6}

\usepackage{verbatim}
\usepackage{calc}
\usepackage{amsmath,amssymb,mathrsfs}
\usepackage{centernot}
\usepackage{amsthm}
\newtheorem{theorem}{Theorem}
\usepackage{IEEEtrantools}
\renewcommand{\IEEEproofindentspace}{2em}
\IEEEsettextheight{3.2cm}{4cm}
\IEEEsettopmargin{t}{3.2cm}
\IEEEquantizetextheight{c}
\IEEEsettextwidth{3.25cm}{3.25cm}
\IEEEsetsidemargin{c}{0cm}
\IEEEsetfootermargin{t}{1.4cm}

\usepackage{fancyhdr}
\pagestyle{fancy}
\lhead{\textsc{How to Typeset Equations in \LaTeX{}}}
\chead{}
\rhead{\thepage}
\lfoot{\small\copyright~Stefan M.~Moser}
\rfoot{\small\today, \version}
\cfoot{}
\setlength{\headwidth}{\textwidth}
\setlength{\headheight}   {14pt}
\renewcommand{\headrulewidth}{0pt}

\usepackage{graphicx}
\usepackage{cases} %For numcases-environment
\usepackage[newzealand]{babel}

\usepackage{mleftright}
\usepackage{needspace}

\usepackage{imakeidx}
\indexsetup{noclearpage=true}
\makeindex[options=-c,intoc]
\renewcommand{\indexspace}{\vspace{7mm}}
%\newcommand{\cindex}[1]{\index{#1@{\tt #1}}}

\usepackage{tikz}
\usetikzlibrary{shadows} 
\usepackage[framemethod=tikz]{mdframed} %should be loaded after amsthm.sty
\global\mdfdefinestyle{myboxstyle}{%
  shadow=true,
  linecolor=black,
  shadowcolor=black,
  shadowsize=6pt,
  nobreak=false,
  innertopmargin=10pt,
  innerbottommargin=10pt,
  leftmargin=5pt,
  rightmargin=5pt,
  needspace=1cm,
  skipabove=10pt,
  skipbelow=15pt,
  middlelinewidth=1pt,
  afterlastframe={\vspace{5pt}},
  aftersingleframe={\vspace{5pt}},
  tikzsetting={%
    draw=black,
    very thick}
}
\newmdenv[style=myboxstyle]{whitebox} % framed box that can be broken at end of page
\newmdenv[style=myboxstyle,backgroundcolor=black!20]{graybox} %gray version of framed box

\usepackage[colorlinks=true,linkcolor=blue]{hyperref}
%save pageref to label
\let\lsspageref\pageref

\newlength{\eqboxstorage}
\newcommand{\eqbox}[1]{
  \setlength{\eqboxstorage}{\fboxsep}
  \setlength{\fboxsep}{6pt}
  \boxed{#1}
  \setlength{\fboxsep}{\eqboxstorage}
}

\newcommand{\sizecorr}[1]{\makebox[0cm]{\phantom{$\displaystyle #1$}}} 
\newcommand{\indep}{\mathrel{\bot}\joinrel\mathrel{\mkern-5mu}\joinrel\mathrel{\bot}}
\newcommand{\dep}{\centernot\indep}
\newcommand{\markov}{\mathrel{\multimap}\joinrel\mathrel{-}\joinrel\mathrel{\mkern-6mu}\joinrel\mathrel{-}} 
\newcommand{\dd}{\mathop{}\!\mathrm{d}}

\makeatletter
\def\IEEElabelanchoreqn#1{\bgroup
\def\@currentlabel{\p@equation\theequation}\relax
\def\@currentHref{\@IEEEtheHrefequation}\label{#1}\relax
\Hy@raisedlink{\hyper@anchorstart{\@currentHref}}\relax
\Hy@raisedlink{\hyper@anchorend}\egroup}
\makeatother
\newcommand{\subnumberinglabel}[1]{\IEEEyesnumber
  \IEEEyessubnumber*\IEEElabelanchoreqn{#1}}

%%%%%%%%%%%%%%%%%%%%%%%%%%%%%%%%%%%%%%%%%%%%%%%%%%%%%%%%%%%%%%%%%
%
% --- latexcode and latexcodeonly environments ----
%
% The idea and code is stolen from Tobias Oetiker's "Not So Short
% Introduction to LaTeX". I adapted the code by replacing \textwidth
% by \linewidth. This version is  hyperref-ready.
% 
% This is an environment to show LaTeX code examples both as its
% actual code and the compiled version. On the left side the source
% code is given and on the right side the compiled result is shown.
% For latexcodeonly only the source code is shown, but it is not
% compiled.
%
% Example:
% \begin{latexcode}
% \Large This is Large
% \end{latexcode}
% \begin{latexcodeonly}
% \Large This is Large
% \end{latexcodeonly}
%
% --- Definitions for latexcode ----
\makeatletter
\newwrite\latexcode@out
\newcounter{exacnt}
\setcounter{exacnt}{1}
\newlength{\savefboxrule}
\newlength{\savefboxsep}
\newlength{\outdent}
\setlength{\outdent}{0cm} %used to be 1cm
\addtolength{\headwidth}{\outdent}
\newenvironment{latexcode}%
{\begingroup% Lets Keep the Changes Local
  \@bsphack
  \immediate\openout \latexcode@out \jobname.exa
  \let\do\@makeother\dospecials\catcode`\^^M\active
  \def\verbatim@processline{%
    \immediate\write\latexcode@out{\the\verbatim@line}}%
  \verbatim@start}%
{\immediate\closeout\latexcode@out\@esphack\endgroup%
                                %
                                % And here comes my part. :-
                                %   
  \stepcounter{exacnt}%
  \setlength{\parindent}{0pt}%
  \par\addvspace{3.0ex plus 0.8ex minus 0.5ex}\vskip -\parskip
%  Page \lsspageref{exa:\theexacnt}
\expandafter\ifx\csname r@exa\theexacnt\endcsname\relax\else
%\ifx\pdfoutput\undefined % We're not running pdftex
%  \ifodd\lsspageref{exa\theexacnt}\hspace*{0pt}\else\hspace*{-\outdent}\fi%
%\else
%% HyPsd@pageref internal hyperref command v6.69c
  \ifodd\HyPsd@pageref{exa\theexacnt}\hspace*{0pt}\else\hspace*{-\outdent}\fi%
%\fi
\fi
\makebox[\linewidth][l]{%
%\raisebox{-\height}[0pt][\totalheight]{%
  \begin{minipage}[c]{0.5\outdent+0.46\linewidth-3mm}%
    \small\verbatiminput{\jobname.exa}
  \end{minipage}%
                                %}%
  \hspace{5mm}%
  \setlength{\savefboxrule}{\fboxrule}%
  \setlength{\fboxrule}{0.1pt}%
  \setlength{\savefboxsep}{\fboxsep}%
  \setlength{\fboxsep}{3mm}%
  % \raisebox{-\height}[0pt][\totalheight]{%
  \fbox{%
    \begin{minipage}{0.5\outdent+0.54\linewidth-3.5mm-2\fboxrule-2\fboxsep}%
      \setlength{\fboxrule}{\savefboxrule}%
      \setlength{\fboxsep}{\savefboxsep}%
      \setlength{\fboxrule}{0.5pt}%
      \setlength{\parskip}{1ex plus 0.4ex minus 0.2ex}%
      \begin{trivlist}\item\small\input{\jobname.exa}
      \end{trivlist}
    \end{minipage}
    }%
%  }%
}\label{exa\theexacnt}%
\par\addvspace{3ex plus 0.8ex minus 0.5ex}\vskip -\parskip
}
\makeatother
% ----- end latexcode -----

% --- Definitions for latexcodeonly ----
\makeatletter
\newwrite\latexcode@out
\setcounter{exacnt}{1}
\setlength{\outdent}{0cm} %used to be 1cm
\addtolength{\headwidth}{\outdent}
\newenvironment{latexcodeonly}%
{\begingroup% Lets Keep the Changes Local
  \@bsphack
  \immediate\openout \latexcode@out \jobname.exa
  \let\do\@makeother\dospecials\catcode`\^^M\active
  \def\verbatim@processline{%
    \immediate\write\latexcode@out{\the\verbatim@line}}%
  \verbatim@start}%
{\immediate\closeout\latexcode@out\@esphack\endgroup%
                                %
                                % And here comes my part. :-
                                %   
  \stepcounter{exacnt}%
  \setlength{\parindent}{0pt}%
  \par\addvspace{3.0ex plus 0.8ex minus 0.5ex}\vskip -\parskip
%  Page \lsspageref{exa:\theexacnt}
\expandafter\ifx\csname r@exa\theexacnt\endcsname\relax\else
%\ifx\pdfoutput\undefined % We're not running pdftex
%  \ifodd\lsspageref{exa\theexacnt}\hspace*{0pt}\else\hspace*{-\outdent}\fi%
%\else
%% HyPsd@pageref internal hyperref command v6.69c
  \ifodd\HyPsd@pageref{exa\theexacnt}\hspace*{0pt}\else\hspace*{-\outdent}\fi%
%\fi
\fi
\makebox[\linewidth][l]{%
%\raisebox{-\height}[0pt][\totalheight]{%
  \begin{minipage}[c]{0.9\outdent+0.46\linewidth-3mm}%
    \small\verbatiminput{\jobname.exa}
  \end{minipage}%
                                %}%
  \hspace{5mm}%
  \setlength{\savefboxrule}{\fboxrule}%
  \setlength{\fboxrule}{0.1pt}%
  \setlength{\savefboxsep}{\fboxsep}%
  \setlength{\fboxsep}{3mm}%
}\label{exa\theexacnt}%
\par\addvspace{3ex plus 0.8ex minus 0.5ex}\vskip -\parskip
}
\makeatother
% ----- end latexcodeonly -----


%%%%%%%%%%%%%%%%%%%%%%%%%%%%%%%%%%%%%%%%%%%%%%%%%%%%%%%%%%%%%%%%%

\title{\textsc{How to Typeset Equations in \LaTeX{}}}

\author{Stefan M.~Moser}

\date{\today\\\version}

\begin{document}

\maketitle

\tableofcontents


\bigskip
\bigskip

Over the years this manual has grown to quite an extended size.  If
you have limited time, read Section~\ref{sec:IEEEeqnarray_intro} to
get the basics.  If you have little more time, read
Sections~\ref{sec:IEEEeqnarray} and \ref{sec:common-usage} to cover
the most common situations.

This manual is written with the newest version of \verb+IEEEtran+ in
mind:\footnote{You can check the version on your system using
  \texttt{kpsewhich IEEEtrantools.sty} to find the path to the used
  file and then viewing it. Any current \LaTeX{}-installation has them
  available and ready to use.} version~1.8b of \verb+IEEEtran.cls+,
and version~1.5 of \verb+IEEEtrantools.sty+.

This manual is continually being updated. Check for the most current
version at \url{http://moser-isi.ethz.ch/}

%%%%%%%%%%%%%%%%%%%%%%%%%%%%%%%%%%%%%%%%%%%%%%%%%%%%%%%%%%%%%%%%%
\section{Introduction}
\label{sec:introduction}

\LaTeX{} is a very powerful tool for typesetting in general and for
typesetting math in particular. In spite of its power, however, there
are still many ways of generating better or less good results.  This
manual offers some tricks and hints that hopefully will lead to the
former\ldots

Note that this manual does neither claim to provide the best nor the
only solution. Its aim is rather to give a couple of rules that can be
followed easily and that will lead to a good layout of all equations
in a document. It is assumed that the reader has already mastered the
basics of \LaTeX{}.

The structure of this document is as follows. We introduce the most
basic equation in Section~\ref{sec:equation};
Section~\ref{sec:multline} then explains some first possible reactions
when an equation is too long.  The most important part of the manual
is contained in Sections~\ref{sec:IEEEeqnarray} and
\ref{sec:common-usage}: there we introduce the powerful
\verb+IEEEeqnarray+-environment that should be used in any case
instead of \verb+align+ or \verb+eqnarray+.

In Section~\ref{sec:advanced-typesetting} some more advanced problems
and possible solutions are discussed, and
Section~\ref{sec:emacs-ieeeeqnarray} contains some hints and tricks
about the editor Emacs. Finally, Section~\ref{sec:some-usef-defin}
makes some suggestions about some special math symbols that cannot be
easily found in \LaTeX{}.

In the following any \LaTeX{} command will be set in
\verb+typewriter font+.  \emph{RHS} stands for \emph{right-hand side},
i.e., all terms on the right of the equality (or inequality)
sign. Similarly, \emph{LHS} stands for \emph{left-hand side}, i.e.,
all terms on the left of the equality sign. To simplify our language,
we will usually talk about \emph{equality}. Obviously, the typesetting
does not change if an expression actually is an inequality.

This documents comes together with some additional files that might be
helpful:
\begin{itemize}
\item \verb+typeset_equations.tex+: \LaTeX{} source file of this
  manual.

\item \verb+dot_emacs+: commands to be included in the preference file
  of Emacs (\verb+.emacs+) (see Section~\ref{sec:emacs-ieeeeqnarray}).
  \index{dot\_emacs}\index{.emacs}

\item \verb+IEEEtrantools.sty+ [2015/08/26 V1.5 by Michael Shell]:
  package needed for the \verb+IEEEeqnarray+-environment. 

\item \verb+IEEEtran.cls+ [2015/08/26 V1.8b by Michael Shell]:
  \LaTeX{} document class package for papers in IEEE format.

\item \verb+IEEEtran_HOWTO.pdf+ [2015/08]: official manual of the
  \verb+IEEEtran+-class. The part about \verb+IEEEeqnarray+ is found
  in Appendix~F.

\end{itemize}
Note that \verb+IEEEtran.cls+ and \verb+IEEEtrantools.sty+ is provided
automatically by any up-to-date \LaTeX{}-distribution.


%%%%%%%%%%%%%%%%%%%%%%%%%%%%%%%%%%%%%%%%%%%%%%%%%%%%%%%%%%%%%%%%%
\section{Single Equations: equation}
\label{sec:equation}

\index{equation}\index{single equation}%
The main strength of \LaTeX{} concerning typesetting of mathematics is
based on the package \verb+amsmath+. Every current distribution of
\LaTeX{} will come with this package included, so you only need to
make sure that the following line is included in the header of your
document:\index{amsmath}
\begin{latexcodeonly}
\usepackage{amsmath}
\end{latexcodeonly}
\noindent
Throughout this document it is assumed that \verb+amsmath+ is loaded.

Single equations should be exclusively typed using the
\verb+equation+-environment:
\begin{latexcode}
\begin{equation}
  a = b + c
\end{equation}
\end{latexcode}
\noindent
In case one does not want to have an equation number, the *-version is
used:
\begin{latexcode}
\begin{equation*}
  a = b + c
\end{equation*}
\end{latexcode}
\noindent
All other possibilities of typesetting simple equations have
disadvantages:
\begin{itemize}
\item \index{displaymath} The \verb+displaymath+-environment offers
  no equation-numbering. To add or to remove a ``*'' in the
  \verb+equation+-environment is much more flexible.
\item \index{\$\$} Commands like \verb+$$...$$+,
    \verb+\[...\]+, etc., have the additional disadvantage that the
    source code is extremely poorly readable. Moreover, \verb+$$...$$+
    is faulty: the vertical spacing after the equation is too large in
    certain situations.
\end{itemize}
\needspace{6\baselineskip}
We summarize:
\begin{whitebox}
  \centering 
  Unless we decide to rely exclusively on \verb+IEEEeqnarray+ (see the
  discussion in Sections~\ref{sec:remark-about-cons} and
  \ref{sec:using-ieee-all}), we should only use \verb+equation+ (and
  no other environment) to produce a single equation.
\end{whitebox}


%%%%%%%%%%%%%%%%%%%%%%%%%%%%%%%%%%%%%%%%%%%%%%%%%%%%%%%%%%%%%%%%%
\section{Single Equations that are Too Long: multline}
\label{sec:multline}

\index{single equation!too long}\index{multline}%
If an equation is too long, we have to wrap it somehow. Unfortunately,
wrapped equations are usually less easy to read than not-wrapped
ones. To improve the readability, one should follow certain rules on
how to do the wrapping:
\begin{whitebox}
  \begin{enumerate}
  \item In general one should always wrap an equation \textbf{before} an
    equality sign or an  operator.
    \index{wrapping}\index{equation!wrapping} 
  \item A wrap before an equality sign is preferable to a wrap
    before any operator.
  \item A wrap before a plus- or minus-operator is preferable to a wrap
    before a multiplication-operator.
  \item Any other type of wrap should be avoided if ever possible.
  \end{enumerate}
\end{whitebox}
The easiest way to achieve such a wrapping is the use of the
\verb+multline+-en\-vi\-ron\-ment:\footnote{As a reminder: it is
  necessary to include the \texttt{amsmath}-package for this command
  to work!}
\begin{latexcode}
\begin{multline}
  a + b + c + d + e + f 
  + g + h + i  
  \\
  = j + k + l + m + n 
\end{multline}
\end{latexcode}
\noindent
The difference to the \verb+equation+-environment is that an arbitrary
line-break (or also multiple line-breaks) can be introduced. This is
done by putting a \verb+\\+ at those places where the equation needs
to be wrapped.

Similarly to \verb+equation*+ there also exists a
\verb+multline*+-version for preventing an equation number.

However, in spite of its ease of use, often the
\verb+IEEEeqnarray+-environment (see Section~\ref{sec:IEEEeqnarray})
will yield better results.  Particularly, consider the following
common situation:
\begin{latexcode}
\begin{equation}
  a = b + c + d + e + f 
  + g + h + i + j 
  + k + l + m + n + o + p 
  \label{eq:equation_too_long}
\end{equation}
\end{latexcode}
\noindent
Here the RHS is too long to fit on one line. The
\verb+multline+-environment will now yield the following:
\begin{latexcode}
\begin{multline}
  a = b + c + d + e + f 
  + g + h + i + j \\
  + k + l + m + n + o + p
\end{multline}
\end{latexcode}
\noindent
This is of course much better than \eqref{eq:equation_too_long}, but
it has the disadvantage that the equality sign loses its natural
stronger importance over the plus operator in front of $k$. A better
solution is provided by the \verb+IEEEeqnarray+-environment that will
be discussed in detail in Sections~\ref{sec:IEEEeqnarray} and
\ref{sec:common-usage}:
\begin{latexcode}
\begin{IEEEeqnarray}{rCl}
  a & = & b + c + d + e + f 
  + g + h + i + j \nonumber\\
  && +\> k + l + m + n + o + p  
  \label{eq:dont_use_multline}
\end{IEEEeqnarray}
\end{latexcode}
\noindent
In this case the second line is horizontally aligned to the first line:
the $+$ in front of $k$ is exactly below $b$, i.e., the RHS is clearly
visible as contrast to the LHS of the equation.

Also note that \verb+multline+ wrongly forces a minimum spacing on the
left of the first line even if it has not enough space on the right,
causing a noncentered equation. This can even lead to the very ugly
typesetting where the second line containing the RHS of an equality is
actually \emph{to the left} of the first line containing the LHS:
\begin{latexcode}
\begin{multline}
  a + b + c + d + e + f + g 
  + h + i + j  \\
  = k + l + m + n + o + p + q
  + r + s + t + u 
\end{multline}
\end{latexcode}
\noindent
Again this looks much better using \verb+IEEEeqnarray+:
\begin{latexcode}
\begin{IEEEeqnarray}{rCl}
  \IEEEeqnarraymulticol{3}{l}{%
    a + b + c + d + e + f + g 
    + h + i + j  
  }\nonumber\\*%
  & = & k + l + m + n + o + p + q
  + r + s + t + u \nonumber\\*
\end{IEEEeqnarray}
\end{latexcode}
\noindent
For more details see Section~\ref{sec:ieeeeqnarraymulticol}.

\needspace{6\baselineskip}
For these reasons we give the following rule:\index{multline}
\begin{whitebox}
  \centering
  The \verb+multline+-environment should exclusively be used in the
  four specific situations described in
  Sections~\ref{sec:case1}--\ref{sec:case5} below.
\end{whitebox}


\subsection{Case 1: The expression is not an equation}
\label{sec:case1}

If the expression is not an equation, i.e., there is no
equality sign, then there exists no RHS or LHS and \verb+multline+
offers a nice solution:
\begin{latexcode}
\begin{multline}
  a + b + c + d + e + f \\
  + g + h + i + j + k + l \\
  + m + n + o + p + q 
\end{multline}
\end{latexcode}


\subsection{Case 2: Additional comment}
\label{sec:case3}

If there is an additional comment at the end of the equation that does
not fit on the same line, then this comment can be put onto the next
line:
\begin{latexcode}
\begin{multline}
  a + b + c + d 
  = e + f + g + h, \quad \\
  \text{for } 0 \le n 
  \le n_{\textnormal{max}}
\end{multline}
\end{latexcode}
  

\subsection{Case 3: LHS too long --- RHS too short}
\label{sec:case4}

\index{RHS!too short, LHS too long}%
\index{LHS!too long, RHS too short}%
If the LHS of a single equation is too long and the RHS is very short,
then one cannot break the equation in front of the equality sign as
wished, but one is forced to do it somewhere on the LHS. In this case
one cannot nicely keep the natural separation of LHS and RHS anyway
and \verb+multline+ offers a good solution:
\begin{latexcode}
\begin{multline}
  a + b + c + d + e + f 
  + g \\+ h + i + j 
  + k + l = m
\end{multline}
\end{latexcode}
 

\subsection{Case 4: A term on the RHS should not be split}
\label{sec:case5}

The following is a special (and rather rare) case: the LHS would be
short enough and/or the RHS long enough in order to wrap the equation
in a way as shown in \eqref{eq:dont_use_multline}, i.e., this usually
would call for the \verb+IEEEeqnarray+-environment.  However, a term
on the RHS is an entity that we rather would not split, but it is too
long to fit:\footnote{For a definition of \texttt{$\backslash$dd}, see
  Section~\ref{sec:some-usef-defin}.}
\begin{latexcode}
\begin{multline}
  h^{-}(X|Y) \le \frac{n+1}{e} 
  - h(X|Y)  
  \\
  + \int p(y) \log \left(
    \frac{\mathsf{E}\bigl[|X|^2
      \big| Y=y\bigr]}{n}
  \right) \dd y
\end{multline}
\end{latexcode}
\noindent
In this example the integral on the RHS is too long, but should not be
split for readability.

Note that even in this case it might be possible to find different
solutions based on \verb+IEEEeqnarray+-environment:
\begin{latexcode}
\begin{IEEEeqnarray}{rCl}
  \IEEEeqnarraymulticol{3}{l}{
    h^{-}(X|Y) 
  }\nonumber\\\quad
  & \le & \frac{n+1}{e} 
  - h(X|Y) \nonumber\\
  && + \int p(y) \log \left(
    \frac{\mathsf{E}\bigl[|X|^2
      \big| Y=y\bigr]}{n}
  \right) \dd y
  \nonumber\\*
\end{IEEEeqnarray}
\end{latexcode}



%%%%%%%%%%%%%%%%%%%%%%%%%%%%%%%%%%%%%%%%%%%%%%%%%%%%%%%%%%%%%%%%%
\section{Multiple Equations: IEEEeqnarray}
\label{sec:IEEEeqnarray}

In the most general situation, we have a sequence of several
equalities that do not fit onto one line. Here we need to work with
horizontal alignment in order to keep the array of equations in a nice
and readable structure.

Before we offer our suggestions on how to do this, we start with a few
\emph{bad} examples that show the biggest drawbacks of common
solutions.


\subsection{Problems with traditional commands}
\label{sec:problems_traditional}

To group multiple equations, the \verb+align+-environment\footnote{The
  \texttt{align}-environment can also be used to group several blocks
  of equations beside each other.  However, also for this situation,
  we recommend to use the \texttt{IEEEeqnarray}-environment with an
  argument like, e.g., \texttt{\{rCl+rCl\}}.} could be used:\index{align}
\begin{latexcode}
\begin{align}
  a & = b + c \\
  & = d + e
\end{align}
\end{latexcode}
\noindent
While this looks neat as long as every equation fits onto one line,
this approach does not work anymore once a single line is too long:
\begin{latexcode}
\begin{align}
  a & = b + c \\
  & = d + e + f + g + h + i 
  + j + k + l \nonumber\\
  & + m + n + o \\
  & = p + q + r + s
\end{align}
\end{latexcode}
\noindent
Here \mbox{$+\> m$} should be below $d$ and not below the equality
sign. Of course, one could add some space by, e.g.,
\verb+\hspace{...}+, but this will never yield a precise arrangement
(and is bad programming style!).

A better solution is offered by the \verb+eqnarray+-environment:
\begin{latexcode}
\begin{eqnarray}
  a & = & b + c \\
  & = & d + e + f + g + h + i 
  + j + k + l \nonumber\\
  && +\> m + n + o \\
  & = & p + q + r + s
\end{eqnarray}
\end{latexcode}
\noindent
The \verb+eqnarray+-environment,\footnote{Actually, \texttt{eqnarray}
  is not an \texttt{amsmath}-command, but stems from the dawn of
  \LaTeX{}.} however, has a few very severe disadvantages:
\begin{itemize}
\item The spaces around the equality signs are too big.
  Particularly, they are \textbf{not} the same as in the
  \verb+multline+- and \verb+equation+-environments:
\begin{latexcode}
\begin{eqnarray}
  a & = & a = a
\end{eqnarray}
\end{latexcode}

\item The expression sometimes overlaps with the equation number even
  though there would be enough room on the left:
\begin{latexcode}
\begin{eqnarray}
  a & = & b + c 
  \\
  & = & d + e + f + g + h^2 
  + i^2 + j 
  \label{eq:faultyeqnarray}
\end{eqnarray}
\end{latexcode}

\item The \verb+eqnarray+-environment offers a command
  \verb+\lefteqn{...}+ that can be used when the LHS is too long:
\begin{latexcode}
\begin{eqnarray}
  \lefteqn{a + b + c + d 
    + e + f + g + h}\nonumber\\
  & = & i + j + k + l + m 
  \\
  & = & n + o + p + q + r + s
\end{eqnarray}
\end{latexcode}
\noindent
Unfortunately, this command is faulty: if the RHS is too short, the
array is not properly centered:\index{lefteqn}
\begin{latexcode}
\begin{eqnarray}
  \lefteqn{a + b + c + d 
    + e + f + g + h} 
  \nonumber\\
  & = & i + j 
\end{eqnarray}
\end{latexcode}
\noindent
Moreover, it is very complicated to change the horizontal alignment of
the equality sign on the second line.
\end{itemize}
\needspace{6\baselineskip}
Thus:\index{eqnarray}
\begin{whitebox}
  \centering
  \textbf{NEVER} ever use the \verb+eqnarray+-environment!
\end{whitebox}

To overcome these problems we recommend the
\verb+IEEEeqnarray+-environment.


\subsection{Solution: basic usage of IEEEeqnarray}
\label{sec:IEEEeqnarray_intro}

The \verb+IEEEeqnarray+-environment is a very powerful command with
many options. In this manual we will only introduce some of the most
important functionalities. For more information we refer to the
official manual.\footnote{The official manual
  \texttt{IEEEtran\_HOWTO.pdf} is distributed together with this short
  introduction. The part about \texttt{IEEEeqnarray} can be found in
  Appendix~F.}  First of all, in order to be able to use the
\verb+IEEEeqnarray+-environment, one needs to include the
package\footnote{This package is also distributed together with this
  manual, but it is already included in any up-to-date \LaTeX{}
  distribution. Note that if a document uses the
  \texttt{IEEEtran}-class, then \texttt{IEEEtrantools} is loaded
  automatically and must not be included separately.}
\verb+IEEEtrantools+. Include the following line in the header of your
document:\index{IEEEtrantools}
\begin{latexcodeonly}
\usepackage{IEEEtrantools}
\end{latexcodeonly}

The strength of \verb+IEEEeqnarray+ is the possibility of specifying
the number of \emph{columns} in the equation array. Usually, this
specification will be \verb+{rCl}+, i.e., three columns, the
first column right-justified, the middle one centered with a little
more space around it (therefore we specify capital \verb+C+ instead of
lower-case \verb+c+) and the third column left-justified:\index{IEEEeqnarray}
\begin{latexcode}
\begin{IEEEeqnarray}{rCl}
  a & = & b + c 
  \\
  & = & d + e + f + g + h 
  + i + j + k \nonumber\\
  && +\> l + m + n + o 
  \\
  & = & p + q + r + s
\end{IEEEeqnarray}
\end{latexcode}
\noindent
However, we can specify any number of needed columns. For example,
\verb+{c}+ will give only one column (which is centered) or
\verb+{rCl"l}+ will add a fourth, left-justified column that is
shifted to the right (the spacing is defined by the \verb+"+), e.g.,
for additional specifications. Moreover, beside \verb+l+, \verb+c+,
\verb+r+, \verb+L+, \verb+C+, \verb+R+ for math mode entries, there
also exists \verb+s+, \verb+t+, \verb+u+ for left, centered, and right
text mode entries, respectively. And additional spacing can be added
by \verb+.+ and \verb+/+ and \verb+?+ and \verb+"+ in increasing
order.\footnote{For examples of spacing, we refer to
  Section~\ref{sec:ieeeeqnarraybox}. More spacing types can be found
  in the examples given in Sections~\ref{sec:linebreak} and
  \ref{sec:putting-qed-right}, and in the official manual.} More
details about the usage of \verb+IEEEeqnarray+ will be given in
Section~\ref{sec:common-usage}.

Note that in contrast to \verb+eqnarray+ the spaces around the equality
signs are correct.


\subsection{A remark about consistency}
\label{sec:remark-about-cons}

There are three more issues that have not been mentioned so far, but
that might cause inconsistencies when all three environments,
\verb+equation+, \verb+multline+, and \verb+IEEEeqnarray+, are used
intermixedly:
\begin{itemize}
\item \verb+multline+ \index{multline} allows for an equation
  starting on top of a page, while \verb+equation+ and
  \verb+IEEEeqnarray+ try to put a line of text first, before the
  equation starts. Moreover, the spacing before and after the
  environment is not exactly identical for \verb+equation+,
  \verb+multline+, and \verb+IEEEeqnarray+.

\item \verb+equation+ \index{equation} uses an automatic mechanism to
  move the equation number onto the next line if the expression is too
  long. While this is convenient, sometimes the equation number is
  forced onto the next line, even if there was still enough space
  available on the line:
\begin{latexcode}
\begin{equation}
  a = \sum_{k=1}^n\sum_{\ell=1}^n 
  \sin \bigl(2\pi \, b_k \, 
  c_{\ell} \, d_k \, e_{\ell} \, 
  f_k \, g_{\ell} \, h  \bigr)  
\end{equation}
\end{latexcode}
  \noindent
  With \verb+IEEEeqnarray+ the placement of the equation number is fully
  under our control:\index{IEEEeqnarray!replacing equation}
\begin{latexcode}
\begin{IEEEeqnarray}{c}
  a = \sum_{k=1}^n\sum_{\ell=1}^n 
  \sin \bigl(2\pi \, b_k \, 
  c_{\ell} \, d_k \, e_{\ell} \, 
  f_k \, g_{\ell} \, h  \bigr) 
  \IEEEeqnarraynumspace
  \label{eq:labelc1}
\end{IEEEeqnarray}
\end{latexcode}
  \noindent
  or
\begin{latexcode}
\begin{IEEEeqnarray}{c}
  a = \sum_{k=1}^n\sum_{\ell=1}^n 
  \sin \bigl(2\pi \, b_k \, 
  c_{\ell} \, d_k \, e_{\ell} \, 
  f_k \, g_{\ell} \, h  \bigr) 
  \nonumber\\*
  \label{eq:labelc2}
\end{IEEEeqnarray}
\end{latexcode}

\item \verb+equation+ forces the equation number to appear in normal
  font, even if the equation is within an environment\footnote{A
    typical example of such a situation is an equation inside of a
    theorem that is typeset in italic font.}  of different font:
\begin{latexcode}
\textbf{\textit{\color{red} 
    This is our main result:
  \begin{equation}
    a = b + c
  \end{equation}}}
\end{latexcode}
  \noindent
  \verb+IEEEeqnarray+ respects the settings of the environment: 
\begin{latexcode}
\textbf{\textit{\color{red} 
    This is our main result:
  \begin{IEEEeqnarray}{c}
    a = b + c
  \end{IEEEeqnarray}}}
\end{latexcode}
  \noindent
  If this is undesired, one can change the behavior of
  \verb+IEEEeqnarray+ to behave\footnote{For an explanation of the
    subnumbering, see Section~\ref{sec:equation-numbering}.} like
  \verb+equation+:\index{equation numbers!italic or bold}
\begin{latexcodeonly}
\renewcommand{\theequationdis}{{\normalfont (\theequation)}}
\renewcommand{\theIEEEsubequationdis}{{\normalfont (\theIEEEsubequation)}}
\end{latexcodeonly}
\renewcommand{\theequationdis}{{\normalfont (\theequation)}}
\renewcommand{\theIEEEsubequationdis}{{\normalfont (\theIEEEsubequation)}}
\begin{latexcode}
\textbf{\textit{\color{red} 
    This is our main result:
  \begin{IEEEeqnarray}{rCl}
    a & = & b + c \\
    & = & d + e \IEEEyesnumber
    \IEEEyessubnumber
  \end{IEEEeqnarray}}}
\end{latexcode}
\end{itemize}


\subsection{Using IEEEeqnarray for all situations}
\label{sec:using-ieee-all}

As seen above, there might be reason to rely on \verb+IEEEeqnarray+
exclusively in all situations. 

To replace an \verb+equation+-environment we use \verb+IEEEeqnarray+
with only one column \verb+{c}+, see \eqref{eq:labelc1} and
\eqref{eq:labelc2} above.

\index{IEEEeqnarray!replacing multline}%
Emulating \verb+multline+ is slightly more complicated: we implement
\verb+IEEEeqnarray+ with only one column \verb+{l}+, use
\verb+\IEEEeqnarraymulticol+\footnote{For a more detailed explanation
  of this command, see Section~\ref{sec:ieeeeqnarraymulticol}.} after
the line-break(s) to adapt the column type of the new line, and
manually add some shift:
\begin{latexcode}
\begin{IEEEeqnarray*}{l}
  a + b + c + d + e + f 
  \\ \qquad 
  +\> g + h + i + j + k + l
  \qquad \\
  \IEEEeqnarraymulticol{1}{r}{
    +\> m + n + o + p + q  }
  \IEEEyesnumber
\end{IEEEeqnarray*}
\end{latexcode}



%%%%%%%%%%%%%%%%%%%%%%%%%%%%%%%%%%%%%%%%%%%%%%%%%%%%%%%%%%%%%%%%%
\section{More Details about IEEEeqnarray}
\label{sec:common-usage}

In the following we will describe how we use \verb+IEEEeqnarray+ to
solve the most common situations.


\subsection{Shift to the left: IEEEeqnarraynumspace}
\label{sec:ieeeeqnarraynumspace}

\index{IEEEeqnarraynumspace}%
\index{RHS!slightly too long}%
\index{shift to the left}%
If a line overlaps with the equation number as in
\eqref{eq:faultyeqnarray}, the command
\begin{latexcodeonly}
\IEEEeqnarraynumspace
\end{latexcodeonly}
\noindent
can be used. It has to be added in the corresponding line and makes
sure that the whole equation array is shifted by the size of the
equation numbers (the shift depends on the size of the number!).
Instead of
\begin{latexcode}
\begin{IEEEeqnarray}{rCl}
  a & = & b + c 
  \\
  & = & d + e + f + g + h 
  + i + j + k + m 
  \\
  & = & l + n + o
\end{IEEEeqnarray}
\end{latexcode}
\noindent
we get
\begin{latexcode}
\begin{IEEEeqnarray}{rCl}
  a & = & b + c 
  \\
  & = & d + e + f + g + h 
  + i + j + k + m
  \IEEEeqnarraynumspace\\
  & = & l + n + o
\end{IEEEeqnarray}
\end{latexcode}

Note that if there is not enough space on the line, this shift will
force the numbers to cross the right boundary of the text. So be
sure to check the result! \index{equation numbers!outside of boundary}
\begin{latexcode}
The boundary of the text can be 
seen from this text above the 
equation array. The number is 
clearly beyond it:
\begin{IEEEeqnarray}{rCl}
  a & = & d + e + f + g + h 
  + i + j + k + l + m + n
  \IEEEeqnarraynumspace
\end{IEEEeqnarray}
\end{latexcode}
\noindent
In such a case one needs to wrap the equation somewhere.


\subsection{First line too long: IEEEeqnarraymulticol}
\label{sec:ieeeeqnarraymulticol}

\index{lefteqn}\index{IEEEeqnarraymulticol}\index{LHS!too long}%
If the LHS is too long and as a replacement for the faulty
\verb+\lefteqn{}+-command, \verb+IEEEeqnarray+ offers the
\verb+\IEEEeqnarraymulticol+-command, which works in all situations:
\begin{latexcode}
\begin{IEEEeqnarray}{rCl}
  \IEEEeqnarraymulticol{3}{l}{
    a + b + c + d + e + f 
    + g + h
  }\nonumber\\* \quad
  & = & i + j 
  \\
  & = & k + l + m
\end{IEEEeqnarray}
\end{latexcode}
\noindent
The usage is identical to the \verb+\multicolumns+-command in the
\verb+tabular+-en\-vi\-ron\-ment. The first argument \verb+{3}+
specifies that three columns shall be combined into one, which will be
left-justified \verb+{l}+. We usually add a \verb+*+ to the line-break
\verb+\\+ to prevent a page-break at this
position.\index{$\backslash\backslash*$}

Note that by adapting the \verb+\quad+-command one can easily adapt
the depth of the equation signs,\footnote{I think that one quad is the
  distance that looks good in most cases.} e.g.,
\begin{latexcode}
\begin{IEEEeqnarray}{rCl}
  \IEEEeqnarraymulticol{3}{l}{
    a + b + c + d + e + f 
    + g + h
  }\nonumber\\* \qquad\qquad
  & = & i + j
  \label{eq:label45}
  \\
  & = & k + l + m
\end{IEEEeqnarray}
\end{latexcode}

Note that \verb+\IEEEeqnarraymulticol+ must be the first command in a
cell. This is usually no problem; however, it might be the cause of
some strange compilation errors. For example, one might put a
\verb+\label+-command on the first line inside\footnote{I strongly
  recommend to put each label at the end of the corresponding
  equation; see Section~\ref{sec:equation-numbering}.} of
\verb+IEEEeqnarray+, which is OK in general, but not OK if it is
followed by the \verb+\IEEEeqnarraymulticol+-command. 


\subsection{Line-break: unary versus binary operators}
\label{sec:linebreak}

\index{unary sign}\index{binary sign}\index{line-break}%
If an equation is split onto two or more lines, \LaTeX{} interprets
the first $+$ or $-$ as a sign instead of an operator.  Therefore, it
is necessary to add an additional space \verb+\>+ between the operator
and the term: instead of
\begin{latexcode}
\begin{IEEEeqnarray}{rCl}
  a & = & b + c 
  \\
  & = & d + e + f + g + h 
  + i + j + k \nonumber\\
  && + l + m + n + o 
  \\
  & = & p + q + r + s
\end{IEEEeqnarray}
\end{latexcode}
\noindent
we should write
\begin{latexcode}
\begin{IEEEeqnarray}{rCl}
  a & = & b + c 
  \\
  & = & d + e + f + g + h 
  + i + j + k \nonumber\\
  && +\> l + m + n + o 
  \label{eq:add_space}
  \\
  & = & p + q + r + s
\end{IEEEeqnarray}
\end{latexcode}
\noindent
(Compare the space between $+$ and $l$!)
  
\textbf{Attention:} The distinction between the \emph{unary operator}
(sign) and the \emph{binary operator} (addition/subtraction) is not
satisfactorily solved in \LaTeX{}.\footnote{The problem actually goes
  back to \TeX{}.} In some cases \LaTeX{} will automatically assume
that the operator cannot be unary and will therefore add additional
spacing. This happens, e.g., in front of
\begin{itemize}
\item an operator name like \verb+\log+, \verb+\sin+, \verb+\det+,
  \verb+\max+, etc.,
\item an integral \verb+\int+ or sum \verb+\sum+,
\item a bracket with adaptive size using \verb+\left+ and
    \verb+\right+ (this is in contrast to normal brackets or brackets
  with fixed size like \verb+\bigl(+ and \verb+\bigr)+).
\end{itemize}
This decision, however, might be faulty. E.g., it makes perfect
sense to have a unary operator in front of the logarithm:
\begin{latexcode}
\begin{IEEEeqnarray*}{rCl"s}
  \log \frac{1}{a} 
  & = & -\log a 
  & (binary, wrong) \\
  & = & -{\log a} 
  & (unary, correct)
\end{IEEEeqnarray*}
\end{latexcode}
\noindent
In this case, you have to correct it manually. Unfortunately, there is
no clean way of doing this. To enforce a unary operator, enclosing the
expression following the unary operator and/or the unary operator
itself into curly brackets \verb+{...}+ will usually work. For the
opposite direction, i.e., to enforce a binary operator (as, e.g.,
needed in \eqref{eq:add_space}), the only option is to put in the
correct space \verb+\>+ manually.\footnote{This spacing command adds
  the flexible space \texttt{medmuskip = 4mu plus 2mu minus 4mu}.}

In the following example, compare the spacing between the first
minus-sign on the RHS and $b$ (or $\log b$):
\begin{latexcode}
\begin{IEEEeqnarray*}{rCl's}
  a & = & - b - b - c 
  & (default unary) \\
  & = & {-} {b} - b - c 
  & (default unary, no effect) \\
  & = & -\> b - b - c 
  & (changed to binary) \\
  & = & - \log b - b - d
  & (default binary) \\
  & = & {-} {\log b} - b - d
  & (changed to unary) \\
  & = & - \log b - b {-} d
  & (changed $-d$ to unary) 
\end{IEEEeqnarray*}
\end{latexcode}

\needspace{6\baselineskip}
We learn:
\begin{whitebox}
  \centering
  Whenever you wrap a line, quickly check the result and verify that
  the spacing is correct!
\end{whitebox}


\subsection{Equation numbers and subnumbers}
\label{sec:equation-numbering}

\index{equation numbers}\index{numbers}\index{subnumbers}%
While \verb+IEEEeqnarray+ assigns an equation number to all lines, the
starred version \verb+IEEEeqnarray*+ suppresses all numbers. This
behavior can be changed individually per line by the commands
\begin{center}
  \verb+\IEEEyesnumber+ and \verb+\IEEEnonumber+ (or
  \verb+\nonumber+). 
\end{center}
For subnumbering the corresponding commands
\begin{center}
  \verb+\IEEEyessubnumber+ and \verb+\IEEEnosubnumber+   
\end{center}
are available. These four commands only affect the line on which they
are invoked, however, there also exist starred versions
\begin{center}
  \verb+\IEEEyesnumber*+, \verb+\IEEEnonumber*+, \\
  \verb+\IEEEyessubnumber*+, \verb+\IEEEnosubnumber*+ 
\end{center}
that will remain active until the end of the
\verb+IEEEeqnarray+-environment or until another starred command is
invoked.
\index{IEEEyesnumber}\index{IEEEnonumber}\index{nonumber}
\index{IEEEyessubnumber}\index{IEEEnosubnumber} 

Consider the following extensive example.
\begin{latexcode}
\begin{IEEEeqnarray*}{rCl}
  a
  & = & b_{1}  \\
  & = & b_{2}  \IEEEyesnumber\\
  & = & b_{3}    \\
  & = & b_{4}  \IEEEyesnumber*\\
  & = & b_{5}    \\
  & = & b_{6}    \\
  & = & b_{7}  \IEEEnonumber\\
  & = & b_{8}    \\
  & = & b_{9}  \IEEEnonumber*\\
  & = & b_{10}   \\
  & = & b_{11} \IEEEyessubnumber*\\
  & = & b_{12}   \\
  & = & b_{13} \IEEEyesnumber\\
  & = & b_{14}   \\
  & = & b_{15} 
\end{IEEEeqnarray*}
(\ldots some text\ldots)
\begin{IEEEeqnarray}{rCl}
  \label{eq:bad_placement}
  a
  & = & b_{16} \IEEEyessubnumber*\\
  & = & b_{17}   \\
  & = & b_{18} \IEEEyesnumber
               \IEEEyessubnumber*\\
  & = & b_{19}   \\
  & = & b_{20} \IEEEnosubnumber*\\
  & = & b_{21}   \\
  & = & b_{22} \nonumber\\
  & = & b_{23}   
\end{IEEEeqnarray}
(\ldots more text\ldots)
\begin{IEEEeqnarray}{rCl}
  \IEEEyesnumber\label{eq:block}
  \IEEEyessubnumber*
  a
  & = & b_{24} \\
  & = & b_{25} 
  \label{eq:subeq_b}\\
  & = & b_{26}   
\end{IEEEeqnarray}
\end{latexcode}
\noindent
Note that the behavior in the line 13 (i.e., the line containing
$b_{13}$) is probably unwanted: there the command
\verb+\IEEEyesnumber+ temporarily switches to a normal equation number
(implicitly resetting the subnumbers), but in the subsequent line the
\verb+\IEEEyessubnumber*+ from line 11 takes control again, i.e.,
subnumbering is reactivated. The correct way of increasing the number
and start directly with a new subnumber is shown in line 18 and in
line 24.  Also note that the subnumbering works even across different
\verb+IEEEeqnarray+-environments, as can be seen in line 16.

The best way of understanding the numbering behavior is to note that
in spite of the eight different commands, there are only three
different modes:
\begin{enumerate}
\item No equation number (corresponding to \verb+\IEEEnonumber+).
\item A normal equation number (corresponding to
  \verb+\IEEEyesnumber+): the equation counter is incremented and then
  displayed.
\item An equation number with subnumber (corresponding to
  \verb+\IEEEyessubnumber+): only the subequation counter is
  incremented and then both the equation and the subequation numbers
  are displayed. (\emph{Attention:} If the equation number shall be
  incremented as well, which is usually the case for the start of a
  new subnumbering, then also \verb+\IEEEyesnumber+ has to be given!)
\end{enumerate}
The understanding of the working of these three modes is also
important when using labels to refer to equations. Note that the label
referring to an equation with a subnumber must always be given
\emph{after} the \verb+\IEEEyessubnumber+ command. Otherwise the label
will refer to the current (or future) main number, which is usually
undesired. E.g., the label \verb+eq:bad_placement+ in line 16
points\footnote{To understand this, note that when the
  \texttt{label}-command was invoked, subnumbering was deactivated. So
  the label only refers to a normal equation number. However, no such
  number was active there either, so the label is passed on to line 18
  where the equation counter is incremented for the first time.}
(wrongly) to \eqref{eq:bad_placement}.

A correct example is shown in \eqref{eq:block} and \eqref{eq:subeq_b}:
the label \verb+\label{eq:block}+ refers to the whole block, and the
label \verb+\label{eq:subeq_b}+ refers to the corresponding
subequation.

\needspace{6\baselineskip}
We learn:\index{labels!placing}
\begin{whitebox}
  \centering
  A label should always be put at the end of the equation it belongs
  to\\(i.e., right in front of the line-break \verb+\\+).
\end{whitebox}
Besides preventing unwanted results, this rules also increases the
readability of the source code and prevents a compilation error in the
situation of an \verb+\IEEEeqnarraymul+ \verb+ticol+-command after a
label-definition. \index{IEEEeqnarraymulticol}


\subsubsection*{Hyperlinks}

\index{hyperlinks}\index{hyperref}%
As this document demonstrates, hyperlinking works (almost) seamlessly
with \verb+IEEEeqn+ \verb+array+. For this document we simply included
\begin{latexcodeonly}
\usepackage[colorlinks=true,linkcolor=blue]{hyperref}
\end{latexcodeonly}
\noindent
in the header, and then all references automatically become hyperlinks.

There is only one small issue that you might have noticed already: the
reference \eqref{eq:block} points into nirvana. The reason for this is
that there is no actual equation number \eqref{eq:block} generated and
therefore \verb+hyperref+ does not create the corresponding
hyperlink. This can be fixed, but requires some more advanced
\LaTeX{}-programming. Copy-paste the following code into the document
header (or your stylefile):
\begin{latexcodeonly}
\makeatletter
\def\IEEElabelanchoreqn#1{\bgroup
\def\@currentlabel{\p@equation\theequation}\relax
\def\@currentHref{\@IEEEtheHrefequation}\label{#1}\relax
\Hy@raisedlink{\hyper@anchorstart{\@currentHref}}\relax
\Hy@raisedlink{\hyper@anchorend}\egroup}
\makeatother
\newcommand{\subnumberinglabel}[1]{\IEEEyesnumber
  \IEEEyessubnumber*\IEEElabelanchoreqn{#1}}
\end{latexcodeonly}
\noindent
\index{IEEElabelanchoreqn}\index{subnumberinglabel}%
Now, \verb+\IEEElabelanchoreqn{...}+ creates an anchor for a hyperlink
to an invisible equation number. The command \verb+\subnumberinglabel+
then sets this anchor and at the same time activates subnumbering,
simplifying our typesetting:\index{reference to subequations}
\index{hyperlinks!block of subequations}
\begin{latexcode}
We have
\begin{IEEEeqnarray}{rCl}
  \subnumberinglabel{eq:block2}
  a & = & b + c
  \label{eq:block2_eq1}\\
  & = & d + e
  \label{eq:block2_eq2}
\end{IEEEeqnarray}
and
\begin{IEEEeqnarray}{c}
  \IEEEyessubnumber*
  f = g - h + i
  \label{eq:block2_eq3}
\end{IEEEeqnarray}
\end{latexcode}
\noindent
Now \eqref{eq:block2} refers to the whole block (the hyperlink points
to the first line of the first equation array), and
\eqref{eq:block2_eq1}, \eqref{eq:block2_eq2}, and
\eqref{eq:block2_eq3} point to the corresponding subequations.


\subsubsection*{Alternative subnumbers: subequations}

\index{subequations}\index{hyperlinks!block of subequations}%
We conclude this section by remarking that \verb+IEEEeqnarray+ is
fully compatible with the \verb+subequations+-environment. Thus,
\eqref{eq:block2} can also be created in the following way:
\begin{latexcode}
We have
\begin{subequations}
  \label{eq:block2_alt}
  \begin{IEEEeqnarray}{rCl}
    a & = & b + c
    \label{eq:block2_eq1_alt}\\
    & = & d + e
    \label{eq:block2_eq2_alt}
  \end{IEEEeqnarray}
  and
  \begin{IEEEeqnarray}{c}
    f = g - h + i
    \label{eq:block2_eq3_alt}
  \end{IEEEeqnarray}
\end{subequations}
\end{latexcode}
\noindent
Note, however, that the hyperlink of \eqref{eq:block2_alt} points to
the beginning of the \verb+subequations+-environment and not onto the
first line of the equation array as in \eqref{eq:block2}!


\subsection{Page-breaks inside of IEEEeqnarray}
\label{sec:pagebr-inside-ieee}

\index{page-breaks}\index{IEEEeqnarray!page-breaks}%
\index{interdisplaylinepenalty}%
By default, \verb+amsmath+ does not allow page-breaks within multiple
equations, which usually is too restrictive, particularly, if a
document contains long equation arrays. This behavior can be changed
by putting the following line into the document header:
\begin{latexcodeonly}
\interdisplaylinepenalty=xx 
\end{latexcodeonly}
\noindent
Here, \verb+xx+ is some number: the larger this number, the less
likely it is that an equation array is broken over to the next
page. So, a value 0 fully allows page-breaks, a value 2500 allows
page-breaks, but only if \LaTeX{} finds no better solution, or a value
10'000 basically prevents page-breaks (which is the default given in
\verb+amsmath+).\footnote{I usually use a value 1000 that in principle
  allows page-breaks, but still asks \LaTeX{} to check if there is no
  other way.}



%%%%%%%%%%%%%%%%%%%%%%%%%%%%%%%%%%%%%%%%%%%%%%%%%%%%%%%%%%%%%%%%%
\section{Advanced Typesetting}
\label{sec:advanced-typesetting}

In this section we address a couple of more advanced typesetting
problems and tools.


\subsection{Aligning several separate equation arrays}
\label{sec:align-sever-separ}

\index{noalign}\index{IEEEeqnarray!align across blocks}%
\index{aligning across blocks}%
Sometimes it looks elegant if one can align not just the equations
within one array, but between several arrays (with regular text in
between). This can be achieved by actually creating one single large
array and add additional text in between. For example,
\eqref{eq:block2} could be typeset as follows:
\begin{latexcode}
We have
\begin{IEEEeqnarray}{rCl}
  \subnumberinglabel{eq:block3}
  a & = & b + c
  \label{eq:block3_eq1}\\
  & = & d + e
  \label{eq:block3_eq2}\\
  \noalign{\noindent and
    \vspace{2\jot}}
  f & = & g - h + i
  \label{eq:block3_eq3}
\end{IEEEeqnarray}
\end{latexcode}
\noindent
Note how the equality-sign in \eqref{eq:block3_eq3} is aligned to the
equality-signs of \eqref{eq:block3_eq1} and \eqref{eq:block3_eq2}.

In the code, we add the text ``and'' into the array using the command
\verb+\noalign{...}+ and then manually add some vertical spacing.


\subsection{IEEEeqnarraybox: general tables and arrays}
\label{sec:ieeeeqnarraybox}

\index{IEEEeqnarraybox}\index{IEEEeqnarrayboxt}\index{IEEEeqnarrayboxm}%
\index{array}\index{tabular}%
The package \verb+IEEEtrantools+ also provides the environment
\verb+IEEEeqnarraybox+. This is basically the same as
\verb+IEEEeqnarray+ but with the difference that it can be nested
within other structures. Therefore it does not generate a full
equation itself nor an equation number. It can be used both in
text-mode (e.g., inside a table) or in math-mode (e.g., inside an
equation).\footnote{In case one does not want to let
  \texttt{IEEEeqnarraybox} to detect the mode automatically, but to
  force one of these two modes, there are two subforms:
  \texttt{IEEEeqnarrayboxm} for math-mode and
  \texttt{IEEEeqnarrayboxt} for text-mode.}  Hence,
\verb+IEEEeqnarraybox+ is a replacement both for \verb+array+ and
\verb+tabular+.

\begin{latexcode}
This is a silly table:
\begin{center}
  \begin{IEEEeqnarraybox}{t.t.t}
    \textbf{Item} & 
    \textbf{Color} & 
    \textbf{Count} \\
    cars & green & 17 \\
    trucks & red & 4 \\
    bikes & blue & 25
  \end{IEEEeqnarraybox}
\end{center}
\end{latexcode}
\noindent
\index{table}%
Note that \verb+t+ in the argument of \verb+IEEEeqnarraybox+ stands
for \emph{centered text} and \verb+.+ adds space between the
columns. Further possible arguments are \verb+s+ for \emph{left text},
\verb+u+ for \emph{right text}, \verb+v+ for a vertical line, and
\verb+V+ for a vertical double-line. More details can be found in
Tables~IV and V on page~18 in the manual \verb+IEEEtran_HOWTO.pdf+.

Another example:\footnote{For another way of
  generating case distinctions, see
  Section~\ref{sec:case-distinctions}.}
\begin{latexcode}
\begin{equation}
  P_U(u) = \left\{ \,
    \begin{IEEEeqnarraybox}[][c]{l?s}
      \IEEEstrut
      0.1 & if $u=0$, \\
      0.3 & if $u=1$, \\
      0.6 & if $u=2$.
      \IEEEstrut
    \end{IEEEeqnarraybox}
  \right.
  \label{eq:example_left_right1}
\end{equation}
\end{latexcode}
\noindent\index{case distinction}%
\index{IEEEstrut}\index{spacing}\index{vertical spacing}%
Here \verb+?+ is a large horizontal space between the columns, and
\verb+\IEEEstrut+ adds a tiny space above the first and below the
bottom line. Moreover, note that the second optional argument
\verb+[c]+ makes sure that the \verb+IEEEeqnarraybox+ is vertically
centered. The other possible values for this option are \verb+[t]+ for
aligning the first row with the surrounding baseline and \verb+[b]+
for aligning the bottom row with the surrounding baseline. Default is
\verb+[b]+, i.e., if we do not specify this option, we get the
following (in this case unwanted) result:
\begin{latexcode}
\begin{equation*}
  P_U(u) = \left\{ \,
    \begin{IEEEeqnarraybox}{l?s}
      0.1 & if $u=0$, \\
      0.3 & if $u=1$, \\
      0.6 & if $u=2$.
    \end{IEEEeqnarraybox}
  \right.
\end{equation*}
\end{latexcode}
\noindent 
We also dropped \verb+\IEEEstrut+ here with the result that the curly
bracket is slightly too small at the top line.  

Actually, these manually placed \verb+\IEEEstrut+ commands are rather
tiring. More\-over, when we would like to add vertical lines in a table,
a first naive application of \verb+IEEEeqnarraybox+ yields the
following:
\begin{latexcode}
\begin{equation*}
  \begin{IEEEeqnarraybox}{c'c;v;c'c'c}
    D_1 & D_2 && X_1 & X_2
    & X_3
    \\\hline
    0 & 0 && +1 & +1 & +1\\
    0 & 1 && +1 & -1 & -1\\
    1 & 0 && -1 & +1 & -1\\
    1 & 1 && -1 & -1 & +1
  \end{IEEEeqnarraybox}
\end{equation*}
\end{latexcode}
\noindent\index{IEEEeqnarraystrutmode}%
We see that \verb+IEEEeqnarraybox+ makes a complete line-break after
each line. This is of course unwanted. Therefore, the command
\verb+\IEEEeqnarraystrutmode+ is provided that switches the spacing
system completely over to struts:
\begin{latexcode}
\begin{equation*}
  \begin{IEEEeqnarraybox}[
     \IEEEeqnarraystrutmode
    ]{c'c;v;c'c'c}
    D_1 & D_2 && X_1 & X_2 & X_3
    \\\hline
    0 & 0 && +1 & +1 & +1\\
    0 & 1 && +1 & -1 & -1\\
    1 & 0 && -1 & +1 & -1\\
    1 & 1 && -1 & -1 & +1
  \end{IEEEeqnarraybox}
\end{equation*}
\end{latexcode}
\noindent\index{table}\index{IEEEeqnarraybox}%
The strutmode also easily allows to ask for more ``air'' between each
line and thereby eliminating the need of manually adding an
\verb+\IEEEstrut+:
\begin{latexcode}
\begin{equation*}
  \begin{IEEEeqnarraybox}[
     \IEEEeqnarraystrutmode
     \IEEEeqnarraystrutsizeadd{3pt}
     {1pt}
    ]{c'c/v/c'c'c}
    D_1 & D_2 & & X_1 & X_2 & X_3
    \\\hline
    0 & 0 && +1 & +1 & +1\\
    0 & 1 && +1 & -1 & -1\\
    1 & 0 && -1 & +1 & -1\\
    1 & 1 && -1 & -1 & +1
  \end{IEEEeqnarraybox}
\end{equation*}
\end{latexcode}
\noindent
Here the first argument of \verb+\IEEEeqnarraystrutsizeadd{3pt}{1pt}+
adds space above into each line, the second adds space below into each
line.
\index{IEEEeqnarraystrutsizeadd}


\subsection{Case distinctions}
\label{sec:case-distinctions}

\index{case distinction}%
Case distinctions can be generated using \verb+IEEEeqnarraybox+ as
shown in Section~\ref{sec:ieeeeqnarraybox}. However, in the standard
situation the usage of \verb+cases+ is simpler and we therefore
recommend to use this:\index{cases}%
\begin{latexcode}
\begin{equation}
  P_U(u) = 
  \begin{cases}
    0.1 & \text{if } u=0,
    \\
    0.3 & \text{if } u=1,
    \\
    0.6 & \text{if } u=2.
  \end{cases}
\end{equation}
\end{latexcode}

For more complicated examples we do need to rely on
\verb+IEEEeqnarraybox+:
\begin{latexcode}
\begin{equation}
  \left.
  \begin{IEEEeqnarraybox}[\IEEEeqnarraystrutmode
    \IEEEeqnarraystrutsizeadd{2pt}{2pt}][c]{rCl}
    x & = & a + b\\
    y & = & a - b
  \end{IEEEeqnarraybox}
  \, \right\}  \iff  \left\{ \,
  \begin{IEEEeqnarraybox}[
    \IEEEeqnarraystrutmode
    \IEEEeqnarraystrutsizeadd{7pt}
    {7pt}][c]{rCl}
    a & = & \frac{x}{2} + \frac{y}{2} 
    \\
    b & = & \frac{x}{2} - \frac{y}{2} 
  \end{IEEEeqnarraybox}
  \right.
  \label{eq:example_left_right2}
\end{equation}
\end{latexcode}

If we would like to have a distinct equation number for each case, the
package\index{cases!individual numbers}\index{numcases}%
\begin{latexcodeonly}
\usepackage{cases}
\end{latexcodeonly}
\noindent
provides by far the easiest solution:
\begin{latexcode}
\begin{numcases}{|x|=} 
  x & for $x \geq 0$,
  \\ 
  -x & for $x < 0$.
\end{numcases}
\end{latexcode}
\noindent
Note the differences to the usual \verb+cases+-environment:
\begin{itemize}
\item The left-hand side must be typeset as compulsory argument to the
  environment. 
\item The second column is not in math-mode but directly in text-mode.
\end{itemize}
For subnumbering we can use the corresponding
\verb+subnumcases+-environment:\index{subnumcases}%
\begin{latexcode}
\begin{subnumcases}{P_U(u)=} 
  0.1 & if $u=0$,
  \\
  0.3 & if $u=1$,
  \\
  0.6 & if $u=2$.
\end{subnumcases}
\end{latexcode}


\subsection{Grouping numbered equations with a bracket}
\label{sec:group-numb-equat}

\index{grouping equations}\index{brackets}%
Sometimes, one would like to group several equations together with a
bracket. We have already seen in \eqref{eq:example_left_right2} how
this can be achieved by using \verb+IEEEeqnarraybox+ inside of an
\verb+equation+-environment:
\begin{latexcode}
\begin{equation}
  \left\{
    \begin{IEEEeqnarraybox}[
      \IEEEeqnarraystrutmode
      \IEEEeqnarraystrutsizeadd{2pt}
      {2pt}
      ][c]{rCl}
      \dot{x} & = & f(x,u)
      \\
      x+\dot{x} & = & h(x)
    \end{IEEEeqnarraybox}
  \right.
\end{equation}
\end{latexcode}
\noindent
The problem here is that since the equation number is provided by the
\verb+equation+-environment, we only get one equation number. But here
in this context, an individual number for each equation would make
much more sense.

We could again rely on \verb+numcases+ (see
Section~\ref{sec:case-distinctions}), but then we have no way of
aligning the equations horizontally:
\begin{latexcode}
\begin{numcases}{}
  \dot{x} = f(x,u)
  \\
  x+\dot{x} = h(x)
\end{numcases}
\end{latexcode}
\noindent
Note that misusing the second column of \verb+numcases+ is not an
option either:
\begin{latexcode}
\ldots very poor typesetting:
\begin{numcases}{}
  \dot{x} & $\displaystyle 
  = f(x,u)$
  \\
  x+\dot{x} & $\displaystyle
  = h(x)$
\end{numcases}
\end{latexcode}

The problem can be solved using \verb+IEEEeqnarray+: We define an
extra column on the most left that will only contain the
bracket. However, as this bracket needs to be far higher than the line
where it is defined, the trick is to use \verb+\smash+ to make its
true height invisible to \verb+IEEEeqnarray+, and then ``design'' its
height manually using the \verb+\IEEEstrut+-command. The number of
necessary \emph{jots} depends on the height of the equation and needs
to be adapted
manually:\index{jot}\index{$\backslash\backslash*$}\index{smash}%
\begin{latexcode}
\begin{IEEEeqnarray}{rrCl} 
  & \dot{x} & = & f(x,u) 
  \\*
  \smash{\left\{
      \IEEEstrut[8\jot]
    \right.}
  & x+\dot{x} & = & h(x)
  \\*
  & x+\ddot{x} & = & g(x) 
\end{IEEEeqnarray}
\end{latexcode}
\noindent
The star in \verb+\\*+ is used to prevent the possibility of a
page-break within the structure.

This works fine as long as the number of equations is odd and the
total height of the equations above the middle row is about the same
as the total height of the equations below. For example, for five
equations (this time using subnumbers for a change):\index{IEEEstrut}
\begin{latexcode}
\begin{IEEEeqnarray}{rrCl} 
  \subnumberinglabel{eq:block4}
  & a_1 + a_2 & = & f(x,u) 
  \\*
  & a_1 & = & \frac{1}{2}h(x) 
  \\*
  \smash{\left\{
      \IEEEstrut[16\jot]
    \right.}
  & b & = & g(x,u)
  \\*
  & y_{\theta} & = & 
  \frac{h(x)}{10}
  \\*
  & b^2 + a_2 & = & g(x,u) 
\end{IEEEeqnarray}
\end{latexcode}
\noindent
However, if the heights of the equations differ greatly:
\begin{latexcode}
Bad example: uneven height
distribution:
\begin{IEEEeqnarray}{rrCl} 
  \subnumberinglabel{eq:uneven} 
  & a_1 + a_2 & = & 
  \sum_{k=1}^{\frac{M}{2}} f_k(x,u) 
  \\*
  \smash{\left\{
      \IEEEstrut[15\jot]
    \right.}
  & b & = & g(x,u)
  \\*
  & y_{\theta} & = & h(x) 
\end{IEEEeqnarray}
\end{latexcode}
\noindent
or if the number of equations is even:
\begin{latexcode}
Another bad example:
even number of equations:
\begin{IEEEeqnarray}{rrCl} 
  & \dot{x} & = & f(x,u) 
  \\*
  \smash{\left\{
      \IEEEstrut[8\jot]
    \right.} \nonumber
  \\*
  & y_{\theta} & = & h(x)  
\end{IEEEeqnarray}
\end{latexcode}
\noindent
we get into a problem. To solve this issue, we need manual
tinkering. In the latter case the basic idea is to use a hidden row at
a place of our choice. To make the row hidden, we need to manually
move down the row above the hidden row, and to move up the row below,
both by about half the usual line spacing:
\begin{latexcode}
\begin{IEEEeqnarray}{rrCl} 
  & \dot{x} & = & f(x,u) 
  \\*[-0.625\normalbaselineskip] 
% start invisible row
  \smash{\left\{
      \IEEEstrut[6\jot]
    \right.} \nonumber
% end invisible row
  \\*[-0.625\normalbaselineskip]
  & x+\dot{x} & = & h(x)  
\end{IEEEeqnarray}
\end{latexcode}
\noindent
In the former case of unequally sized equations, we can put the
bracket on an individual row anywhere and then moving it up or down
depending on how we need it. The example \eqref{eq:uneven} with the
three unequally sized equations then looks as follows:
\begin{latexcode}
\begin{IEEEeqnarray}{rrCl} 
  \subnumberinglabel{eq:uneven2}
  & a_1 + a_2 & = & 
  \sum_{k=1}^{\frac{M}{2}} f_k(x,u) 
  \\*[-0.1\normalbaselineskip]
  \smash{\left\{
      \IEEEstrut[12\jot]
    \right.} \nonumber
  \\*[-0.525\normalbaselineskip]
  & b & = & g(x,u)
  \\*
  & y_{\theta} & = & h(x) 
\end{IEEEeqnarray}
\end{latexcode}
\noindent
Note how we can move the bracket up and down by changing the amount of
shift in both \verb+\\*[...\normalbaselineskip]+-commands: if we add
$+2$ to the first and $-2$ to the second command (which makes sure
that in total we have added $2-2=0$), we obtain:
\begin{latexcode}
\begin{IEEEeqnarray}{rrCl} 
  \subnumberinglabel{eq:uneven3}
  & a_1 + a_2 & = & 
  \sum_{k=1}^{\frac{M}{2}} f_k(x,u) 
  \\*[1.9\normalbaselineskip]
  \smash{\left\{
      \IEEEstrut[12\jot]
    \right.} \nonumber
  \\*[-2.525\normalbaselineskip]
  & b & = & g(x,u)
  \\*
  & y_{\theta} & = & h(x) 
\end{IEEEeqnarray}
\end{latexcode}
\index{normalbaselineskip}%


\subsection{Matrices}
\label{sec:matrices}

\index{matrices}%
Matrices could be generated by \verb+IEEEeqnarraybox+, however, the
environment \verb+pmatrix+ is easier to use:
\begin{latexcode}
\begin{equation}
  \mathsf{P} = 
  \begin{pmatrix}
    p_{11} & p_{12} & \ldots 
    & p_{1n} \\
    p_{21} & p_{22} & \ldots 
    & p_{2n} \\
    \vdots & \vdots & \ddots 
    & \vdots \\
    p_{m1} & p_{m2} & \ldots 
    & p_{mn} 
  \end{pmatrix}
\end{equation}
\end{latexcode}
\noindent
Note that it is not necessary to specify the number of columns (or
rows) in advance. More possibilities are \verb+bmatrix+ (for matrices
with square brackets), \verb+Bmatrix+ (curly brackets), \verb+vmatrix+
($|$), \verb+Vmatrix+ ($\|$), and \verb+matrix+ (no brackets at all).
\index{pmatrix}\index{matrix}\index{bmatrix}\index{Bmatrix}%
\index{vmatrix}\index{Vmatrix}%


\subsection{Adapting the size of brackets}
\label{sec:adapt-size-brack}

\index{left-right}\index{left}\index{right}%
\index{adapting bracket-size}%
\index{brackets!adapting size}%
\LaTeX{} offers the functionality of brackets being automatically
adapted to the size of the expression they embrace. This is done using
the pair of directives \verb+\left+ and \verb+\right+:
\begin{latexcode}
\begin{equation}
  f \left( \sum_{k=1}^n b_k \right)
  =
  f \Biggl( \sum_{k=1}^n b_k \Biggr)
  \label{eq:adapt_bracket_size}
\end{equation}
\end{latexcode}
\noindent
Unfortunately, the \verb+\left+-\verb+\right+ pair has two
weaknesses. First, it adds too much space before and after the
brackets. Compare the space between the $f$ and the opening bracket in
\eqref{eq:adapt_bracket_size}! This can easily be remedied by
including the following two lines into the document
header:\index{mleftright}%
\begin{latexcodeonly}
\usepackage{mleftright}
\mleftright
\end{latexcodeonly}
\noindent
\mleftright%
Then, the example~\eqref{eq:adapt_bracket_size} looks as
follows:
\begin{latexcode}
\begin{equation}
  f \left( \sum_{k=1}^n b_k \right)
  =
  f \Biggl( \sum_{k=1}^n b_k \Biggr)
\end{equation}
\end{latexcode}
\noindent

Second, in certain situations the chosen bracket size is slightly too
big. For example, this happens when expressions with large
superscripts are typeset in a smaller font size like in the following
footnote.\footnote{In footnotes, we get $\left(a^{(1)}\right)$. I
  suggest to choose the bracket size manually using \texttt{bigl(} and
  \texttt{bigr)} in such a case: $\bigl(a^{(1)}\bigr)$.}
In this case it is easiest to adapt the bracket size manually.
\index{bigl-bigr} 


\subsubsection*{Usage}

The brackets do not need to be round, but can be of various types,
e.g.,
\begin{latexcode}
\begin{equation*}
  \left\| \left( 
    \left[ \left\{ \left| 
      \left\lfloor \left\lceil
        \frac{1}{2}
      \right\rceil \right\rfloor  
    \right| \right\} \right] 
  \right) \right\|
\end{equation*}
\end{latexcode}
\noindent
It is important to note that \verb+\left+ and \verb+\right+ always
must occur as a pair, but --- as we have just seen --- they can be
nested. Moreover, the brackets do not need to match:
\begin{latexcode}
\begin{equation*}
  \left( \frac{1}{2}, 1 \right]
  \subset \mathbb{R}
\end{equation*}
\end{latexcode}
\noindent
One side can even be made invisible by using a dot instead of a
bracket (\verb+\left.+ or \verb+\right.+). We have already seen such
examples in \eqref{eq:example_left_right1} or
\eqref{eq:example_left_right2}. 

For an additional element in between a \verb+\left+-\verb+\right+
pair that should have the same size as the surrounding brackets, the
command \verb+\middle+ is available:\index{middle}
\begin{latexcode}
\begin{equation}
  H\left(X \, \middle| \, 
    \frac{Y}{X} \right) 
\end{equation}
\end{latexcode}
\noindent
Here both the size of the vertical bar and of the round brackets are
adapted according to the size of $\frac{Y}{X}$.


\subsubsection*{Line-break}

\index{line-break}\index{left-right!line-break}%
Unfortunately, \verb+\left+-\verb+\right+ pairing cannot be done
across a line-break. So, if we wrap an equation using \verb+multline+
or \verb+IEEEeqnarray+, we cannot have a \verb+\left+ before and the
  corresponding \verb+\right+ after a line-break \verb+\\+. In a first
attempt, we might try to fix this by introducing a \verb+\right.+
before the line-break and a \verb+\left.+ after the line-break, as
  shown in the following example:
\begin{latexcode}
\begin{IEEEeqnarray}{rCl}
  a & = & \log \left( 1 \right.  
  \nonumber\\
  && \qquad \left. + \> 
    \frac{b}{2} \right)
  \label{eq:wrong_try}  
\end{IEEEeqnarray}
\end{latexcode}
\noindent
As can be seen from this example, this approach usually does not work
because the sizes of the opening and closing brackets do not match
anymore. In the example \eqref{eq:wrong_try}, the opening bracket
adapts its size to ``$1$'', while the closing bracket adapts its size
to $\frac{b}{2}$.

There are two ways to try to fix this. The by far easier way is to
choose the bracket size manually:
\begin{latexcode}
\begin{IEEEeqnarray}{rCl}
  a & = & \log \biggl( 1
  \nonumber\\
  && \qquad +\> 
  \frac{b}{2} \biggr)
\end{IEEEeqnarray}
\end{latexcode}
\noindent\index{bigl-bigr}%
There are four sizes available: in increasing order \verb+\bigl+,
\verb+\Bigl+, \verb+\biggl+, and \verb+\Biggl+ (with the corresponding
\verb+..r+-version).  This manual approach will fail, though, if the
expression in the brackets requires a bracket size larger than
\verb+\Biggl+, as shown in the following example:
\begin{latexcode}
\begin{IEEEeqnarray}{rCl}
  a & = & \log \Biggl( 1
  \nonumber\\
  && \qquad + \sum_{k=1}^n 
  \frac{e^{1+\frac{b_k^2}{c_k^2}}}
  {1+\frac{b_k^2}{c_k^2}}
  \Biggr)
  \label{eq:sizecorr1}
\end{IEEEeqnarray}
\end{latexcode}
\noindent
For this case we need a trick: since we want to rely on a 
\begin{latexcodeonly}
\left( ... \right. \\ \left. ... \right)  
\end{latexcodeonly}
\noindent
construction, we need to make sure that both pairs are adapted to the
same size. To that goal we define the following command in the
document header:\index{sizecorr}\index{phantom}
\begin{latexcodeonly}
\newcommand{\sizecorr}[1]{\makebox[0cm]{\phantom{$\displaystyle #1$}}} 
\end{latexcodeonly}
\noindent
We then pick the larger of the two expressions on either side of
\verb+\\+ (in \eqref{eq:sizecorr1} this is the term on the second
line) and typeset it a second time also on the other side of the
line-break (inside of the corresponding \verb+\left+-\verb+\right+
pair). However, since we do not actually want to see this expression
there, we put it into \verb+\sizecorr{}+ and thereby make it both
invisible and of zero width (but correct height!). In the example
\eqref{eq:sizecorr1} this looks as follows:
\begin{latexcode}
\begin{IEEEeqnarray}{rCl}
  a & = & \log \left( 
% copy-paste from below, invisible
  \sizecorr{ 
    \sum_{k=1}^n 
    \frac{e^{1+\frac{b_k^2}{c_k^2}}}
    {1+\frac{b_k^2}{c_k^2}}
  } 
% end copy-paste
  1 \right. \nonumber\\
  && \qquad \left. + \sum_{k=1}^n 
    \frac{e^{1+\frac{b_k^2}{c_k^2}}}
    {1+\frac{b_k^2}{c_k^2}}
  \right)
  \label{eq:sizecorr2}
\end{IEEEeqnarray}
\end{latexcode}
\noindent
Note how the expression inside of \verb+\sizecorr{}+ does not actually
appear, but is used for computing the correct bracket size.


\subsection{Framed equations}
\label{sec:framed-equations}

\index{framed equations}%
To generate equations that are framed, one can use the
\verb+\boxed{...}+-command.  Unfortunately, this usually will yield a
too tight frame around the equation:\index{boxed}
\begin{latexcode}
\begin{equation}
  \boxed{
    a = b + c
  }
\end{equation}
\end{latexcode}
\noindent
To give the frame a little bit more ``air'' we need to redefine the
length-variable \verb+\fboxsep+. We do this in a way that restores its
original definition afterwards:\index{fboxsep}
\begin{latexcode}
\begin{equation}
  \newlength{\fboxstore}
  \setlength{\fboxstore}{\fboxsep}
  \setlength{\fboxsep}{6pt}
  \boxed{
    a = b + c
  }
  \setlength{\fboxsep}{\fboxstore}
\end{equation}
\end{latexcode}
\noindent
Note that the \verb+\newlength+-command must be given only once per
document. To ease one's life, we recommend to define a macro for this
in the document header:\index{eqbox}
\begin{latexcodeonly}
\newlength{\eqboxstorage}
\newcommand{\eqbox}[1]{
  \setlength{\eqboxstorage}{\fboxsep}
  \setlength{\fboxsep}{6pt}
  \boxed{#1}
  \setlength{\fboxsep}{\eqboxstorage}
}
\end{latexcodeonly}
\noindent
Now the framed equation can be produced as follows:
\begin{latexcode}
\begin{equation}
  \eqbox{
    a = b + c
  }
\end{equation}
\end{latexcode}

In case of \verb+multline+ or \verb+IEEEeqnarray+ this approach does
not work because the \verb+boxed{...}+ command does not allow
line-breaks or similar. Therefore we need to rely on
\verb+IEEEeqnarraybox+ for boxes around equations on several lines:
\begin{latexcode}
\begin{equation}
  \eqbox{
    \begin{IEEEeqnarraybox}{rCl}
      a & = & b + c 
      \\
      & = & d + e + f + g + h 
      + i + j + k \\
      && +\> l + m + n + o 
      \\
      & = & p + q + r + s
    \end{IEEEeqnarraybox}
  }
\end{equation}
\end{latexcode}

Some comments:
\begin{itemize}
\item The basic idea here is to replace the original
  \verb+IEEEeqnarray+ command by a \verb+IEEEeqnarraybox+ and then
  wrap everything into an \verb+equation+-environment.

\item The equation number is produced by the surrounding
  \verb+equation+-environment. If we would like to have the equation
  number vertically centered, we need to center the
  \verb+IEEEeqnarraybox+:
\begin{latexcode}
\begin{equation}
  \eqbox{
    \begin{IEEEeqnarraybox}[][c]{rCl}
      a & = & b + c + d + e 
      + f + g + h 
      \\
      && +\> i + j + k + l 
      + m + n 
      \\
      && +\> o + p + q
    \end{IEEEeqnarraybox}
  }
\end{equation}
\end{latexcode}
\noindent
in constrast to
\begin{latexcode}
\begin{equation}
  \eqbox{
    \begin{IEEEeqnarraybox}{rCl}
      a & = & b + c + d + e 
      + f + g + h 
      \\
      && +\> i + j + k + l 
      + m + n 
      \\
      && +\> o + p + q
    \end{IEEEeqnarraybox}
  }
\end{equation}
\end{latexcode}

\item When changing the \verb+IEEEeqnarray+ into a
  \verb+IEEEeqnarraybox+, be careful to delete any remaining
  \verb+\nonumber+ or \verb+\IEEEnonumber+ commands inside of the
  \verb+IEEEeqnar+ \verb+raybox+!  Since \verb+IEEEeqnarraybox+ does
  not know equation numbers anyway, any remaining \verb+\nonumber+
  command will ``leak'' through and prevent \verb+equation+ to put a
  number!
\begin{latexcode}
\begin{equation}
  \eqbox{
    \begin{IEEEeqnarraybox}{rCl}
      a & = & b + c + d + e 
      + f + g + h \nonumber\\
      && +\> i + j + k + l 
    \end{IEEEeqnarraybox}
  }
\end{equation}
\end{latexcode}

\end{itemize}


\subsection{Fancy frames}
\label{sec:more-fancy-frames}

\index{framed equations!fancy}\index{fancy frames}%
\index{mdframed}\index{amsthm}%
Fancier frames can be produced using the \verb+mdframed+ package.  Use
the following commands in the header of your document:\footnote{The
  \texttt{mdframed}-package should be loaded after amsthm.sty.}
\begin{latexcodeonly}
\usepackage{tikz}
\usetikzlibrary{shadows} %defines shadows
\usepackage[framemethod=tikz]{mdframed} 
\end{latexcodeonly}
\noindent
Then we can produce all kinds of fancy frames. We start by defining a
certain style (still in the header of your document):
\begin{latexcodeonly}
\global\mdfdefinestyle{myboxstyle}{%
  shadow=true,
  linecolor=black,
  shadowcolor=black,
  shadowsize=6pt,
  nobreak=false,
  innertopmargin=10pt,
  innerbottommargin=10pt,
  leftmargin=5pt,
  rightmargin=5pt,
  needspace=1cm,
  skipabove=10pt,
  skipbelow=15pt,
  middlelinewidth=1pt,
  afterlastframe={\vspace{5pt}},
  aftersingleframe={\vspace{5pt}},
  tikzsetting={%
    draw=black,
    very thick}
}
\end{latexcodeonly}
\noindent
These settings are quite self-explanatory. Just play around! Now we
define different types of framed boxes:
\index{whitebox}\index{graybox}\index{framed equations!breakable}%
\index{blockwhitebox}\index{blockgraybox}\index{blockbox}%
\begin{latexcodeonly}
% framed box that allows page-breaks
\newmdenv[style=myboxstyle]{whitebox}
\newmdenv[style=myboxstyle,backgroundcolor=black!20]{graybox}

% framed box that CANNOT be broken at end of page
\newmdenv[style=myboxstyle,nobreak=true]{blockwhitebox}
\newmdenv[style=myboxstyle,backgroundcolor=black!20,nobreak=true]{blockgraybox}

% invisible box that CANNOT be broken at end of page
\newmdenv[nobreak=true,hidealllines=true]{blockbox}
\end{latexcodeonly}
\noindent
As the name suggests, the \verb+graybox+ adds a gray background color
into the box, while the background in \verb+whitebox+ remains
white. Moreover, \verb+blockwhitebox+ creates the same framed box as
\verb+whitebox+, but makes sure that whole box is typeset onto one
single page, while the regular \verb+whitebox+ can be split onto two
(or even more) pages.

Examples:
\begin{latexcode}
\begin{whitebox}
  \begin{IEEEeqnarray}[
      \vspace{-\baselineskip}
    ]{rCl}
    a & = & b + c 
    \\
    & = & d + e
  \end{IEEEeqnarray}
\end{whitebox}
\end{latexcode}
\noindent
or
\begin{latexcode}
\begin{graybox}
  \begin{theorem}
    This is a fancy theorem: 
    we know by now that
    \begin{equation}
      a = b + c.
    \end{equation}
  \end{theorem}
\end{graybox}
\end{latexcode}
\noindent
Note that in the former example, we have removed some space above the
equation (that is automatically added by \verb+IEEEeqnarray+) in order
to have proper spacing. In the latter example we have assumed that the
\verb+theorem+-environment has been defined in the header:
\begin{latexcodeonly}
\usepackage{amsthm}
\newtheorem{theorem}{Theorem}
\end{latexcodeonly}
\index{theorem}\index{amsthm}%


\subsection{Putting the QED correctly: proof}
\label{sec:putting-qed-right}

\index{amsthm}\index{QED!proof}\index{proof}%
The package \verb+amsthm+ that we have used in
Section~\ref{sec:more-fancy-frames} to generate a theorem actually
also defines a \verb+proof+-environment:
\begin{latexcode}
\begin{proof}
  This is the proof of some 
  theorem. Once the proof is
  finished, a white box is put
  at the end to denote QED.
\end{proof}
\end{latexcode}
\noindent
The QED-symbol should be put on the last line of the proof. However,
if the last line is an equation, then this is done wrongly:
\begin{latexcode}
\begin{proof}
  This is a proof that ends
  with an equation: (bad)
  \begin{equation*}
    a = b + c.
  \end{equation*}
\end{proof}
\end{latexcode}
\noindent
In such a case, the QED-symbol must be put by hand using the command
\verb+\qedhere+:\index{qedhere}%
\begin{latexcode}
\begin{proof}
  This is a proof that ends
  with an equation: (correct)
  \begin{equation*}
    a = b + c. \qedhere
  \end{equation*}
\end{proof}
\end{latexcode}

Unfortunately, this correction does not work for \verb+IEEEeqnarray+:
\begin{latexcode}
\begin{proof}
  This is a proof that ends
  with an equation array: (wrong)
  \begin{IEEEeqnarray*}{rCl}
    a & = & b + c \\
    & = & d + e. \qedhere
  \end{IEEEeqnarray*}  
\end{proof}
\end{latexcode}
\noindent
The reason for this is the internal structure of \verb+IEEEeqnarray+:
it always puts two invisible columns at both sides of the array that
only contain a stretchable space. Thereby, \verb+IEEEeqnarray+ ensures
that the equation array is horizontally centered. The
\verb+\qedhere+-command should actually be put \emph{outside} this
stretchable space, but this does not happen as these columns are
invisible to the user.

Luckily, there is a very simple remedy: We explicitly define these
stretching columns ourselves!\index{qedhere!IEEEeqnarray}
\begin{latexcode}
\begin{proof}
  This is a proof that ends
  with an equation array: (correct)
  \begin{IEEEeqnarray*}{+rCl+x*}
    a & = & b + c \\
    & = & d + e. & \qedhere
  \end{IEEEeqnarray*}  
\end{proof}
\end{latexcode}
\noindent
Here, the \verb=+= in \verb={+rCl+x*}= denotes a stretchable space,
one on the left of the equations (which, if not specified, will be
done automatically by \verb+IEEEeqnarray+) and one on the right of the
equations.  But now on the right, \emph{after} the stretching column,
we add an empty column \verb=x=. This column will only be needed on
the last line for putting the \verb+\qedhere+-command.  Finally, we
specify a \verb=*=. This is a null-space that prevents
\verb+IEEEeqnarray+ to add another unwanted \verb=+=-space.

In case of a numbered equation, we have a similar problem. If you
compare
\begin{latexcode}
\begin{proof}
  This is a proof that ends with 
  a numbered equation: (bad)
  \begin{equation}
    a = b + c.
  \end{equation}
\end{proof}
\end{latexcode}
\noindent
with
\begin{latexcode}
\begin{proof}
  This is a proof that ends with 
  a numbered equation: (better)
  \begin{equation}
    a = b + c. \qedhere
  \end{equation}
\end{proof}
\end{latexcode}
\noindent
you notice that in the (better) second version the $\Box$ is much
closer to the equation than in the first version.

Similarly, the correct way of putting the QED-symbol at the end of an
equation array is as follows:
\begin{latexcode}
\begin{proof}
  This is a proof that ends
  with an equation array: (correct)
  \begin{IEEEeqnarray}{rCl+x*}
    a & = & b + c \\
    & = & d + e. \label{eq:star}
    \\* &&& \qedhere\nonumber
  \end{IEEEeqnarray}  
\end{proof}
\end{latexcode}
\noindent
which contrasts with the poorer version:
\begin{latexcode}
\begin{proof}
  This is a proof that ends
  with an equation array: (bad)
  \begin{IEEEeqnarray}{rCl}
    a & = & b + c \\
    & = & d + e.
  \end{IEEEeqnarray}  
\end{proof}
\end{latexcode}
\noindent
Note that we use a starred line-break in \eqref{eq:star} to prevent a
page-break just before the QED-sign.

We would like to point out that \verb+equation+ does not handle the
\verb+\qedhere+-command correctly in all cases. Consider the following
example:
\begin{latexcode}
\begin{proof}
  This is a bad example for the 
  usage of \verb+\qedhere+ in
  combination with \verb+equation+:
  \begin{equation}
    a = \sum_{\substack{x_i\\
        |x_i|>0}} f(x_i). 
    \qedhere
  \end{equation}
\end{proof}
\end{latexcode}
\noindent
A much better solution can be achieved with \verb+IEEEeqnarray+:
\begin{latexcode}
\begin{proof}
  This is the corrected example
  using \verb+IEEEeqnarray+:
  \begin{IEEEeqnarray}{c+x*}
    a = \sum_{\substack{x_i\\
        |x_i|>0}} f(x_i).
    \\* & \qedhere\nonumber
  \end{IEEEeqnarray}  
\end{proof}
\end{latexcode}
\noindent
Here, we add an additional line to the equation array without number,
and put \verb+\qedhere+ command there (within the additional empty
column on the right).  Note how the $\Box$ in the bad example is far
too close the equation number and is actually inside the mathematical
expression.  A similar problem also occurs in the case of no equation
number.

\needspace{6\baselineskip}
Hence:\index{qedhere!equation}
\begin{whitebox}
  \centering
  We recommend not to use \verb+\qedhere+ in combination with
  \verb+equation+, but exclusively with \verb+IEEEeqnarray+.
\end{whitebox}


\subsection{Putting the QED correctly: IEEEproof}
\label{sec:putting-qed-right2}

\index{QED!IEEEproof}\index{IEEEproof}%
\verb+IEEEtrantools+ also provides its own proof-environment that is
slightly more flexible than the \verb+proof+ of \verb+amsthm+:
\verb+IEEEproof+. Note that under the \verb+IEEEtran+-class,
\verb+amsthm+ is not permitted and therefore \verb+proof+ is not
defined, i.e., one must use \verb+IEEEproof+.

\verb+IEEEproof+ offers the command \verb+\IEEEQEDhere+ that produces
the QED-symbol right at the place where it is invoked and will switch
off the QED-symbol at the end.\index{IEEEQEDhere}
\begin{latexcode}
\begin{IEEEproof}
  This is a short proof:
  \begin{IEEEeqnarray}{rCl+x*}
    a & = & b + c \\
    & = & d+ e \label{eq:qed}
    \\* &&& \nonumber\IEEEQEDhere
  \end{IEEEeqnarray}
\end{IEEEproof}
\end{latexcode}
\noindent
So, in this sense \verb+\IEEEQEDhere+ plays the same role for
\verb+IEEEproof+ as \verb+\qedhere+ for \verb+proof+. Note, however,
that their behavior is not exactly equivalent: \verb+\IEEEQEDhere+
always puts the QED-symbol \emph{right at the place} it is invoked and
does not move it to the end of the line. So, for example, inside of a
list, an additional \verb+\hfill+ is needed:
\begin{latexcode}
\begin{IEEEproof}
  A proof containing a list and 
  two QED-symbols:
  \begin{enumerate}
  \item Fact one.\IEEEQEDhere
  \item Fact two.\hfill\IEEEQEDhere
  \end{enumerate}
\end{IEEEproof}
\end{latexcode}
\noindent
Unfortunately, \verb+\hfill+ will not work inside an \verb+equation+.
To get the behavior of \verb+\qedhere+ there, one must use
\verb+\IEEEQEDhereeqn+ instead:\index{IEEEQEDhereeqn}
\begin{latexcode}
\begin{IEEEproof}
  Placed directly behind math:
  \begin{equation*}
    a = b + c. \hfill\IEEEQEDhere
  \end{equation*}
  Moved to the end of line:
  \begin{equation*}
    a = b + c. \IEEEQEDhereeqn
  \end{equation*}
\end{IEEEproof}
\end{latexcode}
\noindent
\verb+\IEEEQEDhereeqn+ even works in situations with equation numbers,
however, in contrast to \verb+\qedhere+ it does not move the
QED-symbol to the next line, but puts it in front of the number:
\begin{latexcode}
\begin{IEEEproof}
  Placed directly before the
  equation number:
  \begin{equation}
    a = b + c. \IEEEQEDhereeqn
  \end{equation}
  With some additional spacing:
  \begin{equation}
    a = b + c. \IEEEQEDhereeqn\;
  \end{equation}
\end{IEEEproof}
\end{latexcode}
\noindent
To get the behavior where the QED-symbol is moved to the next line,
use the approach based on \verb+IEEEeqnarray+ as shown in
\eqref{eq:qed}. 

\needspace{6\baselineskip}
Once again:\index{IEEEQEDhereeqn}
\begin{whitebox}
  \centering
  We recommend not to use \verb+\IEEEQEDhere+ and
  \verb+\IEEEQEDhereeqn+ in combination with \verb+equation+, but to
  rely on \verb+\IEEEQEDhere+ and \verb+IEEEeqnarray+ exclusively.
\end{whitebox}


\index{IEEEQEDoff}\index{IEEEQEDclosed}\index{IEEEQEDopen}%
Furthermore, \verb+IEEEproof+ offers the command \verb+\IEEEQEDoff+ to
suppress the QED-symbol completely; it allows to change the QED-symbol
to be an open box as in Section~\ref{sec:putting-qed-right}; and it
allows to adapt the indentation of the proof header (default value is
\verb+2\parindent+). The latter two features are shown in the
following example:
\begin{latexcode}
\renewcommand{\IEEEproofindentspace}{0em}
\renewcommand{\IEEEQED}{\IEEEQEDopen}
\begin{IEEEproof}
  Proof without 
  indentation and an
  open QED-symbol. 
\end{IEEEproof}
\end{latexcode}
\noindent
The default QED-symbol can be reactivated again by redefining
\verb+\IEEEQED+ to be \verb+\IEEEQ+ \verb+EDclosed+.

We end this section by pointing out that IEEE standards do not allow a
QED-symbol and an equation put onto the same line. Instead one should
follow the example \eqref{eq:qed}.


\subsection{Double-column equations in a two-column layout}
\label{sec:double-column-equat}

\index{double-column equations}%
Many scientific publications are in a two-column layout in order to
save space. This means that the available width for the equations is
considerably smaller than for a one-column layout and will cause
correspondingly more line-breaks. Then the advantages of the
\verb+IEEEeqnarray+-environment are even more pronounced.

However, there are very rare situations when the breaking of an
equation into two or more lines will result in a very poor
typesetting, even if \verb+IEEEeqnarray+ with all its tricks is used.
In such a case, a possible solution is to span an equation over both
columns. But the reader be warned:
\begin{whitebox}
  \centering
  Unless there is no other solution, we strongly discourage from the
  usage of double-column equations in a two-column layout for
  aesthetic reasons and because the \LaTeX{} code is rather ugly!
\end{whitebox}

\index{figure}\index{equation!floating}%
The trick is to use the \verb+figure+-environment to create a floating
object containing the equation similarly to a included
graphic. Concretely, we have to use \verb+figure*+ to create a float
that stretches over both columns. Since in this way the object becomes
floating, the equation numbering has to be carefully taken care of.

We explain the details using an example. We start by defining two
auxiliary equation counters:\index{storeeqcounter}\index{tempeqcounter}%
\newcounter{storeeqcounter}
\newcounter{tempeqcounter}
\begin{latexcodeonly}
\newcounter{storeeqcounter}
\newcounter{tempeqcounter}
\end{latexcodeonly}
\noindent 
The counter \verb+storeeqcounter+ will store the equation number that
is assigned to the floating equation, and the counter
\verb+tempeqcounter+ will be used to restore the equation counter to
the correct number after it was temporarily set to the floating
equation's number stored in \verb+storeeqcounter+.

Note that if there are several floating equations in a document, each
needs its own unique definition of a \verb+storeeqcounter+, i.e., one
needs to introduce different names for these counters (e.g.,
\verb+storeeqcounter_one+, \verb+storeeqcounter_two+, etc.).  The
counter \verb+tempeqcounter+ can be reused for all floating equations.

Now, in the text where we will refer to the floating equation, we
need to make sure that the equation number is increased by one (i.e.,
at this place the equation numbering will jump over one number, which
is the number assigned to the floating equation), and then we need to
store this number for later use. This looks as follows:
\begin{latexcode}
\ldots and $a$ is given in
\eqref{eq:floatingeq}
%
%% Increase current equation
%% number and store it:
\addtocounter{equation}{1}%
\setcounter{storeeqcounter}%
 {\value{equation}}%
%
on the top of this page/on top of
Page~\pageref{eq:floatingeq}.
\end{latexcode}
\noindent
Note that one must manually adapt the \LaTeX{} code to either the
phrase ``on the top of this page'' or the phrase ``on top of
Page~\pageref{eq:floatingeq}'', depending on where the equation
actually appears.

Finally we typeset the floating equation:
\begin{latexcodeonly}
\begin{figure*}[!t]
  \normalsize
  \setcounter{tempeqcounter}{\value{equation}} % temp store of current value
  \begin{IEEEeqnarray}{rCl}
    \setcounter{equation}{\value{storeeqcounter}} % number of this equation
    a & = & b + c + d + e + f + g + h + i + j + k + l + m + n + o + p
    \nonumber\\
    && +\> q + r + s + t + u + v + w + x + y + z + \alpha + \beta 
    + \gamma + \delta + \epsilon 
    \label{eq:floatingeq}
  \end{IEEEeqnarray}
  \setcounter{equation}{\value{tempeqcounter}} % restore correct value
  \hrulefill
  \vspace*{4pt}
\end{figure*}
\end{latexcodeonly}
\noindent
The exact location of this definition depends strongly on where the
floating structure should be appear, i.e., it might have to be placed
quite far away from the text where the equation is referred
to.\footnote{It needs to be placed \emph{after} the reference in the
  text, though, as otherwise the equation number stored in
  \texttt{storeeqcounter} is not defined yet. This could again be
  fixed, but only if we set the equation number (i.e.,
  \texttt{storeeqcounter}) manually (ugly!!).} Note that this might
need some trial and error, particularly if there are other floating
objects around to be placed by \LaTeX{}.

Be aware that due to a limitation of \LaTeX{}, double-column floating
objects cannot be placed at the bottom of pages, i.e.,
\verb+\begin{figure*}[!b]+ will not work correctly. This can be
corrected if we include the following line in the header of our
document:\index{stfloats}
\begin{latexcodeonly}
\usepackage{stfloats}
\end{latexcodeonly}
\noindent
\index{fixltx2e}%
However, this package is very invasive and might cause troubles with
other packages.\footnote{In particular, it cannot be used together
  with the package \texttt{fixltx2e.sty}. Luckily, the latter is not
  needed anymore starting with TeXLive 2015.}

\begin{figure*}[!t]
  \normalsize
  \setcounter{tempeqcounter}{\value{equation}} % temp store of current value
  \begin{IEEEeqnarray}{rCl}
    \setcounter{equation}{\value{storeeqcounter}} % number of this equation
    a & = & b + c + d + e + f + g + h + i + j + k + l + m + n + o + p
    \nonumber\\
    && +\> q + r + s + t + u + v + w + x + y + z + \alpha + \beta 
    + \gamma + \delta + \epsilon 
    \label{eq:floatingeq}
  \end{IEEEeqnarray}
  \setcounter{equation}{\value{tempeqcounter}} % restore correct value
  \hrulefill
  \vspace*{4pt}
\end{figure*}


%%%%%%%%%%%%%%%%%%%%%%%%%%%%%%%%%%%%%%%%%%%%%%%%%%%%%%%%%%%%%%%%%
\section{Emacs and IEEEeqnarray}
\label{sec:emacs-ieeeeqnarray}

\index{Emacs}%
When working with Emacs, you can ease your life by defining a few new
commands. In the \verb+dot_emacs+-file that comes together with this
document the following commands are
defined:\index{dot\_emacs}\index{.emacs}% 
\begin{itemize}
\item \textbf{Control-c i}: Insert an \verb+IEEEeqnarray+-environment
  (similar to \textbf{Control-c Control-e}) with argument
  \verb+{rCl}+.
  
\item \textbf{Control-c o}: As \textbf{Control-c i}, but the
  \mbox{*-version}.
  
\item \textbf{Control-c b}: Add a line-break at a specific place. This
  is very helpful in editing too long lines. Suppose you have typed
  the following \LaTeX{} code:
\begin{latexcode}
\begin{IEEEeqnarray}{rCl}
  a & = & b + c \\
  & = & d + e + f + g + h + i 
  + j + k + l + m + n + o
\end{IEEEeqnarray}
\end{latexcode}
\noindent
After compiling you realize that you have to break the line before
$l$. You now just have to put the cursor on the $+$-sign in front of
$l$ and press \textbf{Control-c b}. Then the line is wrapped there and
also the additional space \verb+\>+ is added at the right place:
\begin{latexcode}
\begin{IEEEeqnarray}{rCl}
  a & = & b + c \\
  & = & d + e + f + g + h + i 
  + j + k \nonumber\\
  && +\> l + m + n + o
\end{IEEEeqnarray}
\end{latexcode}
  
\item \textbf{Control-c n}: As \textbf{Control-c b}, but without
  adding the additional space \verb+\>+.

\item \textbf{Control-c Control-b}: Remove a line-break (undo of
  \textbf{Control-c b} and \textbf{Control-c n}). Position the cursor
  before the \verb+\nonumber+ and press \textbf{Control-c Control-b}.

\item \textbf{Control-c m}: Insert a
  \verb+\IEEEeqnarraymulticol+-command. This is very helpful when the
  LHS is too long. Suppose you have typed the following \LaTeX{} code:
\begin{latexcode}
\begin{IEEEeqnarray}{rCl}
  a + b + c + d + e + f 
  + g + h + i + j 
  & = & k + l \\
  & = & m + n
\end{IEEEeqnarray}
\end{latexcode}
\noindent
After compiling you realize that the LHS is too long. You now just
have to put the cursor somewhere on the first line and type
\textbf{Control-c m}.  Then you get
\begin{latexcode}
\begin{IEEEeqnarray}{rCl}
  \IEEEeqnarraymulticol{3}{l}{
    a + b + c + d + e + f 
    + g + h + i + j 
  }\nonumber \\ \quad
  & = & k + l \\
  & = & m + n
\end{IEEEeqnarray}
\end{latexcode}
  
\item Finally, in the \verb+dot_emacs+-file, settings are given that
  make \verb+IEEEeqnarray+ and \verb+IEEEeqnarraybox+ known to Emacs'
  \LaTeX{}-mode, reftex, and ispell. This way many standard Emacs
  commands can be used as usual also in the context of
  \verb+IEEEeqnarray+. For example, \textbf{Control-c (} will add an
  equation label.

\end{itemize}


%%%%%%%%%%%%%%%%%%%%%%%%%%%%%%%%%%%%%%%%%%%%%%%%%%%%%%%%%%%%%%%%%
\section{Some Useful Definitions}
\label{sec:some-usef-defin}

\index{Markov chain}\index{symbols!Markov chain}%
There are a couple of mathematical symbols that cannot easily be found
in \LaTeX{}-symbol collections. In the following, a few such symbols
are listed and a possible way is proposed of how to define them.
\begin{itemize}
\item \emph{Markov Chains:} One of the existing customary ways to
  describe that three random variables form a Markov chain is
\begin{latexcode}
\begin{equation*}
  X \markov Y \markov Z
\end{equation*}
\end{latexcode}
\noindent
Here, the symbol '$\markov$' is defined as a combination of
\verb+\multimap+ ($\multimap$) and two minus-signs
($-$):\index{markov}%
\begin{latexcodeonly}
\newcommand{\markov}{\mathrel{\multimap}\joinrel\mathrel{-}%
                     \joinrel\mathrel{\mkern-6mu}\joinrel\mathrel{-}} 
\end{latexcodeonly}
\noindent
For this definition to work, beside \verb+amsmath+ also the package
\verb+amssymb+ needs to be loaded.\index{amsmath}\index{amssymb}

\item \emph{Independence:} To describe that two random variables are
  statistically independent, I personally prefer the following symbol:
  \index{symbols!independence}\index{independence}%
\begin{latexcode}
\begin{equation*}
  X \indep Y
\end{equation*}
\end{latexcode}
\noindent
Accordingly,\index{symbols!dependence}\index{dependence}%
\begin{latexcode}
\begin{equation*}
  X \dep Y
\end{equation*}
\end{latexcode}
\noindent
denotes that $X$ and $Y$ are not statistically independent.

These two symbols are created by two \verb+\bot+ ($\bot$) signs:
\begin{latexcodeonly}
\newcommand{\indep}{\mathrel{\bot}\joinrel\mathrel{\mkern-5mu}%
                    \joinrel\mathrel{\bot}}
\newcommand{\dep}{\centernot\indep} 
\end{latexcodeonly}
\noindent
For this definition to work, beside \verb+amsmath+ also the package
\verb+centernot+ needs to be loaded.\index{centernot}\index{amsmath}%


\item \emph{Integration-$\dd$:} The $\dd$ in an integral is not a
  variable, but rather an operator. It therefore should not be typeset
  italic $d$, but Roman $\dd$. Moreover, there should be a small
  spacing before the operator:\index{symbols!integration-d}\index{dd}%
  \index{integration-d}%
\begin{latexcode}
\begin{equation*}
  \int_a^b f(x) \dd x = \int_a^b 
  \ln\left(\frac{x}{2}\right) 
  \dd x 
\end{equation*}
\end{latexcode}
\noindent
To make sure that this spacing always works out correctly, I recommend
the following definition:
\begin{latexcodeonly}
\newcommand{\dd}{\mathop{}\!\mathrm{d}}
\end{latexcodeonly}

\end{itemize}




%%%%%%%%%%%%%%%%%%%%%%%%%%%%%%%%%%%%%%%%%%%%%%%%%%%%%%%%%%%%%%%%%
\section{Some Final Remarks and Acknowledgments}
\label{sec:some-final-remarks}

The ``rules'' stated in this document are purely based on my own
experience with typesetting \LaTeX{} in my publications and on my ---
some people might say unfortunate --- habit of incorporating many
mathematical expressions in there.

If you encounter any situation that seems to contradict the
suggestions of this document, then I would be very happy if you could
send me a corresponding \LaTeX{} or PDF file.  As a matter of fact,
any kind of feedback, criticism, suggestion, etc.\ is highly
appreciated!  Write to
\begin{center}
  \includegraphics[height=\baselineskip]{email}
\end{center}
Thanks!

I would like to mention that during the writing and updating of this
document I profited tremendously from the help of Michael Shell, the
author of \verb+IEEEtran+. He was always available for explanations
when I got stuck somewhere. Moreover, I gratefully acknowledge the
comments from (in alphabetical order) Helmut B\"olcskei, Amos
Lapidoth, Edward Ross, Omar Scaglione, and Sergio Verd\'u.

\vspace{0.7cm} 
\hfill Stefan M.~Moser


\vspace{15mm}
\printindex


\end{document}

%%% Local Variables: 
%%% mode: latex
%%% TeX-master: t
%%% End: 
